\documentclass{uonmathreport}

% loads already mathtools, graphicx,
% but you may want to add other packages, like
\usepackage{listings} % to include code in python see https://en.wikibooks.org/wiki/LaTeX/Source_Code_Listings
\usepackage{xcolor}
\usepackage{biblatex}
\addbibresource{progress.bib}
% other useful pagackes include booktabs, hyperref, amsthm, xcolor, todonotes, showkeys, ...
% see https://www.overleaf.com/learn 

% change the following to
% to \PJA = MATH4041, \PJS = MATH4042 or \DIS = MATH4001 (for BSc and MMAth)
% or \MSc (for all Msc dissertations)
\PJA

% adjust the following
\title{Title of the report goes here\\ and it may have several lines}
\author{Alvin Jonel De la Cruz Guerrero}
\academicyear{2022/23}
\supervisor{Dr. Your Supervisor}

% the following are irrelevant for Msc:
\assessmenttype{Review} % or Investigation

% the following are irrelevant for PJS, PJA, DIS:
% Msc: change it to G14PMD and Pure Mathematics, etc ...
\msccode{MATH4021}
\msctitle{Statistics}

% put your own definitions and shorthands here
\newcommand{\ZZ}{\mathbb{Z}}

\begin{document}

\maketitle

\begin{abstract}
The abstract of the report goes here. The abstract should state the
topic(s) under investigation and the main results or
conclusions. Methods or approaches should be stated if this is
appropriate for the topic. The abstract should be self-contained,
concise and clear. The typical length is one paragraph.
\end{abstract}

% Table of contents
\setcounter{tocdepth}{3}  % this will list subsections, but not subsubsections
\tableofcontents 
\newpage
% this is a comment in the file that won't appear in the output


\section{Introduction}

\subsection{Overview}

Options are equity-based derivatives that are primarily used to mitigate risk. The options market is significantly larger compared to other derivatives. In fact, options were the most traded derivatives in 2019, with a volume of 18.55 billion contracts when combining index and individual equities contracts \cite{statista_2019}. An enormous market like the options market demands reliable pricing mechanisms to minimize arbitrage opportunities. Naturally, pricing American options is a broad research area because there is no closed-form solution to the PDE resulting from the Black-Scholes model. Merton \cite{merton_1973} was the first to consider the Black-Scholes model proposed by Black and Scholes \cite{black_scholes_1973} to price European and American options. Although Merton derived a nice formula to price European options, he stated that, in general, a closed-form solution was not attainable for American options. In 1977, Schwartz \cite{schwartz_197779} and Brennan \cite{brennan_1997} proposed using finite difference schemes to solve the pricing problem for American options. The work of Merton and Schwartz served as a foundation for the free boundary problem for American options pricing. The motivation behind the free boundary problem formulation is that for American options, there exists an optimal exercise price that marks the boundary between the region where exercising the option is profitable and the region where it is not. Moreover, this optimal exercise price changes with time, making it impossible for the holder to determine when to exercise. Based on the work of Landau \cite{landau_1950_heat_ci}, Wu et al. \cite{wu1997front} formulated the front-fixing method as an approach to solve the free boundary problem for options, in which the Landau transformation is used to transform the optimal exercise price or the moving boundary into a fixed boundary. Multiple transformations have been proposed by Huang et al. \cite{huang_2000}, Nielsen et al. \cite{nielsen_2001}, and Company et al. \cite{company_egorova_jodar_2014}. Around the same period, Dewynne et al. \cite{dewynne_howison_rupf_wilmott_1993} took a different approach to solve the pricing problem. The idea was to reformulate the free boundary problem as a problem with variational inequalities that, when finite difference is applied, transforms into a linear complementary problem \cite{dantzig_1968}.

\subsection{Aim} 

The goal of this work is to implement the numerical methods proposed by Nielsen et al. \cite{nielsen_2001} and Company et al. \cite{company_egorova_jodar_2014} to solve the free boundary formulation of the pricing problem for American options \cite{dewynne_howison_rupf_wilmott_1993}. Moreover, Company et al. \cite{company_egorova_jodar_2014} and Nielsen et al. \cite{nielsen_2001} proposed schemes for pricing American put options with underlying assets that have non-paying dividends. However, we aim to derive analogous schemes for pricing call contracts as well, considering underlying assets that have a continuous dividend yield, such as the S\&P500. Additionally, we want to conduct a convergence analysis of each of the methods. Furthermore, we consider the linear complementary formulation proposed by Dewynne \cite{dewynne_howison_rupf_wilmott_1993} \cite{wilmott_howison_dewynne_1995}. Finally, we perform a convergence analysis of each of the methods implemented.

\subsection{Main Achievements}

In this work, we were able to derive the corresponding PDE problem resulting from applying the transformations proposed by Nielsen et al. \cite{nielsen_2001} and Company et al. \cite{company_egorova_jodar_2014} for call and put options with underlying assets featuring a continuous dividend yield. Moreover, we implemented explicit and implicit schemes for Nielsen et al.'s method \cite{nielsen_2001}, an explicit scheme for Company et al.'s method \cite{company_egorova_jodar_2014}, and the theta method for the linear complementary problem proposed by \cite{wilmott_howison_dewynne_1995} using Python. Finally, we derived the order of convergence for each of the implemented methods.

\subsection{Outline}

The outline of this paper is as follows. In Section 2, we explore the Black-Scholes model for American options for assets that pay dividends, resulting in the free boundary formulation of the pricing problem. Furthermore, we delve into the front-fixing method as a strategy for fixing the moving boundary by applying a change of variable. We consider the changes of variables proposed by Company et al. \cite{company_egorova_jodar_2014} and Nielsen et al. \cite{nielsen_2001}. In Section 3, we explore explicit and implicit schemes to solve the partial differential equations resulting from applying the front-fixing method and the Nielsen transformation to the free boundary problem obtained in Section 1. We conclude Section 3 with numerical experiments and convergence analysis for the numerical schemes presented based on the work of Nielsen and Company (see Appendix \ref{sec:company_explicit_scheme}). In Section 5, we explore a reformulation of the pricing problem as a variational inequality and the linear complementary system of equations resulting from it. Additionally, we delve into the theta method, which is a more general numerical method that yields explicit, implicit, and Crank-Nicholson schemes. Finally, in the same section, we discuss the results of solving the linear complementary problem reformulation of pricing American options using the theta method.

\section{Black-Scholes model} \label{sec:blackscholes}

\subsection{Preliminaries}

A common problem in finance is pricing financial derivatives, often referred to simply as derivatives. In essence, derivatives are contracts set between parties whose value over time derives from the price of their underlying assets. A notorious family of derivatives in financial markets are "options". Options are contracts set between two parties in which the holder has the right to sell or buy, commonly referred to as exercising, an underlying asset at a pre-established price, also known as the "strike price", in the future. Options are referred to as "call options" or "put options" if the exercise position is to buy or to sell, respectively. Similarly, options are classified depending on their exercise style. In that regard, the simplest options are European options. European options give the right to exercise at the expiration date of the contract. Another well-known type of option is the American option. American options work similarly to European options, with the difference that they can be exercised at any point in time between the beginning and the expiration date of the contract. Obviously, American and European options are almost similar, and they only differ in at the times which the holder can exercise them. Therefore, we will start describing the pricing problem from the European options perspective and extended to the case of American options.

Let us define the payoff function of European option as
\begin{subequations}
  \begin{align}
    \text{\textbf{Call:}} \qquad H_{\text{Eur}}(S) = \max(S - K, 0) \\
    \text{\textbf{Put:}} \qquad H_{\text{Eur}}(S) = \max(K - S, 0)
  \end{align}
  \label{eq:blackscholes:preliminaries:european_put_payoff}
\end{subequations}
where $K$ is the strike price and remains constant through lifespan of the option,
$S \in [0, \infty]$ is the asset price at the maturity date. We can extend the European payoff to American options by introducing
the time axis to equation above.
\begin{subequations}
  \begin{align}
    \text{\textbf{Call:}} \qquad H(S, t) = \max(S - K, 0) \\
    \text{\textbf{Put:}} \qquad H(S,t) = \max(K - S, 0)
  \end{align}
  \label{eq:blackscholes:preliminaries:american_put_payoff}
\end{subequations}
The interval $[0, T]$ represent the lifespan of option where $T$ is the elapsed time, in years, since the starting and expiration date of the contract. For example, $T=1$ indicates that the contract expires one year after starting date. Moreover, $t\in[0, T]$ is the elapsed time, in years, since the starting date of the contract. While the payoff of European options is defined only at $t=T$, American option's payoff is defined for all $(S, t) \in [0, \infty]\times[0, T]$.

Options provide greater flexibility to holders by eliminating their exposure to negative payoffs. Therefore, writers charge premiums to the holders to acquire the contract. The premium is often referred to as the price or value of the option, and the problem of determining this value is called option pricing. Let $V_t$ represent the value of the option at time $t$. For instance, $V_0$ represents the value of the option at the beginning of the contract. When pricing options, it is crucial to find the fair price; otherwise, the writer or holder of the option could devise a scheme in which the option will always be profitable for them. In other words, options pricing must adhere to the principle of no-arbitrage. Therefore, we assume that the writer of the option uses the premium to construct a portfolio consisting of $\phi_0$ units of the asset and invests $\psi_0$ units of cash in a risk-free asset $B_t$, such as a US Treasury bills, certificates of deposit, or a bank account. Then, the writer rebalances the portfolio $(\phi_0, \psi_0)$ to hedge against any potential claims from the holder at any future time $0 < t \le T$. Consequently, at any time $t$, the writer holds a portfolio $(\phi_t, \psi_t)$ with a value
\begin{equation*}
  \Pi_t = \phi_t S_t + \psi_t B_t
\end{equation*}
Moreover, the portfolio is self-financing. In other words, the changes in
portfolio depend on the changes in $S_t$ and $B_t$,
and the rebalancing of portfolio $(\phi_t, \psi_t)$
\begin{align*}
   & d\Pi_t = \phi_tdS_t + \psi_t dB_t \\
   & S_t d\phi_t + B_t d\psi_t = 0
\end{align*}
Finally, the portfolio value matches the option value
\begin{equation*}
  \Pi_t = V_t
\end{equation*}
at any time $0 \le t \le T$. Using the self-financing portfolio hedging strategy, The Black-Scholes model presents a mathematical model for the dynamics of an option's price. The model makes certain assumptions about the market. A complete list of all the assumptions can be found in \cite{merton_1973} and \cite{wilmott_howison_dewynne_1995}. In the next part, we enumerate some them. Firstly, the asset price $S_t$ is distributed as a log-normally
\begin{equation}
  \label{eq:blackscholes:preliminaries:asset_price}
  S_t = S_0 \exp\bigg\{\int_{0}^{t} \big(r(s) - \dfrac{1}{2}\sigma(s)\big)ds + \sqrt{t}Z\bigg\}
\end{equation}
where the risk-free interest rate $r(t)$ and the price volatility $\sigma(t)$ are deterministic functions of time during the life of the option. Secondly, the bank account $B(t)$ is a deterministic function
\begin{equation*}
  dB = r(t)B(t)dt
\end{equation*}
Finally, the asset does not pay dividends. Additionally, we will assume that
the risk-free interest rate and asset price volatility are constant during the
life of the option. Later on, we will address the assumption about dividends.

By applying the Black-Scholes model to price European options, the famous
Black-Scholes PDE is obtained
\begin{equation*}
  \label{eq:blackscholes:preliminaries:european_option_pde}
  \begin{cases}
    \dfrac{\partial{V}}{\partial{t}} + \dfrac{1}{2}\sigma^{2} S^2 \dfrac{\partial^2{V}}{\partial{S^2}} + r S \dfrac{\partial{V}}{\partial{S}} - rV = 0 & \text{for $t\in[0,T)$ and $S\in[0, \infty)$} \\ V(S, T) =
    H(S, T) & \text{for $S\in[0, \infty)$}
  \end{cases}
\end{equation*}
where $V(S, t)$ is a deterministic function. A derivation of the Black-Scholes PDE for European options can be found in \cite{merton_1973} and \cite{wilmott_howison_dewynne_1995}. We previously mentioned that Black-Scholes model assumes that the underlying asset does not pay dividends. In most cases, assets such as stocks pay out dividends just a few times at year. In this case, dividends are to be modelled discretely. However, there are certain assets that pay out a proportion of the current price during and interval of time. For instance, indexes such as the SPX. Thus, in such cases, it is useful to model dividends as a continuous yield. \cite{wilmott_howison_dewynne_1995} shows a slight adjusted asset price model that includes continuous yield dividends
\begin{equation}
  \label{eq:blackscholes:preliminaries:dividends_asset_price}
  S_t = S_0 \exp\bigg\{\int_{0}^{t} \big(r(s) - \delta(s) - \dfrac{1}{2}\sigma(s)\big)ds + \sqrt{t}Z\bigg\}
\end{equation}
Note that when the asset does not pay out dividends $\delta(t) = 0$, the asset price model will be exactly as in \eqref{eq:blackscholes:preliminaries:asset_price}. Similarly to as we did for the risk-free interest rate and the volatility of the asset price, we will assume that continuous dividends yield is as constant from now on. As a consequence of
the asset price mode \eqref{eq:blackscholes:preliminaries:dividends_asset_price}, the Black-Scholes PDE changes to 
\begin{equation}
  \label{eq:chapter2:european_option_pde_with_dividens}
  \begin{cases}
    \dfrac{\partial{V}}{\partial{t}} + \mathcal{L}_{\text{BS}}(V) = 0 & \text{for $t\in[0,T)$ and $S\in[0, \infty)$} \\ 
    V(S, T) = H(S,T) & \text{for $S\in[0, \infty)$}
  \end{cases}
\end{equation}
where $\mathcal{L}_{\text{BS}}(f)(x)$ is the linear parabolic operator applied to the function $f \in \mathcal{C}^2$
\begin{equation}
  \label{eq:blackscholes:preliminaries:linear_parabolic_operator}
  \mathcal{L}_{\text{BS}}(f)(x) := \dfrac{1}{2}\sigma^{2} x^2 \dfrac{\partial^2{f}}{\partial{x^2}} + (r - \delta) x \dfrac{\partial{f}}{\partial{x}} - rf(x)
\end{equation}
Note that for conciseness, we write $\mathcal{L}_\text{BS}(V)(S)$ as $\mathcal{L}_\text{BS}(V)$. Similarly, to the asset price model with dividends, the Black-Scholes PDE with dividends fall back to the original Black-Scholes PDE if the asset does not pay out dividends $\delta = 0$. So far, we have presented an equation that describes the dynamics of the value $V(S,t)$ of European options. So now, we will consider the pricing of American options. By applying the Black-Scholes model to price American options, Merton \cite*{merton_1973} derives some important facts. Firstly, the value $V(S,t)$ is bounded from below by the payoff function:
\begin{align}
  \label{eq:blackscholes:american_options_price_lower_bound}
  V(S, t) \ge H(S, t) \qquad \text{for $t \in [0, T]$}
\end{align}
Moreover, the domain of $V(S, t)$ can be separated into the exercise region in 
\begin{equation}
  \mathcal{S} := \{(S, t) : V(S, t) = H(S, t)\}
  \label{eq:blackscholes:preliminaries:exercise_region}
\end{equation}
in which it is profitable for the holder to exercise the option, the continuation region
\begin{equation}
  \label{eq:blackscholes:preliminaries:continuation_region}
  \mathcal{C} := \{(S, t) : V(S, t) > H(S, t)\}
\end{equation} 
in which it is preferable to continue holding the option because exercising is not profitable, and the optimal exercise boundary that separates the continuation region and exercise region
\begin{equation}
  \label{eq:blackscholes:preliminaries:optimal_exercise_boundary}
  \partial \mathcal{C} := \{(S, t) : S = \bar{S}(t)\}
\end{equation}
where $\bar{S}(t)$ is the optimal exercise price. 
\begin{figure}[H]
  \centering
  \begin{subfigure}{0.4\textwidth}
    \centering
    \includegraphics[width=\textwidth]{chapters/chapter2/AmericanCallOptionValue}
    \caption{Call option}
    \label{fig:blackscholes:preliminaries:american_call_value_vs_curve}
  \end{subfigure}
  \hfill
  \begin{subfigure}{0.4\textwidth}
    \centering
    \includegraphics[width=\textwidth]{chapters/chapter2/AmericanPutOptionValue.pdf}
    \caption{Put option}
    \label{fig:blackscholes:preliminaries:american_put_value_vs_curve}
  \end{subfigure}
  \caption{Value $V(S, t)$ of American option value curve. }
  \label{fig:blackscholes:preliminaries:american_option_value_vs_curve}
\end{figure}
Lastly, the price dynamics of
American options is governed by the same Black-Scholes PDE as European options in the continuation region. Pricing American options only requires solving $V(S,t)$ at continuation region and finding the optimal exercise price that serves as a boundary of continuation region. 
\begin{align}
  \begin{cases}
    \dfrac{\partial{V}}{\partial{t}} + \dfrac{1}{2}\sigma^{2} S^2 \dfrac{\partial^2{V}}{\partial{S^2}} + (r - \delta)S \dfrac{\partial{V}}{\partial{S}} - rV = 0 & \text{for $(S, t) \in \mathcal{C}$} \\ V(S, t) = H(S, t) &
    \text{for $(S,t)\in \partial\mathcal{C}$}
  \end{cases}
  \label{eq:blackscholes:preliminaries:american_options_pde_free_boundary_problem}
\end{align}
The problem above is also known as the free boundary formulation of the pricing problem because it requires to solve a PDE with a moving boundary. The terminal
condition of the PDE will be given at time $T$. At the expiration date of the contract, the holder will either exercise or not the option. Therefore, the value of the option will be equal to the payoff function. Obviously, in that case, the optimal exercise price $S(T)$ will be equal to the strike price $K$. Hence,
\begin{align}
  V(S,T) = H(S, T), \qquad \bar{S}(T) = K
  \label{eq:blackscholes:preliminaries:american_options_terminal_condition}
\end{align}
Next, we need to establish boundary conditions for the system
\eqref{eq:blackscholes:preliminaries:american_options_pde_free_boundary_problem}.
Generally, when pricing options, we need two boundaries conditions. 
As it can be observed in figure (\ref{fig:blackscholes:preliminaries:american_option_value_vs_curve}), for call options, the left boundary condition $V(0,t)=0$ is given at $S=0$ and the right boundary conditions $V(\bar{S}(t),t)=\bar{S}(t) - K$ is given at the optimal exercise price $\bar{S}(t)$. Analogously, for put options, the left boundary condition $V(\bar{S}(t),t)=K - \bar{S}(t)$ at $\bar{S}(t)$ and right boundary is $V(S, t)=0$ for an arbitrary large $S$. Finally, $V(S,t)$ touches the payoff $H(S,t)$ tangentially at the optimal exercise price $\bar{S}(t)$
\begin{subequations} \label{eq:blackscholes:preliminaries:smooth_passing_condition}
  \begin{align}
    \text{\textbf{Call:}} \qquad &\dfrac{\partial{V}}{\partial{S}}(\bar{S}(t), t) = 1\\
    \text{\textbf{Put:}} \qquad &\dfrac{\partial{V}}{\partial{S}}(\bar{S}(t), t) = -1
  \end{align}
\end{subequations}
which is called the smooth pasting condition and later on will help us in obtaining $\bar{S}(t)$. By grouping \eqref{eq:blackscholes:preliminaries:american_options_pde_free_boundary_problem}, \eqref{eq:blackscholes:preliminaries:american_options_terminal_condition} and \eqref{eq:blackscholes:preliminaries:smooth_passing_condition} in one equation,
we obtain the system

\begin{subequations} \label{eq:blackscholes:preliminaries:american_options_pde_free_boundary_problem_full}
  \begin{align}
    \text{\textbf{Call}:} \quad &
    \begin{cases}
      \dfrac{\partial{V}}{\partial{t}} + \dfrac{1}{2}\sigma^{2} S^2 \dfrac{\partial^2{V}}{\partial{S}^2} + (r - \delta)S\dfrac{\partial{V}}{\partial{S}} - rV = 0 & \text{for $S \in (0,\bar{S}(t))$ and $t \in [0, T)$} \\ 
      V(S, T) = S - K \\
      \bar{S}(T) = K \\ 
      V(0, t) = 0 \\
      \dfrac{\partial{V}}{\partial{S}}(\bar{S}(t), t) = 1
    \end{cases} \\
    \text{\textbf{Put}:} \quad &
    \begin{cases}
      \dfrac{\partial{V}}{\partial{t}} + \dfrac{1}{2}\sigma^{2} S^2 \dfrac{\partial^2{V}}{\partial{S}^2} + (r - \delta)S\dfrac{\partial{V}}{\partial{S}} - rV = 0 & \text{for $S \in (\bar{S}(t), \infty)$, and $t \in [0, T)$} \\
      V(S, T) = K - S \\
      \bar{S}(T) = K \\ 
      \lim_{S\rightarrow\infty}V(S, t) = 0 \\ 
      \dfrac{\partial{V}}{\partial{S}}(\bar{S}(t), t) = -1
    \end{cases}
  \end{align}
\end{subequations}
\subsection{Front-Fixing method}
In the previous section, we presented the pricing of American options problem.
By applying the Black-Scholes model, we derived the Black-Scholes PDE that
describes the price dynamics in the continuation region $\mathcal{C}$ of call
and put options. Moreover, we presented the moving boundary condition
$\bar{S}(t)$ for this PDE. The moving boundary condition $\bar{S}(t)$ makes the
Black-Scholes PDE more involved since we also need to determine this boundary
as time changes. This type of problems are known as free boundary problems. The
front fixing method is a strategy in which a transformation is used to map the  domain from the original problem to a new domain where moving boundary remains fixed as time changes. In this section, we explore two transformation based on the work of Nielsen et al. \cite{nielsen_2001}, and the work of Company and et al. \cite{company_egorova_jodar_2014}.
\subsubsection{Nielsen transformation} \label{sec:blackscholes:frontfixingmethod:inversetransform}
The Nielsen transformation suggests a really simple transformation in which
the asset price $S$ is divided by the optimal exercise price $\bar{S}$
\begin{equation}
  x = \dfrac{S}{\bar{S}(t)}
  \label{eq:blackscholes:frontfixingmethod:inversetransform}
\end{equation}
Clearly, the moving boundary in the original problem will be fixed when $S=\bar{S}(t)$ at $x=1$. Now, we define $v(x,t)$ as the value function of the option but under the front fixing domain given by $x$
\begin{equation}
  v(x, t) := V(S, t)
  \label{eq:blackscholes:frontfixingmethod:inversetransform:value_function}
\end{equation}
Moreover, we want to understand how this transformation affects the Black-Scholes PDE, the boundary, terminal and contact point conditions given in equation in \eqref{eq:blackscholes:preliminaries:american_options_pde_free_boundary_problem_full}.

Firstly, we start with the Black-Scholes PDE which is defined at the interval 
$S\in(0, \bar{S}(t))$ for call options or the open interval $S\in(\bar{S}(t), \infty)$ for put options. Under the front fixing domain, the transformed PDE will be defined in the interval $x\in(0, 1)$ for call and $x\in(1, \infty)$ for put. Moreover, we apply the chain rule to rewrite the Black-Scholes PDE in terms of $v$, so that, the new PDE is given as
\begin{subequations} \label{eq:blackscholes:frontfixingmethod:american_options_pde}
  \begin{align}
    \text{\textbf{Call:}} \qquad
    \dfrac{\partial{v}}{\partial{t}} + \dfrac{1}{2}\sigma^{2} x^2 \dfrac{\partial^2{v}}{\partial{x}^2} + \bigg[(r - \delta) - \dfrac{\bar{S}^\prime(t)}{\bar{S}(t)}\bigg]x\dfrac{\partial{v}}{\partial{x}} - rv = 0 \quad & \text{for $x \in [0, 1)$ and $t \in [0, T)$} \\
    \text{\textbf{Put:}} \qquad
    \dfrac{\partial{v}}{\partial{t}} + \dfrac{1}{2}\sigma^{2} x^2 \dfrac{\partial^2{v}}{\partial{x}^2} + \bigg[(r - \delta) - \dfrac{\bar{S}^\prime(t)}{\bar{S}(t)}\bigg]x\dfrac{\partial{v}}{\partial{x}} - rv = 0 \quad & \text{for $x > 1$ and $t \in (0, T]$}
  \end{align}
\end{subequations}
Similarly, we express the boundary conditions in the front fixing domain. We already stated that the Nielsen transformation fixes the moving boundary $\bar{S}$ at $x=1$. Additionally, $x$ goes to infinity as $S$ goes to infinity, and $x=0$ for $S=0$. Therefore, the boundary condition opposite to the optimal exercise price $\bar{S}(t)$ will remain as in the original problem. Hence,
as it can be observed in figure (\ref{fig:blackscholes:frontfixingmethod:nielsen_value_vs_curve}), the call option has left boundary condition $v(0, t) = 0$ at $x=0$ and right boundary condition $v(1, t) = \bar{S}(t) - K$ at $x=1$. Alternatively, 
the put option has left boundary condition $v(1, t) = K - \bar{S}(t)$ at $x=1$ and right boundary condition $v(x, t) = 0$ at a sufficiently large $x$.
\begin{figure}[H]
  \centering
  \begin{subfigure}{0.45\textwidth}
    \centering
    \includegraphics[width=\textwidth]{chapters/chapter2/NielsenCallOption}
    \caption{Call option}
    \label{fig:blackscholes:frontfixingmethod:nielsen_call_value_vs_curve}
  \end{subfigure}
  \begin{subfigure}{0.45\textwidth}
    \centering
    \includegraphics[width=\textwidth]{chapters/chapter2/NielsenPutOption}
    \caption{Put option}
    \label{fig:blackscholes:frontfixingmethod:nielsen_put_value_vs_curve}
  \end{subfigure}
  \caption{Value $v(x,t) := V(S,t)$ in the front fixing domain defined by Nielsen transformation.}
  \label{fig:blackscholes:frontfixingmethod:nielsen_value_vs_curve}
\end{figure}
Likewise, we express the contact point condition in terms of $v(x,t)$. Recall that at the contact point, the slope $V(S,t)$ with respect to $S$ is the same as the slope of the linear segment in the payoff function. This can be seen clearly in figure (\ref{fig:blackscholes:preliminaries:american_option_value_vs_curve}). Hence, by the chain rule, the contact point condition of $v(x,t)$ is given by
\begin{subequations} \label{eq:blackscholes:frontfixingmethod:inversetransform:american_options_optimal_price_contact_point_condition}
  \begin{align}
    \text{\textbf{Call:}} \qquad & \dfrac{\partial v}{\partial x}(1, t) = \bar{S}(t) \\
    \text{\textbf{Put:}} \qquad & \dfrac{\partial v}{\partial x}(1, t) = -\bar{S}(t)
  \end{align}
\end{subequations}
Finally, recall that the terminal condition of $\bar{S}(t)$ is given by \eqref{eq:blackscholes:preliminaries:american_options_terminal_condition}. Moreover,
$x>=1$ for call options, and $x<=1$ for put options. Hence, by simple substitution, we can rewrite the terminal conditions of $v(x, t)$ as
\begin{subequations} \label{eq:blackscholes:frontfixingmethod:inversetransform:american_options_terminal_condition}
  \begin{align}
    \text{\textbf{Call:}} \qquad & v(x, T) = \max(x\bar{S}(T) - K) = K \max(x - 1, 0) = 0 \\
    \text{\textbf{Put:}} \qquad & v(x, T) = \max(K - x\bar{S}(T)) = K \max(1 - x, 0) = 0
  \end{align}
\end{subequations}
In summary, by groping equations
\eqref{eq:blackscholes:frontfixingmethod:american_options_pde},
\eqref{eq:blackscholes:frontfixingmethod:inversetransform:american_options_terminal_condition}, and
\eqref{eq:blackscholes:frontfixingmethod:inversetransform:american_options_optimal_price_contact_point_condition},
we obtain the system
\begin{subequations} \label{eq:blackscholes:frontfixingmethod:inversetransform:american_options_bs_pde}
  \begin{align}
    \text{\textbf{Call:}} \quad &
    \begin{cases}
      \dfrac{\partial{v}}{\partial{t}} + \dfrac{1}{2}\sigma^{2} x^2 \dfrac{\partial^2{v}}{\partial{x}^2} + \bigg[(r - \delta) -
      \dfrac{\bar{S}^\prime(t)}{\bar{S}(t)}\bigg]x\dfrac{\partial{v}}{\partial{x}} - rv = 0 & \text{for $x \in (0, 1)$ and $t \in [0, T)$} \\ 
      v(x, T) = 0 & \text{for $x\in[0, 1]$}  \\
      \bar{S}(T) = K \\ 
      v(0, t) = 0 & \text{for $t\in[0, T)$}\\ 
      v(1, t) = \bar{S}(t) - K & \text{for $t\in[0, T)$}\\ 
      \dfrac{\partial{v}}{\partial{x}}(1, t) = \bar{S}(t) & \text{for $t\in[0, T)$}
    \end{cases}\\
    \text{\textbf{Put:}} \quad &
    \begin{cases}
      \dfrac{\partial{v}}{\partial{t}} + \dfrac{1}{2}\sigma^{2} x^2 \dfrac{\partial^2{v}}{\partial{x}^2} + \bigg[(r - \delta) -
      \dfrac{\bar{S}^\prime(t)}{\bar{S}(t)}\bigg]x\dfrac{\partial{v}}{\partial{x}} - rv = 0 & \text{for $x > 1$ and $t \in [0, T)$} \\ 
      v(x, T) = 0 & \text{for $x \ge 1$} \\
      \bar{S}(T) = K \\ 
      v(1, t) = K - \bar{S}(t) & \text{for $t\in[0, T)$} \\
      \lim_{x\rightarrow\infty}v(x, t) = 0 & \text{for $t\in[0, T)$} \\
      \dfrac{\partial{v}}{\partial{x}}(1, t) = -\bar{S}(t) & \text{for $t\in[0, T)$}
    \end{cases}
  \end{align}
\end{subequations}
\subsubsection{Company transformation}
The Company transformation proposes set of change of variable for the asset price $S$, the time $t$, the value function $V(S,t)$ and the moving boundary
\begin{equation}
  x := \log \dfrac{S}{\bar{S}_f(t)}, \quad \tau := T - t, \quad v(x, \tau) := \dfrac{V(S, t)}{K}, \quad \bar{S}_f(\tau) := \dfrac{\bar{S}(t)}{K} 
\end{equation}
Let us break down the transformations. Firstly, the transformation proposed are written forward in time. Therefore, $\tau = 0$ refers to the expiration date of the options $t=T$. Secondly, both the value function and the optimal exercise price is scaled by the strike price. Finally, the new moving boundary is fixed at $S=\bar{S}_f(t)$ or $x=0$.

Similarly, as we did for the Nielsen method, we rewrite the Black-Scholes PDE in terms of $v(x, \tau)$. Note that as $x$ goes to infinity $S$ goes to infinity. Conversely, as $x$ goes to negative infinity $S$ goes to zero. Moreover, $S=\bar{S}(t)$ at $x=0$. Using the previous information, we deduce  that the Black-Scholes PDE is defined in the intervals $x\in(-\infty, 0)$ for call options and $x\in(0, \infty)$ for put options. Therefore, we have
\begin{subequations}
  \begin{align}
      \text{\textbf{Call:}} \quad \dfrac{\partial v}{\partial \tau} - \dfrac{1}{2}\sigma^2\dfrac{\partial^2 v}{\partial x^2} - \bigg((r-\delta) + \dfrac{\sigma^2}{2} - \dfrac{\bar{S}'(\tau)}{\bar{S}(\tau)} \bigg)\dfrac{\partial v}{\partial x} + rv = 0 \quad \text{for $x < 0$ and $\tau \in (0, T]$} \\
      \text{\textbf{Put:}} \quad \dfrac{\partial v}{\partial \tau} - \dfrac{1}{2}\sigma^2\dfrac{\partial^2 v}{\partial x^2} - \bigg((r-\delta) - \dfrac{\sigma^2}{2} - \dfrac{\bar{S}'(\tau)}{\bar{S}(\tau)} \bigg)\dfrac{\partial v}{\partial x} + rv = 0 \quad \text{for $x > 0$ and $\tau \in (0, T]$}
  \end{align}
\end{subequations}
Note that the term on the new PDE that correspond to the terms in the linear parabolic operator $\mathcal{L}V$ defined in \eqref{eq:blackscholes:preliminaries:linear_parabolic_operator} are negative because the Black-Scholes PDE was inverted in time.

Again, the boundary conditions for the call option in the original domain are
$V(0,t)$ at $S=0$ and $V(\bar{S}, t) = \bar{S} - K$  at $S=\bar{S}(t)$, and when transforming those boundary conditions to the front fixing domain, they become $v(x, \tau) = 0$ for a sufficiently negative $x$ and $v(0, \tau) := \bar{S}_f(\tau) - 1 = V(\bar{S}, t) / K$ at $x=0$. Similarly, the boundary conditions for the put option in the original domain are $V(\bar{S}(t), t) = K - \bar{S}$ at $S=\bar{S}(t)$ and $V(S, t) = 0$ for a sufficiently large $S$, and under the front fixing domain, they become $v(0, \tau) = 1 - \bar{S}_f(\tau) = V(\bar{S}, t) / K$ at $x=0$ and $v(x, \tau) = 0$ for a sufficiently large $x$. Similarly, to as we did for the Nielsen transformation, we also rewrite the contact point condition
\begin{subequations}
  \begin{align}
    \text{\textbf{Call:}} \qquad & \dfrac{\partial v}{\partial x}(0, \tau) =  \bar{S}_f(\tau)\\
    \text{\textbf{Put:}} \qquad & \dfrac{\partial v}{\partial x}(0, \tau) = -\bar{S}_f(\tau)
  \end{align}
\end{subequations}
\begin{figure}[H]
  \centering
  \begin{subfigure}{0.4\textwidth}
    \centering
    \includegraphics[width=\textwidth]{chapters/chapter2/CompanyCallOption.pdf}
    \caption{Call option}
    \label{fig:blackscholes:frontfixingmethod:company_call_value_vs_curve}
  \end{subfigure}
  \begin{subfigure}{0.5\textwidth}
    \centering
    \includegraphics[width=\textwidth]{chapters/chapter2/CompanyPutOption.pdf}
    \caption{Put option}
    \label{fig:blackscholes:frontfixingmethod:company_put_value_vs_curve}
  \end{subfigure}
  \caption{Value $v(x,t) := V(S,t) / K$ in the front fixing domain defined by Company transformation.}
  \label{fig:blackscholes:frontfixingmethod:company_value_vs_curve}
\end{figure}
Since the transformed PDE is forward in time, we have to come up with initial conditions for $\bar{S}_f(\tau)$ and $v(x, \tau)$. For $\bar{S}_f(\tau)$, the initial condition is given by
\begin{equation}
  \bar{S}_f(0) = \dfrac{\bar{S}(T)}{K} = 1
\end{equation}
Moreover, for call options, the initial condition is given by $v(x, 0) = V(S, T) / K = \max\big(\bar{S}_f(0)e^{x} - 1, 0\big) = \max\big(e^{x} - 1, 0\big) = 0$ since $x$ is always negative. Similarly, for put options, the initial condition is given by $v(x, 0) = V(S, T) / K = \max\big(1 - \bar{S}_f(0)e^x, 0\big) = \max\big(1 - e^x, 0\big) = 0$ since $x$ is always positive. Hence,
\begin{subequations} \label{eq:blackscholes:frontfixingmethod:logtransform:american_options_terminal_condition}
  \begin{align}
    \text{\textbf{Call:}} \qquad & v(x, 0) = 0 \\
    \text{\textbf{Put:}} \qquad & v(x, 0) = 0
  \end{align}
\end{subequations}
Finally, grouping the equations together, we have the system
\begin{subequations} \label{eq:blackscholes:frontfixingmethod:logtransform:american_options_bs_pde}
  \begin{align}
    \text{\textbf{Call:}} \quad &
    \begin{cases}
      \dfrac{\partial v}{\partial \tau} - \dfrac{1}{2}\sigma^2\dfrac{\partial^2 v}{\partial x^2} - \bigg((r-\delta) + \dfrac{\sigma^2}{2} - \dfrac{\bar{S}'(\tau)}{\bar{S}(\tau)} \bigg)\dfrac{\partial v}{\partial x} + rv = 0 & \text{for $x < 0$ and $\tau \in (0, T]$}  \\ 
      v(x, 0) = 0 & \text{for $x < 0$} \\ 
      \bar{S}_f(0) = 1  \\ 
      \lim_{x\rightarrow-\infty}{v(x, \tau)} = 0 & \text{for $\tau \in (0, T]$}  \\ 
      \dfrac{\partial{v}}{\partial{x}}(0, \tau) = \bar{S}_f(\tau) & \text{for $\tau \in (0, T]$} 
    \end{cases} \\
    \text{\textbf{Put:}} \quad &
    \begin{cases}
      \dfrac{\partial v}{\partial \tau} - \dfrac{1}{2}\sigma^2\dfrac{\partial^2 v}{\partial x^2} - \bigg((r-\delta) - \dfrac{\sigma^2}{2} - \dfrac{\bar{S}'(\tau)}{\bar{S}(\tau)} \bigg)\dfrac{\partial v}{\partial x} + rv = 0 & \text{for $x > 0$ and $\tau \in (0, T]$} \\ 
      v(x, 0) = 0 & \text{for $x > 0$}\\
      \bar{S}_f(0) = 1 \\ 
      \lim_{x\rightarrow\infty} v(x, \tau) = 0 & \text{for $\tau \in (0, T]$} \\ 
      \dfrac{\partial{v}}{\partial{x}}(0, \tau) = -\bar{S}_f(\tau) & \text{for $\tau \in (0, T]$}
    \end{cases}
  \end{align}
\end{subequations}
\include{chapters/chapter3_finite_diferences_schemes}
\section{Linear complementary problem}


Now, note that the bound of $V(S, t)$ in each region is
\begin{align*}
  &V(S, t) - H(S, t) > 0 \qquad \text{for all $(S,t) \in \mathcal{C}$} \\ 
  &V(S, t) - H(S, t) = 0 \qquad \text{for all $(S,t) \in \mathcal{S}$}
\end{align*}

Similarly, the bound of $\mathcal{L}_{\text{BS}}(V)$ is
\begin{align*}
  &\frac{\partial{V}}{\partial{t}} + \frac{1}{2}\sigma^{2} S^2 \frac{\partial^2{V}}{\partial{S^2}} + (r - \delta)S \frac{\partial{V}}{\partial{S}} - rV = 0 \qquad \text{for $(S,t) \in \mathcal{C}$} \\
  &\frac{\partial{V}}{\partial{t}} + \frac{1}{2}\sigma^{2} S^2 \frac{\partial^2{V}}{\partial{S^2}} + (r - \delta)S \frac{\partial{V}}{\partial{S}} - rV < 0 \qquad \text{for $(S,t) \in \mathcal{S}$}
\end{align*}

Therefore, grouping the bounds above we form a linear complementary system of equations
{
  \color{red}  
  \begin{align}
    \begin{cases}
      \big[\frac{\partial V}{\partial t} - \mathcal{L}_{\text{BS}}(V)\big] \cdot [V(S,t) - H(S,t)] = 0 & \text{for all $(S,t)$} \\
      V(S, t) - H(S, t) \ge 0 & \text{for all $(S, t)$}\\
      \frac{\partial V}{\partial t} - \mathcal{L}_{\text{BS}}(V) \le 0 &  \text{for all $(S, t)$}\\
      V(S, T) = H(S, T) \\  
    \end{cases}
    \label{eq:background:finance:linear_complementary_problem}
  \end{align}
}

The benefit of the reformulation (\ref*{eq:background:finance:linear_complementary_problem})
is that there is no dependence on the unknown boundary of the continuation region.
Later in section (XXX) and section (XXX), we will explore numerical methods for solving 
both type of problems.



The Black-Scholes PDE can be transformed to heat diffusion PDE using the following
change of variables

\begin{align*}
  S &= Ke^x \\
  t &= T - \dfrac{2\tau}{\sigma^2} \\ 
  q &:= \dfrac{2r}{\sigma^2} \\
  q_{\delta} &:= \dfrac{2(r-\delta)}{\sigma^2} \\
  \alpha &:= \dfrac{1}{2}(q_{\delta} - 1) \\
  \beta &:= \dfrac{1}{4}(q_{\delta} - 1)^2 + q \\
  v(x, \tau) &:= e^{-(\alpha x + \beta \tau)}y(x, \tau)= V(S, t)
\end{align*}

The system (\ref{eq:background:finance:free_boundary_problem}) 
is the free boundary formulation for the pricing problem for American options.
A detailed derivation of (\ref{eq:background:finance:american_options_pde}) 
can be found at [REFERENCES].


The equation (\ref*{eq:background:finance:american_options_pde}) 
is a paraboblic PDE. Moreover, by applying the transformation,


the equation (\ref*{eq:background:finance:american_options_pde}) converts 
to the heat diffusion PDE.

\begin{equation}
  h(x, \tau) := \dfrac{H(S, t)}{K} = \begin{cases}
    \max(e^{x} - 1, 0)\\
    \max(1 - e^{x}, 0)
  \end{cases} 
\end{equation}

\begin{equation}
  \bar{x}(\tau) := \log{\bar{S}(t)} - \log{K} 
\end{equation}

\begin{align}
  \begin{cases}
    \dfrac{\partial y}{\partial \tau} = \dfrac{\partial^2 y}{\partial x^2} & \text{for $\tau\in[0,\dfrac{\sigma^2}{2}T)$ and $x\in(\bar{x}(t), \infty)$} \\
    y(x, \tau) = e^{(\alpha x + \beta \tau)}h(x, \tau) & \text{for $\tau\in[0, \dfrac{\sigma^2}{2}T]$ and $x\in(-\infty, \bar{x}(\tau)]$} \\
    \bar{x}(0) = 0
  \end{cases}
  \label{eq:background:finance:american_option_heat_equation}
\end{align}

We can reformulate equation (\ref*{eq:background:finance:american_option_heat_equation})
as:

\begin{equation}
  g := e^{\alpha x + \beta \tau}h(x, \tau)
\end{equation}

\begin{align}
  \begin{cases}
    \big(\dfrac{\partial y}{\partial \tau} - \dfrac{\partial^2 y}{\partial x^2}\big)(y  - g) =0 \\
    \dfrac{\partial y}{\partial \tau} - \dfrac{\partial^2 y}{\partial x^2} \ge 0 \quad y - g \ge 0 \\
    y(x, 0) = g(x, 0)
  \end{cases}
\end{align}


By exploring the geometric properties of the value function $V(S,t)$, 
we can determine useful conditions that will later help on in solving the equation 
(\ref*{eq:background:finance:american_options_pde}). Firstly, at
any given time $0 \le t \le T$, American options match the linear segment of the payoff
function within the stopping region. Therefore, we could say that 

\begin{align}
  \dfrac{\partial V}{\partial S}(S, t) =  \begin{cases}
    -1 & \text{(put)} \\ 
    1 & \text{(call)}
  \end{cases}
  \label{eq:background:finance:american_option_left_boundary}
\end{align}

Moreover, as the price goes to infinity the value of the option tends to zero

\begin{align}
  \lim_{S \rightarrow \infty}V(S, t) = 0 
  \label{eq:background:finance:american_option_stopping_right_boundary}
\end{align}



Pricing American options requires using numerical methods. The Black-Scholes PDE 
in (XXX) can be converted to the heat diffusion equation



To obtain such approximation, we rely on central difference approximations (REFERENCE).



\subsection{Explicit scheme}

An explicit scheme is one where we approximate the time partial derivative using
a forward difference approximation. Hence, the PDE in (XXX) is approximated as

\begin{equation}
  \dfrac{y^{n+1}_{i} - y^{n}_{i}}{\Delta \tau} = \dfrac{y^{n}_{i-1} - 2y^{n}_{i} + y^{n}_{i+1}}{(\Delta x)^2}
\end{equation}

By rearranging the terms,

\begin{equation}
  \lambda := \dfrac{\Delta \tau}{(\Delta x)^2}
\end{equation}

\begin{equation}
  y^{n+1}_i = \lambda y^{n}_{i-1} + (1 - 2\lambda)y^{n}_{i} + \lambda y^{n}_{i+1}
\end{equation}

It is shown by reference [REFERENCE] that method (XXX) is stable and consistent 
under the following condition

\begin{equation}
  0 < \Delta \tau \le \dfrac{(\Delta x)^2}{2}
\end{equation}

Moreover, the method has order of convergence $O(\Delta \tau, (\Delta x)^{2})$.

\subsection{Implicit scheme}

The implicit scheme approximates the time derivative using a backward difference

\begin{equation}
  \dfrac{y^{n+1}_{i} - y^{n}_{i}}{\Delta \tau} = \dfrac{y^{n+1}_{i-1} - 2y^{n+1}_{i} + y^{n+1}_{i+1}}{(\Delta x)^2}
\end{equation}

\begin{equation}
  y^{n+1}_{i} - \lambda (y^{n+1}_{i-1} - 2y^{n+1}_{i} + y^{n+1}_{i+1}) = y^{n}_{i}  
\end{equation}

\begin{equation}
  K := \begin{bmatrix}
    2 & -1     & & 0 \\ 
   -1 & \ddots & \ddots \\
      & \ddots & \ddots & \ddots \\
    0 & & \ddots & \ddots & \\
  \end{bmatrix} 
\end{equation}

\begin{equation}
  (I + \lambda K)\boldsymbol{y}^{n+1} = \boldsymbol{y}^{n}
\end{equation}

\subsection{Theta method}

\begin{equation}
  \dfrac{y^{n+1}_{i} - y^{n}_{i}}{\Delta \tau} = (1-\theta)\dfrac{y^{n}_{i-1} - 2y^{n}_{i} + y^{n}_{i+1}}{(\Delta x)^2} +  \theta\dfrac{y^{n+1}_{i-1} - 2y^{n+1}_{i} + y^{n+1}_{i+1}}{(\Delta x)^2}
\end{equation}

\begin{equation}
  y^{n+1}_{i} - \lambda\theta(y^{n+1}_{i-1} - 2y^{n+1}_{i} + y^{n+1}_{i+1}) =  y^{n}_{i} + (1-\theta)\lambda(y^{n}_{i-1} - 2y^{n}_{i} + y^{n}_{i+1})
\end{equation}

\begin{equation}
  (1 + \lambda\theta K)\boldsymbol{y}^{n+1} = (1-\lambda\theta K)\boldsymbol{y}^{n} 
\end{equation}

\subsubsection{Theta method}

We discretize the system of equation containing (3.17) and (3.18) to solve it. 
Firstly, we define an uniform meshgrid within the region $[1, x_\text{max}]\times [0, T]$ and 
with resolution $\Delta x$ and $\Delta t$.

\begin{align*}
    M := \dfrac{x_{\text{max}} - 1}{\Delta x}
    \qquad
    N := \dfrac{T}{\Delta t}
\end{align*}

\begin{align*}
   & & x_i &:= 1 + i{\Delta x} &\text{for $i = 0,\dots,M$} & & \\ 
   & & t_n &:= n{\Delta t} &\text{for $n = 0,\dots,N$} & &
\end{align*}

Now we define the approximations

\begin{equation}
    v^{n}_{i} \approx v(x_{i}, t_{n}) \qquad \text{for $(x_{i}, t_{n}) \in \{x_k\}^{M}_{0} \times \{t_k\}^{N}_{0}$} 
\end{equation}
\begin{equation}
    \bar{S}^{n} \approx \bar{S}(t_{n}) \qquad \text{for $t_{n} \in \{t_k\}^{N}_{0}$} 
\end{equation}

By the boundary conditions, we can derive an expression for

\begin{align}
    v^{n}_{0} &= K - \bar{S}^{n} \qquad \text{for $n = 0, \dots, N - 1, N$} \\
    v^{n}_{M+1} &= 0 \qquad \text{for $n = 0, \dots, N - 1, N$}
\end{align}

Additionally by using the smothness condition, we get:

\begin{align}
    \dfrac{v^{n}_{1} - v^{n}_0}{\Delta x} = -\bar{S}^{n}
\end{align}

or 

\begin{align}
    v^{n}_{1}= K - (1 + {\Delta x})\bar{S}^{n}
\end{align}

Next, we equation (3.XX) discretize using centered finite difference. The discretization
method, we use is the theta method which a interpolation between an implicit and explicit scheme.

\begin{equation}
    \begin{split}
        v^{t+1}_{i} &- v^{t}_{i} + \theta \bigg\{ \dfrac{1}{2}\sigma^2 x^{2}_{i} \dfrac{\Delta t}{(\Delta x)^2} (v^{t}_{i-1} - 2 v^{t}_{i} + v^{t}_{i+1}) + \bigg[ (r - \delta) - \dfrac{1}{\bar{S}^t} \dfrac{\bar{S}^{t+1} - \bar{S}^{t}}{\Delta t} \bigg] \dfrac{\Delta t}{2\Delta x} (v_{i+1}^{t} - v_{i-1}^{t}) - r v^{t}_{i} \Delta t \bigg\}
        \\ & + (1-\theta) \bigg\{ \dfrac{1}{2}\sigma^2 x^{2}_{i} \dfrac{\Delta t}{(\Delta x)^2}(v^{t+1}_{i-1} - 2 v^{t+1}_{i} + v^{t+1}_{i+1}) 
        \\ &  + \bigg[ (r - \delta) - \dfrac{1}{\bar{S}^{t+1}} \dfrac{\bar{S}^{t+1} - \bar{S}^{t}}{\Delta t} \bigg]\dfrac{\Delta t}{2\Delta x}(v_{i+1}^{t+1} - v_{i-1}^{t+1}) - r v^{t+1}_{i} \Delta t \bigg\} = 0
    \end{split}
\end{equation}

To simplify the expression above, we introduce the following terms

\begin{equation}
    \lambda := \dfrac{\Delta t}{(\Delta x)^2}
\end{equation}

\begin{align}
    \alpha_i &:= 1 + \theta (\lambda \sigma^2 x_{i}^{2} + r{\Delta t}) \\
    \beta_i &:= - \dfrac{1}{2} \lambda \theta \bigg[ \sigma^{2} x_{i}^{2} - x_i \Delta x (r - \delta) \bigg]  - \dfrac{1}{2} \lambda \theta   x_i \Delta x \dfrac{\bar{S}^{n+1} - \bar{S}^{n}}{\Delta t \bar{S}^n}  \\
    \gamma_i &:= -\dfrac{1}{2} \lambda \theta \bigg[ \sigma^{2} x_{i}^{2} + x_i \Delta x (r - \delta) \bigg]  + \dfrac{1}{2} \lambda \theta  x_i \Delta x  \dfrac{\bar{S}^{n+1} - \bar{S}^{n}}{\Delta t \bar{S}^n}\\
    a_i &:= 1 - (1-\theta) (\lambda \sigma^2 x_{i}^{2} +  r{\Delta t}) \\
    b_i &:= \dfrac{1}{2} (1-\theta) \lambda \bigg[\sigma^{2} x_{i}^{2} - x_i \Delta x \bigg( (r - \delta) - \dfrac{1}{\Delta t} \bigg) \bigg] \\
    c_i &:= \dfrac{1}{2} (1-\theta) \lambda \bigg[ \sigma^{2} x_{i}^{2} +  x_i \Delta x \bigg( (r - \delta) - \dfrac{1}{\Delta t} \bigg) \bigg] \\
    d^{n+1}_i &:= (1-\theta) \dfrac{x_i}{2 \Delta x}  \dfrac{v^{n+1}_{i+1} - v^{n+1}_{i-1}}{\bar{S}^{n+1}}
\end{align}

Now, the expression above becomes

\begin{equation}
    \beta^{n}_{i} v^{n}_{i-1} + \alpha^{n}_{i} v^{n}_{i} + \gamma^{n}_{i} v^{n}_{i+1} = b_i v^{n+1}_{i-1} + a_i v^{n+1}_{i} + c_i v^{n+1}_{i+1} + d^{n+1}_{i}\bar{S}^{n}
\end{equation}

\begin{align}
    \gamma^{n}_{1} v^{n}_{2} = b_1 v^{n+1}_{0} + a_1 v^{n+1}_{1} + c_1 v^{n+1}_{2} + d^{n+1}_{1}\bar{S}^{n} - \beta^{n}_{1} (K - \bar{S}^n) - \alpha^{n}_{1} (K - (1+\Delta x)\bar{S}^n) 
\end{align}

\begin{equation}
    \alpha^{n}_{2} v^{n}_{2} + \gamma^{n}_{2} v^{n}_{3} = b_2 v^{n+1}_{1} + a_2 v^{n+1}_{2} + c_2 v^{n+1}_{3} + d^{n+1}_{2}\bar{S}^{n} - \beta^{n}_{2} (K - (1+\Delta x)\bar{S}^n) 
\end{equation}

\begin{equation}
    \beta^{n}_{M} v^{n}_{M-1} + \alpha^{n}_{M} v^{n}_{M} = b_i v^{n+1}_{i-1} + a_i v^{n+1}_{i} + c_i v^{n+1}_{i+1} + d^{n+1}_{i}\bar{S}^{n}
\end{equation}

\section{Conclusions}

An option provides the right to buy or sell an asset at a predetermined strike price in the future. Generally, investment firms are responsible for writing these contracts and selling them to investors. Subsequently, investors hold these contracts to hedge against potential changes in price. When a holder already owns an asset and wants to hedge against a potential price drop, they enter into a put option. On the contrary, if a holder aims to hedge against an increase in the price of an asset they intend to acquire, they enter into a call option contract. A wide variety of options are available in the market. Among the most notorious contracts, we have European and American options. European options are contracts that can exercise at the expiration date only. Likewise, American options are contracts that can be exercised before or at the maturity date. 

Investment firms charge premiums to investors for entering an option contract. Then, the writer uses the premium to hedge the possible claims that the holder will have in the future. Charging the correct premium is important because it minimizes the arbitrage opportunities for either the holder and writer. Clearly, pricing schemes depend on the type of the contract. In general, The Black-Scholes formula is used to price European options. Sadly, no formula is available for American options. Therefore, firms rely on numerical methods to come up with some approximation of the price. Numerous numerical methods for pricing American options derive from the Black-Scholes PDE. 

In this report we have implemented, and analyzed numerical schemes derived from the free boundary and the variational inequalities' formulation of the pricing problem. Specifically, we have discussed: An explicit and implicit front fixing schemes for solving the free boundary problem based on the Nielsen transformation, an explicit front fixing scheme based on the Company transformation, and the explicit, implicit and Crank-Nicholson PSOR schemes.

The explicit front fixing schemes were derived from applying central finite difference and forward/backward difference. However, they differ in how the contact point condition is approximated. Specifically, The Nielsen front fixing schemes use forward difference (or backward difference for call options) to approximate the contact point condition while the Company explicit scheme use central finite difference. This explains why Nielsen front fixing schemes yielded first order convergence with respect to the spatial discretization parameter $\Delta{x}$ while Company explicit scheme yielded second order. Moreover, as it is normally the case for explicit and implicit central finite differences, both Nielsen and Company schemes exhibits first order convergence with respect to the temporal discretization parameter $\Delta{t}$. While the explicit front fixing schemes for Nielsen and Company transformation are conditionally stable, they both proved to be substantially faster and much more accurate than the implicit scheme. 

We discourage the use of the Nielsen implicit scheme. As we already told, the implicit scheme is less accurate by far. The reason behind this is that the implicit scheme requires to solve a nonlinear system of equation at each time step in the grid. Moreover, the approximation errors produced by the nonlinear solver get accumulated over time, affecting the overall accuracy of the method. We could decrease the approximation error in the nonlinear solver by decreasing its tolerance, and increasing its maximum number of iteration. However, we found that as we do that, the overall performance of the method reduces substantially. To make things even worse, the size of the nonlinear equations is inversely proportional to the spatial discretization parameter. Therefore, the computational resources required by the implicit method grows substantially as we decrease the spatial discretization parameters. For instance, for a grid of $M$ nodes in the spatial direction, the non-linear solver needs to invert a Jacobian matrix of $M\times M$ entries. In other words, decreasing the spatial discretization parameters by a decimal point, requires 100 times more memory. To summarize, we can increase the accuracy of implicit method by decreasing the tolerance of the non-linear solver and by decreasing the discretization parameters of the grid but by sacrificing the performance of the method and increasing the memory consumption substantially.

Similar to the Company front fixing schemes, the PSOR schemes showed to have second order convergence with respect to the spatial discretization parameter $\Delta{x}$. Moreover, it showed to have first order convergence for the explicit ($\theta=0$) and implicit ($\theta=1$) schemes, and second order convergence for the Crank-Nicholson ($\theta=0.5$) in with respect to the temporal discretization parameter $\Delta{t}$. Analogous to the explicit front fixing schemes, the PSOR explicit scheme is conditionally stable. When comparing the performance of the explicit PSOR to the explicit front fixing schemes, the explicit front fixing schemes showed to be substantially faster. In that regard, the issue with explicit PSOR is that as you decrease $\Delta{x}$, the complementary equations grow, hence, taking more time to solve the linear complementary problem. Similar argument can be done when comparing explicit front fixing schemes to the implicit and Crank-Nicholson PSOR schemes. In spite of that, the Crank-Nicholson PSOR scheme is second order convergence in time, therefore, by choosing $\Delta{x}$ to be smaller than $\Delta{t}$, we could have greater performance maintaining the accuracy. 

Concluding, the numerical experiments conducted showed that the explicit front-fixing schemes offers superior performance and accuracy than the implicit front fixing scheme and the all the PSOR schemes. Between the Company explicit front fixing scheme and the Nielsen explicit front fixing scheme, we recommend using the Company transformation because it has second order convergence in space, hence, yielding to smaller approximation errors. Similarly, the explicit PSOR scheme exhibited smaller approximation errors and computational than the implicit and Crank-Nicholson PSOR schemes and the implicit front fixing scheme. However, it is still rather slow compared to the explicit front fixing schemes.

\section{Further research}

Further research opportunities arise from this report. Firstly, we saw that generally, people transform the Black-Scholes PDE to the heat diffusion equation. Although this was done for the LCP problem, the front fixing schemes were derived within the financial domain which might led to worse approximations or slower performance. Moreover, we might modify Nielsen front fixing scheme so that it approximates the contact point condition using central finite differences. Likewise, for the Nielsen implicit front fixing scheme, we could explore using nonlinear methods for large scale-scale nonlinear systems such as the one proposed by\cite{lacruz_2006}. Also, we might also derive Crank-Nicholson schemes for the Nielsen and Company front fixing schemes. Finally, we might consider using real market data to evaluate how good are the numerical schemes presented under real market conditions.



\section{Another section} \label{sec:my1}

\subsection{A subsection} \label{subsec:theory}

Subsections may be used. Use a clear structure in your report.

We denote the set of real numbers by
$\mathbb{R}$, the set of integers by $\ZZ$ and the set of complex
numbers by $\mathbb{C}$. Our analysis is based on the equation
$e^{\pi i} = -1$ and the relation
\begin{equation}
  \frac{2}{4} = \frac{1}{2}   \label{eq:myeq1}
\end{equation} % no empty line after this
which we verify in the appendix \ref{app:calculations}.
Useful consequences are
\begin{align}
  \frac{4}{8} &= \frac{1}{2} \\
  \frac{4}{12} + \frac{1}{\Gamma(s)}\int_0^{\infty} \frac{t^{s-1}}{e^t-1} dt
     &= \frac{1}{3} +\sum_{n=1}^{\infty} \frac{1}{n^s}\\
  \frac{2}{10} &= \frac{1}{5} 
\end{align}
For any $0\neq a\in \ZZ$, the equality
\begin{equation*} % * for no numbering
 \frac{2 a}{4 a} = \frac{1}{2}
\end{equation*}
follows from equation \eqref{eq:myeq1}.

\subsection{Another subsection} \label{subsec:application}

\subsubsection{A subsubsection} \label{subsubsec:red}

Sometimes subsubsections may be appropriate.

\subsubsection{Another subsubsection} \label{subsubsec:green}



This could contain a table of interesting numbers
\begin{center}
  \begin{tabular}{r|cccccc}
    $n$   & 1 & 2 & 3 & 4 & 5 & 6 \\ \hline
    $F_n$ & 1 & 1 & 2 & 3 & 5 & 8 \\
    $B_n$ & $\tfrac{1}{2}$ & $\tfrac{1}{6}$ & 0 & $-\tfrac{1}{30}$ & 0 &  $\tfrac{1}{42}$ \\
    $p_n$ & 2 & 3& 5& 7 & 11 & 13 \\
  \end{tabular}
\end{center}

\section{Yet another section} \label{sec:my2}

Graphics can be included. Figure \ref{fig:bsd} shows an example.
Learn about floats and pictures in the \LaTeX\ wikibook to place
the figures at the right place.
%
\begin{figure}
 \begin{center}
   \includegraphics[width=0.7\textwidth]{bsd.jpg}
 \end{center}
 \caption{Oh look, something happens here !}
 \label{fig:bsd}
\end{figure}

\section{Conclusions} \label{sec:conclusions}

Further help on \LaTeX\ can be found easily on the internet. The \LaTeX\
wikibook\footnote{\tt http://en.wikibooks.org/wiki/LaTeX} contains a lot.
For instance you would find there how to type theorems and proofs nicely.
Or how to include source code written in some programming language like
python. There are long lists available with all sorts of common
mathematical symbols like $\xi$, $\nabla$, $\infty$, $\log$, $\iff$, etc.

\newpage

\appendix

\section{Raw data} \label{app:rawdata}

Material that needs to be included but would distract from the main
line of presentation can be put in appendices.
Examples of such material are raw
data, computing codes and details of calculations.

But note tha the maximal number of pages includes the appendix and the references.

\section{Calculations for section \ref{sec:my1}} \label{app:calculations}

In this appendix we could verify equation \eqref{eq:myeq1} or present the code that was used. 
\begin{lstlisting}[language=Python]
def gcd(a,b):
    """
    Return the greatest common divisor
    of a and b 
    """
    while b > 0:
        (a, b) = (b, a % b)
    return a
\end{lstlisting}

\newpage

\printbibliography

\end{document}
