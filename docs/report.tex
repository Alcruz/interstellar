\documentclass{uonmathreport}

% loads already mathtools, graphicx,
% but you may want to add other packages, like
\usepackage{algorithm}
\usepackage{algpseudocodex}
\usepackage{biblatex}
\usepackage{booktabs}
\usepackage{listings} % to include code in python see https://en.wikibooks.org/wiki/LaTeX/Source_Code_Listings
\usepackage{xcolor}
\usepackage{float}
\usepackage{aliascnt}
\usepackage{graphicx}
\usepackage{caption}
\usepackage{subcaption}
\usepackage{hyperref}
\usepackage{amssymb}
\usepackage{pdfpages}

\definecolor{codegreen}{rgb}{0,0.6,0}
\definecolor{codegray}{rgb}{0.5,0.5,0.5}
\definecolor{codepurple}{rgb}{0.58,0,0.82}
\definecolor{backcolour}{rgb}{0.95,0.95,0.92}

\lstdefinestyle{mystyle}{
    backgroundcolor=\color{backcolour},   
    commentstyle=\color{codegreen},
    keywordstyle=\color{magenta},
    numberstyle=\tiny\color{codegray},
    stringstyle=\color{codepurple},
    basicstyle=\ttfamily\footnotesize,
    breakatwhitespace=false,         
    breaklines=true,                 
    captionpos=t,                    
    keepspaces=true,                 
    numbers=left,                    
    numbersep=5pt,                  
    showspaces=false,                
    showstringspaces=false,
    showtabs=false,                  
    tabsize=2
}

\lstset{style=mystyle}

\addbibresource{progress.bib}
% other useful pagackes include booktabs, hyperref, amsthm, xcolor, todonotes, showkeys, ...
% see https://www.overleaf.com/learn 

% change the following to
% to \PJA = MATH4041, \PJS = MATH4042 or \DIS = MATH4001 (for BSc and MMAth)
% or \MSc (for all Msc dissertations)
\MSc

% adjust the following
\title{Numerical Methods for Pricing American Options with Continuous Dividend Yields Assets.}
\author{Alvin Jonel De la Cruz Guerrero}
\academicyear{2022/23}
\supervisor{Dr. Anna Kalogirou}

% the following are irrelevant for Msc:
% \assessmenttype{Dissertation} % or Investigation

% the following are irrelevant for PJS, PJA, DIS:
% Msc: change it to G14PMD and Pure Mathematics, etc ...
\msccode{MATH4062}
\msctitle{Financial and Computational Mathematics}

% put your own definitions and shorthands here
\newcommand{\ZZ}{\mathbb{Z}}
\newcommand{\expnumber}[2]{{#1}\mathrm{e}{#2}}

\newaliascnt{eqfloat}{equation}
\newfloat{eqfloat}{h}{eqflts}
\floatname{eqfloat}{Equation}

\newcommand*{\ORGeqfloat}{}
\let\ORGeqfloat\eqfloat
\def\eqfloat{%
  \let\ORIGINALcaption\caption
  \def\caption{%
    \addtocounter{equation}{-1}%
    \ORIGINALcaption
  }%
  \ORGeqfloat
}

\parindent 0cm

\begin{document}

\maketitle

\begin{abstract}

  In this work, we present numerical methods for pricing American options with continuous dividend-yield assets. We focus on two important formulations of the pricing problem: the free boundary problem and the variational inequalities problem. In the free boundary formulation, we solve the Black-Scholes PDE with a moving boundary condition. The front-fixing method proposes using a change of variables that transforms the free boundary problem into a PDE where the moving boundary is fixed and is explicitly present. We consider the changes of variables proposed by Nielsen et al. and Company et al. Then we implement explicit and implicit finite difference schemes to solve the front-fixed PDE. On the other hand, for the variational complementary problem, we propose solving a system of variational inequalities derived from transforming the Black-Scholes PDE into the heat diffusion equation. By applying the theta method, the variational inequalities problem converts to a linear complementary problem in which a linear equation needs to be solved subject to complementary conditions. Then, we implement the PSOR method to solve the LCP problem iteratively. Finally, by conducting numerical experiments, we conclude that both explicit methods for Nielsen and Company transformations are efficient and accurate. Moreover, we discourage the use of the implicit method for the Nielsen transformation because it leads to a higher approximation error. Lastly, although the LCP problem yields accurate approximations, the method is relatively inefficient compared to the explicit methods for the Nielsen transformation.
\end{abstract}

% Table of contents
\setcounter{tocdepth}{3}  % this will list subsections, but not subsubsections
\tableofcontents 
\newpage
\numberwithin{figure}{section}
\numberwithin{table}{section}
\numberwithin{algorithm}{section}
\numberwithin{lstlisting}{section}

\section{Introduction}

\subsection{Overview}

Options are equity-based derivatives that are primarily used to mitigate risk. The options market is significantly larger compared to other derivatives. In fact, options were the most traded derivatives in 2019, with a volume of 18.55 billion contracts when combining index and individual equities contracts \cite{statista_2019}. An enormous market like the options market demands reliable pricing mechanisms to minimize arbitrage opportunities. Naturally, pricing American options is a broad research area because there is no closed-form solution to the PDE resulting from the Black-Scholes model. Merton \cite{merton_1973} was the first to consider the Black-Scholes model proposed by Black and Scholes \cite{black_scholes_1973} to price European and American options. Although Merton derived a nice formula to price European options, he stated that, in general, a closed-form solution was not attainable for American options. In 1977, Schwartz \cite{schwartz_197779} and Brennan \cite{brennan_1997} proposed using finite difference schemes to solve the pricing problem for American options. The work of Merton and Schwartz served as a foundation for the free boundary problem for American options pricing. The motivation behind the free boundary problem formulation is that for American options, there exists an optimal exercise price that marks the boundary between the region where exercising the option is profitable and the region where it is not. Moreover, this optimal exercise price changes with time, making it impossible for the holder to determine when to exercise. Based on the work of Landau \cite{landau_1950_heat_ci}, Wu et al. \cite{wu1997front} formulated the front-fixing method as an approach to solve the free boundary problem for options, in which the Landau transformation is used to transform the optimal exercise price or the moving boundary into a fixed boundary. Multiple transformations have been proposed by Huang et al. \cite{huang_2000}, Nielsen et al. \cite{nielsen_2001}, and Company et al. \cite{company_egorova_jodar_2014}. Around the same period, Dewynne et al. \cite{dewynne_howison_rupf_wilmott_1993} took a different approach to solve the pricing problem. The idea was to reformulate the free boundary problem as a problem with variational inequalities that, when finite difference is applied, transforms into a linear complementary problem \cite{dantzig_1968}.

\subsection{Aim} 

The goal of this work is to implement the numerical methods proposed by Nielsen et al. \cite{nielsen_2001} and Company et al. \cite{company_egorova_jodar_2014} to solve the free boundary formulation of the pricing problem for American options \cite{dewynne_howison_rupf_wilmott_1993}. Moreover, Company et al. \cite{company_egorova_jodar_2014} and Nielsen et al. \cite{nielsen_2001} proposed schemes for pricing American put options with underlying assets that have non-paying dividends. However, we aim to derive analogous schemes for pricing call contracts as well, considering underlying assets that have a continuous dividend yield, such as the S\&P500. Additionally, we want to conduct a convergence analysis of each of the methods. Furthermore, we consider the linear complementary formulation proposed by Dewynne \cite{dewynne_howison_rupf_wilmott_1993} \cite{wilmott_howison_dewynne_1995}. Finally, we perform a convergence analysis of each of the methods implemented.

\subsection{Main Achievements}

In this work, we were able to derive the corresponding PDE problem resulting from applying the transformations proposed by Nielsen et al. \cite{nielsen_2001} and Company et al. \cite{company_egorova_jodar_2014} for call and put options with underlying assets featuring a continuous dividend yield. Moreover, we implemented explicit and implicit schemes for Nielsen et al.'s method \cite{nielsen_2001}, an explicit scheme for Company et al.'s method \cite{company_egorova_jodar_2014}, and the theta method for the linear complementary problem proposed by \cite{wilmott_howison_dewynne_1995} using Python. Finally, we derived the order of convergence for each of the implemented methods.

\subsection{Outline}

The outline of this paper is as follows. In Section 2, we explore the Black-Scholes model for American options for assets that pay dividends, resulting in the free boundary formulation of the pricing problem. Furthermore, we delve into the front-fixing method as a strategy for fixing the moving boundary by applying a change of variable. We consider the changes of variables proposed by Company et al. \cite{company_egorova_jodar_2014} and Nielsen et al. \cite{nielsen_2001}. In Section 3, we explore explicit and implicit schemes to solve the partial differential equations resulting from applying the front-fixing method and the Nielsen transformation to the free boundary problem obtained in Section 1. We conclude Section 3 with numerical experiments and convergence analysis for the numerical schemes presented based on the work of Nielsen and Company (see Appendix \ref{sec:company_explicit_scheme}). In Section 5, we explore a reformulation of the pricing problem as a variational inequality and the linear complementary system of equations resulting from it. Additionally, we delve into the theta method, which is a more general numerical method that yields explicit, implicit, and Crank-Nicholson schemes. Finally, in the same section, we discuss the results of solving the linear complementary problem reformulation of pricing American options using the theta method.

\section{Black-Scholes model} \label{sec:blackscholes}

\subsection{Preliminaries}

A common problem in finance is pricing financial derivatives, often referred to simply as derivatives. In essence, derivatives are contracts set between parties whose value over time derives from the price of their underlying assets. A notorious family of derivatives in financial markets are "options". Options are contracts set between two parties in which the holder has the right to sell or buy, commonly referred to as exercising, an underlying asset at a pre-established price, also known as the "strike price", in the future. Options are referred to as "call options" or "put options" if the exercise position is to buy or to sell, respectively. Similarly, options are classified depending on their exercise style. In that regard, the simplest options are European options. European options give the right to exercise at the expiration date of the contract. Another well-known type of option is the American option. American options work similarly to European options, with the difference that they can be exercised at any point in time between the beginning and the expiration date of the contract. Obviously, American and European options are almost similar, and they only differ in at the times which the holder can exercise them. Therefore, we will start describing the pricing problem from the European options perspective and extended to the case of American options.

Let us define the payoff function of European option as
\begin{subequations}
  \begin{align}
    \text{\textbf{Call:}} \qquad H_{\text{Eur}}(S) = \max(S - K, 0) \\
    \text{\textbf{Put:}} \qquad H_{\text{Eur}}(S) = \max(K - S, 0)
  \end{align}
  \label{eq:blackscholes:preliminaries:european_put_payoff}
\end{subequations}
where $K$ is the strike price and remains constant through lifespan of the option,
$S \in [0, \infty]$ is the asset price at the maturity date. We can extend the European payoff to American options by introducing
the time axis to equation above.
\begin{subequations}
  \begin{align}
    \text{\textbf{Call:}} \qquad H(S, t) = \max(S - K, 0) \\
    \text{\textbf{Put:}} \qquad H(S,t) = \max(K - S, 0)
  \end{align}
  \label{eq:blackscholes:preliminaries:american_put_payoff}
\end{subequations}
The interval $[0, T]$ represent the lifespan of option where $T$ is the elapsed time, in years, since the starting and expiration date of the contract. For example, $T=1$ indicates that the contract expires one year after starting date. Moreover, $t\in[0, T]$ is the elapsed time, in years, since the starting date of the contract. While the payoff of European options is defined only at $t=T$, American option's payoff is defined for all $(S, t) \in [0, \infty]\times[0, T]$.

Options provide greater flexibility to holders by eliminating their exposure to negative payoffs. Therefore, writers charge premiums to the holders to acquire the contract. The premium is often referred to as the price or value of the option, and the problem of determining this value is called option pricing. Let $V_t$ represent the value of the option at time $t$. For instance, $V_0$ represents the value of the option at the beginning of the contract. When pricing options, it is crucial to find the fair price; otherwise, the writer or holder of the option could devise a scheme in which the option will always be profitable for them. In other words, options pricing must adhere to the principle of no-arbitrage. Therefore, we assume that the writer of the option uses the premium to construct a portfolio consisting of $\phi_0$ units of the asset and invests $\psi_0$ units of cash in a risk-free asset $B_t$, such as a US Treasury bills, certificates of deposit, or a bank account. Then, the writer rebalances the portfolio $(\phi_0, \psi_0)$ to hedge against any potential claims from the holder at any future time $0 < t \le T$. Consequently, at any time $t$, the writer holds a portfolio $(\phi_t, \psi_t)$ with a value
\begin{equation*}
  \Pi_t = \phi_t S_t + \psi_t B_t
\end{equation*}
Moreover, the portfolio is self-financing. In other words, the changes in
portfolio depend on the changes in $S_t$ and $B_t$,
and the rebalancing of portfolio $(\phi_t, \psi_t)$
\begin{align*}
   & d\Pi_t = \phi_tdS_t + \psi_t dB_t \\
   & S_t d\phi_t + B_t d\psi_t = 0
\end{align*}
Finally, the portfolio value matches the option value
\begin{equation*}
  \Pi_t = V_t
\end{equation*}
at any time $0 \le t \le T$. Using the self-financing portfolio hedging strategy, The Black-Scholes model presents a mathematical model for the dynamics of an option's price. The model makes certain assumptions about the market. A complete list of all the assumptions can be found in \cite{merton_1973} and \cite{wilmott_howison_dewynne_1995}. In the next part, we enumerate some them. Firstly, the asset price $S_t$ is distributed as a log-normally
\begin{equation}
  \label{eq:blackscholes:preliminaries:asset_price}
  S_t = S_0 \exp\bigg\{\int_{0}^{t} \big(r(s) - \dfrac{1}{2}\sigma(s)\big)ds + \sqrt{t}Z\bigg\}
\end{equation}
where the risk-free interest rate $r(t)$ and the price volatility $\sigma(t)$ are deterministic functions of time during the life of the option. Secondly, the bank account $B(t)$ is a deterministic function
\begin{equation*}
  dB = r(t)B(t)dt
\end{equation*}
Finally, the asset does not pay dividends. Additionally, we will assume that
the risk-free interest rate and asset price volatility are constant during the
life of the option. Later on, we will address the assumption about dividends.

By applying the Black-Scholes model to price European options, the famous
Black-Scholes PDE is obtained
\begin{equation*}
  \label{eq:blackscholes:preliminaries:european_option_pde}
  \begin{cases}
    \dfrac{\partial{V}}{\partial{t}} + \dfrac{1}{2}\sigma^{2} S^2 \dfrac{\partial^2{V}}{\partial{S^2}} + r S \dfrac{\partial{V}}{\partial{S}} - rV = 0 & \text{for $t\in[0,T)$ and $S\in[0, \infty)$} \\ V(S, T) =
    H(S, T) & \text{for $S\in[0, \infty)$}
  \end{cases}
\end{equation*}
where $V(S, t)$ is a deterministic function. A derivation of the Black-Scholes PDE for European options can be found in \cite{merton_1973} and \cite{wilmott_howison_dewynne_1995}. We previously mentioned that Black-Scholes model assumes that the underlying asset does not pay dividends. In most cases, assets such as stocks pay out dividends just a few times at year. In this case, dividends are to be modelled discretely. However, there are certain assets that pay out a proportion of the current price during and interval of time. For instance, indexes such as the SPX. Thus, in such cases, it is useful to model dividends as a continuous yield. \cite{wilmott_howison_dewynne_1995} shows a slight adjusted asset price model that includes continuous yield dividends
\begin{equation}
  \label{eq:blackscholes:preliminaries:dividends_asset_price}
  S_t = S_0 \exp\bigg\{\int_{0}^{t} \big(r(s) - \delta(s) - \dfrac{1}{2}\sigma(s)\big)ds + \sqrt{t}Z\bigg\}
\end{equation}
Note that when the asset does not pay out dividends $\delta(t) = 0$, the asset price model will be exactly as in \eqref{eq:blackscholes:preliminaries:asset_price}. Similarly to as we did for the risk-free interest rate and the volatility of the asset price, we will assume that continuous dividends yield is as constant from now on. As a consequence of
the asset price mode \eqref{eq:blackscholes:preliminaries:dividends_asset_price}, the Black-Scholes PDE changes to 
\begin{equation}
  \label{eq:chapter2:european_option_pde_with_dividens}
  \begin{cases}
    \dfrac{\partial{V}}{\partial{t}} + \mathcal{L}_{\text{BS}}(V) = 0 & \text{for $t\in[0,T)$ and $S\in[0, \infty)$} \\ 
    V(S, T) = H(S,T) & \text{for $S\in[0, \infty)$}
  \end{cases}
\end{equation}
where $\mathcal{L}_{\text{BS}}(f)(x)$ is the linear parabolic operator applied to the function $f \in \mathcal{C}^2$
\begin{equation}
  \label{eq:blackscholes:preliminaries:linear_parabolic_operator}
  \mathcal{L}_{\text{BS}}(f)(x) := \dfrac{1}{2}\sigma^{2} x^2 \dfrac{\partial^2{f}}{\partial{x^2}} + (r - \delta) x \dfrac{\partial{f}}{\partial{x}} - rf(x)
\end{equation}
Note that for conciseness, we write $\mathcal{L}_\text{BS}(V)(S)$ as $\mathcal{L}_\text{BS}(V)$. Similarly, to the asset price model with dividends, the Black-Scholes PDE with dividends fall back to the original Black-Scholes PDE if the asset does not pay out dividends $\delta = 0$. So far, we have presented an equation that describes the dynamics of the value $V(S,t)$ of European options. So now, we will consider the pricing of American options. By applying the Black-Scholes model to price American options, Merton \cite*{merton_1973} derives some important facts. Firstly, the value $V(S,t)$ is bounded from below by the payoff function:
\begin{align}
  \label{eq:blackscholes:american_options_price_lower_bound}
  V(S, t) \ge H(S, t) \qquad \text{for $t \in [0, T]$}
\end{align}
Moreover, the domain of $V(S, t)$ can be separated into the exercise region in 
\begin{equation}
  \mathcal{S} := \{(S, t) : V(S, t) = H(S, t)\}
  \label{eq:blackscholes:preliminaries:exercise_region}
\end{equation}
in which it is profitable for the holder to exercise the option, the continuation region
\begin{equation}
  \label{eq:blackscholes:preliminaries:continuation_region}
  \mathcal{C} := \{(S, t) : V(S, t) > H(S, t)\}
\end{equation} 
in which it is preferable to continue holding the option because exercising is not profitable, and the optimal exercise boundary that separates the continuation region and exercise region
\begin{equation}
  \label{eq:blackscholes:preliminaries:optimal_exercise_boundary}
  \partial \mathcal{C} := \{(S, t) : S = \bar{S}(t)\}
\end{equation}
where $\bar{S}(t)$ is the optimal exercise price. 
\begin{figure}[H]
  \centering
  \begin{subfigure}{0.4\textwidth}
    \centering
    \includegraphics[width=\textwidth]{chapters/chapter2/AmericanCallOptionValue}
    \caption{Call option}
    \label{fig:blackscholes:preliminaries:american_call_value_vs_curve}
  \end{subfigure}
  \hfill
  \begin{subfigure}{0.4\textwidth}
    \centering
    \includegraphics[width=\textwidth]{chapters/chapter2/AmericanPutOptionValue.pdf}
    \caption{Put option}
    \label{fig:blackscholes:preliminaries:american_put_value_vs_curve}
  \end{subfigure}
  \caption{Value $V(S, t)$ of American option value curve. }
  \label{fig:blackscholes:preliminaries:american_option_value_vs_curve}
\end{figure}
Lastly, the price dynamics of
American options is governed by the same Black-Scholes PDE as European options in the continuation region. Pricing American options only requires solving $V(S,t)$ at continuation region and finding the optimal exercise price that serves as a boundary of continuation region. 
\begin{align}
  \begin{cases}
    \dfrac{\partial{V}}{\partial{t}} + \dfrac{1}{2}\sigma^{2} S^2 \dfrac{\partial^2{V}}{\partial{S^2}} + (r - \delta)S \dfrac{\partial{V}}{\partial{S}} - rV = 0 & \text{for $(S, t) \in \mathcal{C}$} \\ V(S, t) = H(S, t) &
    \text{for $(S,t)\in \partial\mathcal{C}$}
  \end{cases}
  \label{eq:blackscholes:preliminaries:american_options_pde_free_boundary_problem}
\end{align}
The problem above is also known as the free boundary formulation of the pricing problem because it requires to solve a PDE with a moving boundary. The terminal
condition of the PDE will be given at time $T$. At the expiration date of the contract, the holder will either exercise or not the option. Therefore, the value of the option will be equal to the payoff function. Obviously, in that case, the optimal exercise price $S(T)$ will be equal to the strike price $K$. Hence,
\begin{align}
  V(S,T) = H(S, T), \qquad \bar{S}(T) = K
  \label{eq:blackscholes:preliminaries:american_options_terminal_condition}
\end{align}
Next, we need to establish boundary conditions for the system
\eqref{eq:blackscholes:preliminaries:american_options_pde_free_boundary_problem}.
Generally, when pricing options, we need two boundaries conditions. 
As it can be observed in figure (\ref{fig:blackscholes:preliminaries:american_option_value_vs_curve}), for call options, the left boundary condition $V(0,t)=0$ is given at $S=0$ and the right boundary conditions $V(\bar{S}(t),t)=\bar{S}(t) - K$ is given at the optimal exercise price $\bar{S}(t)$. Analogously, for put options, the left boundary condition $V(\bar{S}(t),t)=K - \bar{S}(t)$ at $\bar{S}(t)$ and right boundary is $V(S, t)=0$ for an arbitrary large $S$. Finally, $V(S,t)$ touches the payoff $H(S,t)$ tangentially at the optimal exercise price $\bar{S}(t)$
\begin{subequations} \label{eq:blackscholes:preliminaries:smooth_passing_condition}
  \begin{align}
    \text{\textbf{Call:}} \qquad &\dfrac{\partial{V}}{\partial{S}}(\bar{S}(t), t) = 1\\
    \text{\textbf{Put:}} \qquad &\dfrac{\partial{V}}{\partial{S}}(\bar{S}(t), t) = -1
  \end{align}
\end{subequations}
which is called the smooth pasting condition and later on will help us in obtaining $\bar{S}(t)$. By grouping \eqref{eq:blackscholes:preliminaries:american_options_pde_free_boundary_problem}, \eqref{eq:blackscholes:preliminaries:american_options_terminal_condition} and \eqref{eq:blackscholes:preliminaries:smooth_passing_condition} in one equation,
we obtain the system

\begin{subequations} \label{eq:blackscholes:preliminaries:american_options_pde_free_boundary_problem_full}
  \begin{align}
    \text{\textbf{Call}:} \quad &
    \begin{cases}
      \dfrac{\partial{V}}{\partial{t}} + \dfrac{1}{2}\sigma^{2} S^2 \dfrac{\partial^2{V}}{\partial{S}^2} + (r - \delta)S\dfrac{\partial{V}}{\partial{S}} - rV = 0 & \text{for $S \in (0,\bar{S}(t))$ and $t \in [0, T)$} \\ 
      V(S, T) = S - K \\
      \bar{S}(T) = K \\ 
      V(0, t) = 0 \\
      \dfrac{\partial{V}}{\partial{S}}(\bar{S}(t), t) = 1
    \end{cases} \\
    \text{\textbf{Put}:} \quad &
    \begin{cases}
      \dfrac{\partial{V}}{\partial{t}} + \dfrac{1}{2}\sigma^{2} S^2 \dfrac{\partial^2{V}}{\partial{S}^2} + (r - \delta)S\dfrac{\partial{V}}{\partial{S}} - rV = 0 & \text{for $S \in (\bar{S}(t), \infty)$, and $t \in [0, T)$} \\
      V(S, T) = K - S \\
      \bar{S}(T) = K \\ 
      \lim_{S\rightarrow\infty}V(S, t) = 0 \\ 
      \dfrac{\partial{V}}{\partial{S}}(\bar{S}(t), t) = -1
    \end{cases}
  \end{align}
\end{subequations}
\subsection{Front-Fixing method}
In the previous section, we presented the pricing of American options problem.
By applying the Black-Scholes model, we derived the Black-Scholes PDE that
describes the price dynamics in the continuation region $\mathcal{C}$ of call
and put options. Moreover, we presented the moving boundary condition
$\bar{S}(t)$ for this PDE. The moving boundary condition $\bar{S}(t)$ makes the
Black-Scholes PDE more involved since we also need to determine this boundary
as time changes. This type of problems are known as free boundary problems. The
front fixing method is a strategy in which a transformation is used to map the  domain from the original problem to a new domain where moving boundary remains fixed as time changes. In this section, we explore two transformation based on the work of Nielsen et al. \cite{nielsen_2001}, and the work of Company and et al. \cite{company_egorova_jodar_2014}.
\subsubsection{Nielsen transformation} \label{sec:blackscholes:frontfixingmethod:inversetransform}
The Nielsen transformation suggests a really simple transformation in which
the asset price $S$ is divided by the optimal exercise price $\bar{S}$
\begin{equation}
  x = \dfrac{S}{\bar{S}(t)}
  \label{eq:blackscholes:frontfixingmethod:inversetransform}
\end{equation}
Clearly, the moving boundary in the original problem will be fixed when $S=\bar{S}(t)$ at $x=1$. Now, we define $v(x,t)$ as the value function of the option but under the front fixing domain given by $x$
\begin{equation}
  v(x, t) := V(S, t)
  \label{eq:blackscholes:frontfixingmethod:inversetransform:value_function}
\end{equation}
Moreover, we want to understand how this transformation affects the Black-Scholes PDE, the boundary, terminal and contact point conditions given in equation in \eqref{eq:blackscholes:preliminaries:american_options_pde_free_boundary_problem_full}.

Firstly, we start with the Black-Scholes PDE which is defined at the interval 
$S\in(0, \bar{S}(t))$ for call options or the open interval $S\in(\bar{S}(t), \infty)$ for put options. Under the front fixing domain, the transformed PDE will be defined in the interval $x\in(0, 1)$ for call and $x\in(1, \infty)$ for put. Moreover, we apply the chain rule to rewrite the Black-Scholes PDE in terms of $v$, so that, the new PDE is given as
\begin{subequations} \label{eq:blackscholes:frontfixingmethod:american_options_pde}
  \begin{align}
    \text{\textbf{Call:}} \qquad
    \dfrac{\partial{v}}{\partial{t}} + \dfrac{1}{2}\sigma^{2} x^2 \dfrac{\partial^2{v}}{\partial{x}^2} + \bigg[(r - \delta) - \dfrac{\bar{S}^\prime(t)}{\bar{S}(t)}\bigg]x\dfrac{\partial{v}}{\partial{x}} - rv = 0 \quad & \text{for $x \in [0, 1)$ and $t \in [0, T)$} \\
    \text{\textbf{Put:}} \qquad
    \dfrac{\partial{v}}{\partial{t}} + \dfrac{1}{2}\sigma^{2} x^2 \dfrac{\partial^2{v}}{\partial{x}^2} + \bigg[(r - \delta) - \dfrac{\bar{S}^\prime(t)}{\bar{S}(t)}\bigg]x\dfrac{\partial{v}}{\partial{x}} - rv = 0 \quad & \text{for $x > 1$ and $t \in (0, T]$}
  \end{align}
\end{subequations}
Similarly, we express the boundary conditions in the front fixing domain. We already stated that the Nielsen transformation fixes the moving boundary $\bar{S}$ at $x=1$. Additionally, $x$ goes to infinity as $S$ goes to infinity, and $x=0$ for $S=0$. Therefore, the boundary condition opposite to the optimal exercise price $\bar{S}(t)$ will remain as in the original problem. Hence,
as it can be observed in figure (\ref{fig:blackscholes:frontfixingmethod:nielsen_value_vs_curve}), the call option has left boundary condition $v(0, t) = 0$ at $x=0$ and right boundary condition $v(1, t) = \bar{S}(t) - K$ at $x=1$. Alternatively, 
the put option has left boundary condition $v(1, t) = K - \bar{S}(t)$ at $x=1$ and right boundary condition $v(x, t) = 0$ at a sufficiently large $x$.
\begin{figure}[H]
  \centering
  \begin{subfigure}{0.45\textwidth}
    \centering
    \includegraphics[width=\textwidth]{chapters/chapter2/NielsenCallOption}
    \caption{Call option}
    \label{fig:blackscholes:frontfixingmethod:nielsen_call_value_vs_curve}
  \end{subfigure}
  \begin{subfigure}{0.45\textwidth}
    \centering
    \includegraphics[width=\textwidth]{chapters/chapter2/NielsenPutOption}
    \caption{Put option}
    \label{fig:blackscholes:frontfixingmethod:nielsen_put_value_vs_curve}
  \end{subfigure}
  \caption{Value $v(x,t) := V(S,t)$ in the front fixing domain defined by Nielsen transformation.}
  \label{fig:blackscholes:frontfixingmethod:nielsen_value_vs_curve}
\end{figure}
Likewise, we express the contact point condition in terms of $v(x,t)$. Recall that at the contact point, the slope $V(S,t)$ with respect to $S$ is the same as the slope of the linear segment in the payoff function. This can be seen clearly in figure (\ref{fig:blackscholes:preliminaries:american_option_value_vs_curve}). Hence, by the chain rule, the contact point condition of $v(x,t)$ is given by
\begin{subequations} \label{eq:blackscholes:frontfixingmethod:inversetransform:american_options_optimal_price_contact_point_condition}
  \begin{align}
    \text{\textbf{Call:}} \qquad & \dfrac{\partial v}{\partial x}(1, t) = \bar{S}(t) \\
    \text{\textbf{Put:}} \qquad & \dfrac{\partial v}{\partial x}(1, t) = -\bar{S}(t)
  \end{align}
\end{subequations}
Finally, recall that the terminal condition of $\bar{S}(t)$ is given by \eqref{eq:blackscholes:preliminaries:american_options_terminal_condition}. Moreover,
$x>=1$ for call options, and $x<=1$ for put options. Hence, by simple substitution, we can rewrite the terminal conditions of $v(x, t)$ as
\begin{subequations} \label{eq:blackscholes:frontfixingmethod:inversetransform:american_options_terminal_condition}
  \begin{align}
    \text{\textbf{Call:}} \qquad & v(x, T) = \max(x\bar{S}(T) - K) = K \max(x - 1, 0) = 0 \\
    \text{\textbf{Put:}} \qquad & v(x, T) = \max(K - x\bar{S}(T)) = K \max(1 - x, 0) = 0
  \end{align}
\end{subequations}
In summary, by groping equations
\eqref{eq:blackscholes:frontfixingmethod:american_options_pde},
\eqref{eq:blackscholes:frontfixingmethod:inversetransform:american_options_terminal_condition}, and
\eqref{eq:blackscholes:frontfixingmethod:inversetransform:american_options_optimal_price_contact_point_condition},
we obtain the system
\begin{subequations} \label{eq:blackscholes:frontfixingmethod:inversetransform:american_options_bs_pde}
  \begin{align}
    \text{\textbf{Call:}} \quad &
    \begin{cases}
      \dfrac{\partial{v}}{\partial{t}} + \dfrac{1}{2}\sigma^{2} x^2 \dfrac{\partial^2{v}}{\partial{x}^2} + \bigg[(r - \delta) -
      \dfrac{\bar{S}^\prime(t)}{\bar{S}(t)}\bigg]x\dfrac{\partial{v}}{\partial{x}} - rv = 0 & \text{for $x \in (0, 1)$ and $t \in [0, T)$} \\ 
      v(x, T) = 0 & \text{for $x\in[0, 1]$}  \\
      \bar{S}(T) = K \\ 
      v(0, t) = 0 & \text{for $t\in[0, T)$}\\ 
      v(1, t) = \bar{S}(t) - K & \text{for $t\in[0, T)$}\\ 
      \dfrac{\partial{v}}{\partial{x}}(1, t) = \bar{S}(t) & \text{for $t\in[0, T)$}
    \end{cases}\\
    \text{\textbf{Put:}} \quad &
    \begin{cases}
      \dfrac{\partial{v}}{\partial{t}} + \dfrac{1}{2}\sigma^{2} x^2 \dfrac{\partial^2{v}}{\partial{x}^2} + \bigg[(r - \delta) -
      \dfrac{\bar{S}^\prime(t)}{\bar{S}(t)}\bigg]x\dfrac{\partial{v}}{\partial{x}} - rv = 0 & \text{for $x > 1$ and $t \in [0, T)$} \\ 
      v(x, T) = 0 & \text{for $x \ge 1$} \\
      \bar{S}(T) = K \\ 
      v(1, t) = K - \bar{S}(t) & \text{for $t\in[0, T)$} \\
      \lim_{x\rightarrow\infty}v(x, t) = 0 & \text{for $t\in[0, T)$} \\
      \dfrac{\partial{v}}{\partial{x}}(1, t) = -\bar{S}(t) & \text{for $t\in[0, T)$}
    \end{cases}
  \end{align}
\end{subequations}
\subsubsection{Company transformation}
The Company transformation proposes set of change of variable for the asset price $S$, the time $t$, the value function $V(S,t)$ and the moving boundary
\begin{equation}
  x := \log \dfrac{S}{\bar{S}_f(t)}, \quad \tau := T - t, \quad v(x, \tau) := \dfrac{V(S, t)}{K}, \quad \bar{S}_f(\tau) := \dfrac{\bar{S}(t)}{K} 
\end{equation}
Let us break down the transformations. Firstly, the transformation proposed are written forward in time. Therefore, $\tau = 0$ refers to the expiration date of the options $t=T$. Secondly, both the value function and the optimal exercise price is scaled by the strike price. Finally, the new moving boundary is fixed at $S=\bar{S}_f(t)$ or $x=0$.

Similarly, as we did for the Nielsen method, we rewrite the Black-Scholes PDE in terms of $v(x, \tau)$. Note that as $x$ goes to infinity $S$ goes to infinity. Conversely, as $x$ goes to negative infinity $S$ goes to zero. Moreover, $S=\bar{S}(t)$ at $x=0$. Using the previous information, we deduce  that the Black-Scholes PDE is defined in the intervals $x\in(-\infty, 0)$ for call options and $x\in(0, \infty)$ for put options. Therefore, we have
\begin{subequations}
  \begin{align}
      \text{\textbf{Call:}} \quad \dfrac{\partial v}{\partial \tau} - \dfrac{1}{2}\sigma^2\dfrac{\partial^2 v}{\partial x^2} - \bigg((r-\delta) + \dfrac{\sigma^2}{2} - \dfrac{\bar{S}'(\tau)}{\bar{S}(\tau)} \bigg)\dfrac{\partial v}{\partial x} + rv = 0 \quad \text{for $x < 0$ and $\tau \in (0, T]$} \\
      \text{\textbf{Put:}} \quad \dfrac{\partial v}{\partial \tau} - \dfrac{1}{2}\sigma^2\dfrac{\partial^2 v}{\partial x^2} - \bigg((r-\delta) - \dfrac{\sigma^2}{2} - \dfrac{\bar{S}'(\tau)}{\bar{S}(\tau)} \bigg)\dfrac{\partial v}{\partial x} + rv = 0 \quad \text{for $x > 0$ and $\tau \in (0, T]$}
  \end{align}
\end{subequations}
Note that the term on the new PDE that correspond to the terms in the linear parabolic operator $\mathcal{L}V$ defined in \eqref{eq:blackscholes:preliminaries:linear_parabolic_operator} are negative because the Black-Scholes PDE was inverted in time.

Again, the boundary conditions for the call option in the original domain are
$V(0,t)$ at $S=0$ and $V(\bar{S}, t) = \bar{S} - K$  at $S=\bar{S}(t)$, and when transforming those boundary conditions to the front fixing domain, they become $v(x, \tau) = 0$ for a sufficiently negative $x$ and $v(0, \tau) := \bar{S}_f(\tau) - 1 = V(\bar{S}, t) / K$ at $x=0$. Similarly, the boundary conditions for the put option in the original domain are $V(\bar{S}(t), t) = K - \bar{S}$ at $S=\bar{S}(t)$ and $V(S, t) = 0$ for a sufficiently large $S$, and under the front fixing domain, they become $v(0, \tau) = 1 - \bar{S}_f(\tau) = V(\bar{S}, t) / K$ at $x=0$ and $v(x, \tau) = 0$ for a sufficiently large $x$. Similarly, to as we did for the Nielsen transformation, we also rewrite the contact point condition
\begin{subequations}
  \begin{align}
    \text{\textbf{Call:}} \qquad & \dfrac{\partial v}{\partial x}(0, \tau) =  \bar{S}_f(\tau)\\
    \text{\textbf{Put:}} \qquad & \dfrac{\partial v}{\partial x}(0, \tau) = -\bar{S}_f(\tau)
  \end{align}
\end{subequations}
\begin{figure}[H]
  \centering
  \begin{subfigure}{0.4\textwidth}
    \centering
    \includegraphics[width=\textwidth]{chapters/chapter2/CompanyCallOption.pdf}
    \caption{Call option}
    \label{fig:blackscholes:frontfixingmethod:company_call_value_vs_curve}
  \end{subfigure}
  \begin{subfigure}{0.5\textwidth}
    \centering
    \includegraphics[width=\textwidth]{chapters/chapter2/CompanyPutOption.pdf}
    \caption{Put option}
    \label{fig:blackscholes:frontfixingmethod:company_put_value_vs_curve}
  \end{subfigure}
  \caption{Value $v(x,t) := V(S,t) / K$ in the front fixing domain defined by Company transformation.}
  \label{fig:blackscholes:frontfixingmethod:company_value_vs_curve}
\end{figure}
Since the transformed PDE is forward in time, we have to come up with initial conditions for $\bar{S}_f(\tau)$ and $v(x, \tau)$. For $\bar{S}_f(\tau)$, the initial condition is given by
\begin{equation}
  \bar{S}_f(0) = \dfrac{\bar{S}(T)}{K} = 1
\end{equation}
Moreover, for call options, the initial condition is given by $v(x, 0) = V(S, T) / K = \max\big(\bar{S}_f(0)e^{x} - 1, 0\big) = \max\big(e^{x} - 1, 0\big) = 0$ since $x$ is always negative. Similarly, for put options, the initial condition is given by $v(x, 0) = V(S, T) / K = \max\big(1 - \bar{S}_f(0)e^x, 0\big) = \max\big(1 - e^x, 0\big) = 0$ since $x$ is always positive. Hence,
\begin{subequations} \label{eq:blackscholes:frontfixingmethod:logtransform:american_options_terminal_condition}
  \begin{align}
    \text{\textbf{Call:}} \qquad & v(x, 0) = 0 \\
    \text{\textbf{Put:}} \qquad & v(x, 0) = 0
  \end{align}
\end{subequations}
Finally, grouping the equations together, we have the system
\begin{subequations} \label{eq:blackscholes:frontfixingmethod:logtransform:american_options_bs_pde}
  \begin{align}
    \text{\textbf{Call:}} \quad &
    \begin{cases}
      \dfrac{\partial v}{\partial \tau} - \dfrac{1}{2}\sigma^2\dfrac{\partial^2 v}{\partial x^2} - \bigg((r-\delta) + \dfrac{\sigma^2}{2} - \dfrac{\bar{S}'(\tau)}{\bar{S}(\tau)} \bigg)\dfrac{\partial v}{\partial x} + rv = 0 & \text{for $x < 0$ and $\tau \in (0, T]$}  \\ 
      v(x, 0) = 0 & \text{for $x < 0$} \\ 
      \bar{S}_f(0) = 1  \\ 
      \lim_{x\rightarrow-\infty}{v(x, \tau)} = 0 & \text{for $\tau \in (0, T]$}  \\ 
      \dfrac{\partial{v}}{\partial{x}}(0, \tau) = \bar{S}_f(\tau) & \text{for $\tau \in (0, T]$} 
    \end{cases} \\
    \text{\textbf{Put:}} \quad &
    \begin{cases}
      \dfrac{\partial v}{\partial \tau} - \dfrac{1}{2}\sigma^2\dfrac{\partial^2 v}{\partial x^2} - \bigg((r-\delta) - \dfrac{\sigma^2}{2} - \dfrac{\bar{S}'(\tau)}{\bar{S}(\tau)} \bigg)\dfrac{\partial v}{\partial x} + rv = 0 & \text{for $x > 0$ and $\tau \in (0, T]$} \\ 
      v(x, 0) = 0 & \text{for $x > 0$}\\
      \bar{S}_f(0) = 1 \\ 
      \lim_{x\rightarrow\infty} v(x, \tau) = 0 & \text{for $\tau \in (0, T]$} \\ 
      \dfrac{\partial{v}}{\partial{x}}(0, \tau) = -\bar{S}_f(\tau) & \text{for $\tau \in (0, T]$}
    \end{cases}
  \end{align}
\end{subequations}
\section{Finite difference schemes} \label{sec:finitedifferencesschemes}
\subsection{Overview}

In this section, we present explicit and implicit central finite difference schemes for solving the PDE problem in \eqref{eq:blackscholes:frontfixingmethod:inversetransform:american_options_bs_pde}. Previously, we considered the pricing problem of American options which requires solving the free boundary problem defined in \eqref{eq:blackscholes:preliminaries:american_options_pde_free_boundary_problem_full}.
Then, we presented the front fixing method as a strategy to fix the moving boundary using a change of variable. Moreover, we derived the PDE problem for call and put options that resulted from applying the Nielsen transformation suggested by \cite{nielsen_2001}, and the Company transformation suggested by \cite{company_egorova_jodar_2014}, resulting in the systems \eqref{eq:blackscholes:frontfixingmethod:inversetransform:american_options_bs_pde} and \eqref{eq:blackscholes:frontfixingmethod:logtransform:american_options_bs_pde}, respectively. In the following part, we present numerical methods for solving \eqref{eq:blackscholes:frontfixingmethod:inversetransform:american_options_bs_pde}. But before we jump into that, we define what it means to compute a numerical solution to a PDE problem.

Recall that the solution $v(x,t)$ of 
\eqref{eq:blackscholes:frontfixingmethod:inversetransform:american_options_bs_pde} is defined in the continuous region 
\begin{subequations}
  \label{eq:finitedifferencesschemes:overview:continous_domain}
\begin{align}
  \text{\textbf{Call:}} \qquad& \mathcal{T}: [0, T], \qquad \mathcal{X}: [0, 1], \qquad  \mathcal{F}: \mathcal{X} \times \mathcal{T} \\
  \text{\textbf{Put:}} \qquad& \mathcal{T}:[0, T], \qquad \mathcal{X}:[1, \infty), \qquad \mathcal{F}: \mathcal{X} \times \mathcal{T},
\end{align}
\end{subequations}
Now, we want to discretize $\mathcal{F}$ using the grid $\mathcal{G}$ with $N+1$ and $M+1$ nodes
\begin{align}
  \label{eq:finitedifferencesschemes:overview:grid}
  \mathcal{G} := \{(x_i, t_n): (i, n) \in \{0,\dots,M+1\}\times\{0,\dots,N+1\}\}
\end{align}
where
\begin{align}
  \label{eq:finitedifferencesschemes:overview:grid_2}
  x_i &:= x_{\text{min}} + i\Delta x &  \qquad \text{for $i = 0,\dots, M+1$} \\
  t_n &:= t_{\text{min}} + i{\Delta t} & \qquad \text{for $i = 0,\dots, N+1$} \\
  \Delta{x} &:= \dfrac{x_{\text{max}} - x_{\text{min}}}{M+1} \\ 
  \Delta{t} &:= \dfrac{t_{\text{max}} - t_{\text{min}}}{N+1}
\end{align}
Each contiguous node will be separated by $\Delta{x}$ on the spatial axis and $\Delta{t}$ on the temporal axis. As $\Delta{x}$ and $\Delta{t}$ decreases, the number of nodes in the grid will increase. Therefore, we refer to $\Delta{x}$ and $\Delta{t}$ as the resolution of the grid.
\begin{figure}[H]
  \label{fig:finitedifferencesschemes:overview:grid}
  \centering
  \includegraphics[scale=0.5]{chapters/chapter3/GridAproximation.pdf}
  \caption{The grid $\mathcal{G}$ and the approximation $v^{n}_{i} \approx v(x_i, t_n)$ in each node.}
\end{figure}
From \eqref{eq:finitedifferencesschemes:overview:continous_domain}, it is clear that $t_\text{min} = 0$ and $t_\text{max}=T$. Moreover, for call options, $x_\text{min} = 0$ and $x_\text{max}=1$. Likewise, for put options, $x_\text{min}=1$ and $x_\text{max}=x_\infty$ where $x_\infty$ is arbitrary large value.
Now that we defined our grid, our goal is to approximate the value function $v(x, t)$ and the optimal exercise price $\bar{S}(t)$ at each node of the grid $\mathcal{G}$
\begin{align*}
  v^{n}_i \approx v(x_i,t_n), \quad \bar{S}^{n} \approx \bar{S}(t_n)
\end{align*}
Moreover, we want that the error of the approximation converges to zero value. Specifically, we want that the approximation error at each node 
\begin{align}
  \label{eq:finitedifferencesschemes:overview:local_truncation_error}
  e^{n}_i := v^{n}_i-v(x_i, t_n)
\end{align}
goes to zero as $\Delta{x}$ and $\Delta{t}$ decrease. \eqref{eq:finitedifferencesschemes:overview:local_truncation_error} is the local truncation error, and it measures the approximation error at time $t_n$. It is important to state if a single node has inferior order than the rest of the nodes, it might degrade the order of the truncation error in overall.   

Finally, we need ways to approximate derivatives. Here is where finite differences schemes come into play. The idea of finite differences is trivial which is approximating derivatives as the difference of contiguous nodes in the grid. Let us say we are at point $x$, then the forward differences approximate the derivative as
\begin{align*}
 \dfrac{f(x + h) - f(x)}{h} = \dfrac{df}{dx} + O(h)
\end{align*}
Conversely, the backward difference approximate the derivative as 
\begin{align*}
  \dfrac{f(x) - f(x-h)}{h} = \dfrac{df}{dx} + O(h)
\end{align*}
As you can observe forward and backward difference approximation yield a local truncation error of $O(h)$. Moreover, the central finite difference approximate the first order derivative as 
\begin{align*}
  \dfrac{f(x+h) - f(x-h)}{2h} = \dfrac{df}{dx} + O(h^2)
 \end{align*}
and for second order derivatives as
\begin{align*}
  \dfrac{f(x+h) - 2f(x) + f(x-h)}{h^2} = \dfrac{d^2f}{dx^2} + O(h^2)
 \end{align*}
Note that both approximation offers a better order of convergence that forward and backward difference, but you are required to come up with strategies for approximating the derivative at the boundary of your grid where $x+h$ or $x-h$ is not defined.

\subsection{Explicit scheme}
Generally, explicit schemes use forward finite difference to approximate the temporal partial derivative and central finite difference to approximate the spatial derivative at time $t_{n+1}$ and position $x_i$. However, since the problem \eqref{eq:blackscholes:frontfixingmethod:inversetransform:american_options_bs_pde} is written backward in time, we use backward finite difference at $t_{n+1}$, and a central finite difference at $x_i$.

\begin{figure}[H]
  \centering
  \includegraphics[scale=.8]{chapters/chapter3/ExplicitStencil.pdf}
  \caption{Stencil diagram of the explicit scheme.}
  \label{fig:finitedifferencesschemes:explicit_stencil}
\end{figure}

The central finite difference for the first order and second order spatial partial derivative is given by
\begin{align}
  \label{eq:finitedifferencesschemes:explicit:spatial_first_order_central_finite_difference}
  \dfrac{v^{n+1}_{i+1} - v^{n+1}_{i-1}}{2 \Delta{x}} =& \dfrac{\partial{v}}{\partial{x}}+ O(\Delta{x}^2) \qquad & \text{for $i = 1, \dots, M$} \\
  \label{eq:finitedifferencesschemes:explicit:spatial_second_order_central_finite_difference}
  \dfrac{v^{n+1}_{i+1} - 2v^{n+1}_{i} + v^{n+1}_{i-1}}{\Delta{x}^2} =& \dfrac{\partial^2{v}}{\partial{x^2}}+ O(\Delta{x}^2) \qquad & \text{for $i = 1, \dots, M$}
\end{align}
As it can be observed in figure \eqref{fig:finitedifferencesschemes:explicit_stencil}, the first and second order central finite difference approximations at node $(x_i, t_{n+1})$ require to compute the difference at the nodes $(x_{i-1}, t_{n+1})$ and $(x_{i+1}, t_{n+1})$. Hence, we can only approximate the spatial partial derivative at the internal region of the grid $\mathcal{G}$ given by the nodes $(x_i, t_n)$ for $i=1,\dots,M$. Also note, that the central finite difference has second order convergence in space. In other words, as we decrease $\Delta{x}$ by one decimal place, the approximation error will decrease by two decimal places.

Analogously, the backward difference approximation at $t_{n+1}$ for $v(x, t)$ and the optimal exercise price $\bar{S}(t)$ is given by
\begin{align}
  \label{eq:finitedifferencesschemes:explicit:temporal_backward_finite_difference}
  \dfrac{v^{n+1}_{i} - v^{n}_{i}}{\Delta{t}} &= \dfrac{\partial{v}}{\partial{t}}+ O(\Delta{t}) \qquad & \text{for $n = N,\dots,0$ } \\
  \label{eq:finitedifferencesschemes:explicit:front_temporal_backward_finite_difference}
  \dfrac{\bar{S}^{n+1}-\bar{S}^{n}}{\Delta t} &= \bar{S}'(t) + O(\Delta{t}) \qquad & \text{for $n = N,\dots,0$ }
\end{align}
Contrary to the central finite difference, the backward finite difference approximations have first order convergence in time. While it would be desirable to have second order convergence for the temporal partial derivative approximation, it is not possible use central finite difference because we would be required to have two boundary conditions in the time axis. By combining the finite difference approximations \eqref{eq:finitedifferencesschemes:explicit:spatial_first_order_central_finite_difference}, \eqref{eq:finitedifferencesschemes:explicit:spatial_second_order_central_finite_difference}, \eqref{eq:finitedifferencesschemes:explicit:temporal_backward_finite_difference}, and \eqref{eq:finitedifferencesschemes:explicit:temporal_backward_finite_difference},the approximation of the PDE in \eqref{eq:blackscholes:frontfixingmethod:inversetransform:american_options_bs_pde} is given by 
\begin{equation*}
  \begin{split}
    \dfrac{v^{n+1}_{i} - v^{n}_{i}}{\Delta{t}} & + \dfrac{1}{2}\sigma^2 x_i^2 \dfrac{v^{n+1}_{i-1} - 2v^{n+1}_{i} + v^{n+1}_{i+1}}{(\Delta{x})^2} \\ 
     & + x_i\bigg( (r-\delta) - \dfrac{1}{\bar{S}^{n+1}}\dfrac{\bar{S}^{n+1} - \bar{S}^{n}}{\Delta{t}} \bigg)\dfrac{v^{n+1}_{i+1} - v^{n+1}_{i-1}}{2\Delta{x}} - rv^{n+1}_{i} = 0
  \end{split}
\end{equation*}
for $i = 1, \dots, M$ and $n = N, \dots, 0$. To simplify the expression above, we introduce the terms 
\begin{align*}
  \lambda &:= \dfrac{\Delta{t}}{(\Delta{x})^2} \\
  A_i &:= \dfrac{\lambda}{2}\sigma^2x^{2}_i - \dfrac{\lambda}{2}\bigg((r-\delta) - \dfrac{1}{\Delta{t}}\bigg)x_i\Delta{x} & \text{for $i = 1, \dots, M$} \\ 
  B_i &:= 1 - \lambda\sigma^2x_i^2 - r\Delta{t} & \text{for $i = 1, \dots, M$} \\
  C_i &:= \dfrac{\lambda}{2}\sigma^2x^{2}_i + \dfrac{\lambda}{2}\bigg((r-\delta) - \dfrac{1}{\Delta{t}}\bigg)x_i\Delta{x} &  \text{for $i = 1, \dots, M$} \\
  D^{n+1}_{i} &:= \dfrac{x_i}{2\Delta{x}}\dfrac{v^{n+1}_{i+1} - v^{n+1}_{i-1}}{\bar{S}^{n+1}} &  \text{for $i = 1, \dots, M$}
\end{align*}
Then, we rearrange the finite difference approximation of the PDE as 
\begin{equation}
  v^{n}_{i} - D^{n+1}_{i}\bar{S}^n = A_i v^{n+1}_{i-1} + B_{i}v^{n+1}_{i} + C_{i}v^{n+1}_{i+1}
  \label{eq:finitedifferencesschemes:explicit:pde_simplified}
\end{equation}
for $i = 1, \dots, M$ and $t = N, \dots, 0$. Moreover, the PDE problem in \eqref{eq:blackscholes:frontfixingmethod:inversetransform:american_options_bs_pde} have well-defined spatial boundary conditions. For call options, the boundary conditions are located at $x=0$ and $x=1$. Similarly, for put options, the boundary conditions are located at $x=1$ and at a sufficient large $x$. However, since the $\mathcal{G}$ is defined in terms of $x_\text{min}$ and $x_\text{max}$, regardless of the option type, the boundary conditions will be always at $x_0$ and $x_{M+1}$.
\begin{subequations}
  \label{eq:finitedifferencesschemes:explicit:boundary_conditions}
  \begin{align}
    \text{\textbf{Call:}} \qquad & v^{n}_{0} = 0, \qquad v^{n}_{M+1} = \bar{S}^{n} - K\\
    \text{\textbf{Put:}} \qquad & v^{n}_{0} = K - \bar{S}^{n}, \qquad v^{n}_{M+1} = 0
  \end{align}
\end{subequations}
Likewise, the terminal conditions are located at $t_{N+1}$ $i=0,\dots,M+1$
\begin{subequations}
  \label{eq:finitedifferencesschemes:explicit:terminal_conditions}
  \begin{equation}
    v^{N+1}_{i} = 0, \qquad \bar{S}^{N+1} = K
  \end{equation}
\end{subequations}
Moreover, for the problem \eqref{eq:blackscholes:frontfixingmethod:inversetransform:american_options_bs_pde}, we have contact point condition \eqref{eq:blackscholes:frontfixingmethod:inversetransform:american_options_optimal_price_contact_point_condition}. The contact point condition gives the slope at $x=1$. When the option is a call option, $x=1$ correspond to $x_{M+1}$ in the grid $\mathcal{G}$. Reciprocally, for a put option, $x=1$ correspond to $x_0$. Therefore, by using backward difference at $x_{M+1}$ and forward difference at $x_0$, the contact point approximation for call and put options, respectively.
\begin{align*}
  \text{\textbf{Call:}} \qquad & \dfrac{v^{n}_{M+1} - v^{n}_{M}}{\Delta{x}} = \dfrac{\partial{v}}{\partial{x}}(1, t) + O(\Delta{x}) \\
  \text{\textbf{Put:}} \qquad & \dfrac{v^{n}_{1} - v^{n}_{0}}{\Delta{x}} = \dfrac{\partial{v}}{\partial{x}}(1, t)+ O(\Delta{x}) 
\end{align*}
Using the contact point condition, we obtain an explicit expression for $v^{n}_{M}$
\begin{subequations}
  \label{eq:finitedifferencesschemes:explicit:contact_point_approximation_2}
  \begin{align}
    \text{\textbf{Call:}} \qquad& v^{n}_{M} = v^{n}_{M+1} - \Delta{x}\bar{S}^{n} = (1-\Delta{x})\bar{S}^n - K & \qquad \text{for $n = N,\dots,0$ }\\
    \text{\textbf{Put:}} \qquad& v^{n}_{1} = v^{n}_{0} - \Delta{x}\bar{S}^{n} = K - (1+\Delta{x})\bar{S}^n & \qquad \text{for $n = N,\dots,0$ }
  \end{align}    
\end{subequations}
Note that the approximation for $v^{n}_{M}$ has first order convergence in space which could degrade the global convergence of the explicit method to first order in space even if we are using central finite difference to approximate the spatial partial derivatives of $v(x,t)$. Similarly, we can obtain explicit expression for $\bar{S}^{n}$ by computing \eqref{eq:finitedifferencesschemes:explicit:pde_simplified} at $x_M$ and at $x_1$ for call and put options, respectively. Then, rearranging the resulting expression in terms of $\bar{S}^n$ 
\begin{subequations}
  \label{eq:finitedifferencesschemes:explicit:optimal_exercise_price_approximation}
  \begin{align}
    \text{\textbf{Call:}} \qquad& \bar{S}^{n} = \dfrac{K + A_{M}v^{n+1}_{M-1} + B_{M}v^{n+1}_{M} + C_{M}v^{n+1}_{M+1}}{(1-\Delta{x}) - D^{n+1}_{M}} \\
    \text{\textbf{Put:}} \qquad& \bar{S}^{n} = \dfrac{K - (A_{1}v^{n+1}_{0} + B_{1}v^{n+1}_{1} + C_{1}v^{n+1}_{2})}{D^{n+1}_1 + (1+\Delta{x})}
  \end{align}
\end{subequations}
for $n = N,\dots,0$. Thus, combining \eqref{eq:finitedifferencesschemes:explicit:pde_simplified}, \eqref{eq:finitedifferencesschemes:explicit:boundary_conditions}, 
\eqref{eq:finitedifferencesschemes:explicit:terminal_conditions},
\eqref{eq:finitedifferencesschemes:explicit:contact_point_approximation_2}, and \eqref{eq:finitedifferencesschemes:explicit:optimal_exercise_price_approximation}, the explicit scheme of PDE problem \eqref{eq:blackscholes:frontfixingmethod:inversetransform:american_options_bs_pde} is given by
\begin{subequations}
  \label{eq:finitedifferencesschemes:explicit:nielsen_system_of_equation}
  \begin{align}
    \text{\textbf{Call:}} \quad& \begin{cases}
      v^{n}_{i} - D^{n+1}_{i}\bar{S}^n = A_i v^{n+1}_{i-1} + B_{i}v^{n+1}_{i} + C_{i}v^{n+1}_{i+1} & \text{for $i = 1, \dots, M-1$ and $n = N,\dots, 0$}\\
      v^{N+1}_i = 0 & \text{for $i = 0, \dots, M+1$}  \\
      \bar{S}^{N+1} = K \\
      v^{n}_0 = 0 & \text{for $n = N, \dots, 0$} \\ 
      v^{n}_{M} = (1-\Delta{x})\bar{S}^n - K & \text{for $n = N, \dots, 0$} \\
      v^{n}_{M+1} = \bar{S}^n - K  & \text{for $n = N, \dots, 0$}
    \end{cases}\\
    \text{\textbf{Put:}} \quad&  \begin{cases}
      v^{n}_{i} - D^{n+1}_{i}\bar{S}^n = A_i v^{n+1}_{i-1} + B_{i}v^{n+1}_{i} + C_{i}v^{n+1}_{i+1} & \text{for $i = 2, \dots, M$ and $n = N,\dots,0$} \\
      v^{N+1}_i = 0 & \text{for $i = 0, \dots, M+1$} \\ 
      \bar{S}^{N+1} = K \\
      v^{n}_{0} = K - \bar{S}^{n} & \text{for $n = N, \dots, 0$} \\
      v^{n}_{1} =  K - (1+\Delta{x})\bar{S}^{n}  & \text{for $n = N, \dots, 0$} \\
      v^{n}_{M} = 0 & \text{for $n = N, \dots 0$}
    \end{cases}
  \end{align}
\end{subequations}
Finally, we formulate an algorithm for solving the system \eqref{eq:finitedifferencesschemes:explicit:nielsen_system_of_equation}
\begin{algorithm}[H]
  \caption{Explicit method for call options} \label{alg:finitedifferencesschemes:explicit:call_explicit_method_algorithm}
  \begin{algorithmic}
  \Ensure $\lambda \le 0.5$
  
  \For{$i = 0,\dots,M+1$} 
    \State $v^{N+1}_i = 0 $
  \EndFor
  
  \State $\bar{S}^{N+1} = K$

  \For{$i = 1,\dots,M$} 
    \State $A_i = \dfrac{\lambda}{2}\sigma^2x^{2}_i - \dfrac{\lambda}{2}\bigg((r-\delta) - \dfrac{1}{\Delta{t}}\bigg)x_i\Delta{x}$
    \State $B_i = 1 - \lambda\sigma^2x_i^2 - r\Delta{t} $
    \State $C_i = \dfrac{\lambda}{2}\sigma^2x^{2}_i + \dfrac{\lambda}{2}\bigg((r-\delta) - \dfrac{1}{\Delta{t}}\bigg)x_i\Delta{x} $
  \EndFor
  
  \For{$n = N, \dots, 0$}
    \For{$i = 1, \dots, M$}
      \State $D^{n+1}_i = \dfrac{x_i}{2\Delta{x}}\dfrac{v^{n+1}_{i+1} - v^{n+1}_{i-1}}{\bar{S}^{n+1}}$
    \EndFor
    \State $\bar{S}^n = \dfrac{K + A_{M}v^{n+1}_{M-1} + B_{M}v^{n+1}_{M} + C_{M}v^{n+1}_{M+1}}{(1-\Delta{x}) - D^{n+1}_{M}}$
    \State $v^{n}_{0} = 0$
    \State $v^{n}_{M} = (1-\Delta{x})\bar{S}^{n} - K$
    \State $v^{n}_{M+1} = \bar{S}^{n} - K$
    \For{$i = 1, \dots, M-1$}
      \State $v^{n}_{i} = A_i v^{n+1}_{i-1} + B_{i}v^{n+1}_{i} + C_{i}v^{n+1}_{i+1} + D^{n+1}_{i}\bar{S}^n$
    \EndFor
  \EndFor
\end{algorithmic}
\end{algorithm}

\begin{algorithm}[H]
  \caption{Explicit method for put options}\label{alg:finitedifferencesschemes:explicit:put_explicit_method_algorithm}
  \begin{algorithmic}
  \For{$i = 0,\dots,M+1$} 
    \State $v^{N+1}_i = 0 $
  \EndFor
  \State $\bar{S}^{N+1} = K$
  \For{$i = 1,\dots,M$} 
    \State $A_i = \dfrac{\lambda}{2}\sigma^2x^{2}_i - \dfrac{\lambda}{2}\bigg((r-\delta) - \dfrac{1}{\Delta{t}}\bigg)x_i\Delta{x}$
    \State $B_i = 1 - \lambda\sigma^2x_i^2 - r\Delta{t} $
    \State $C_i = \dfrac{\lambda}{2}\sigma^2x^{2}_i + \dfrac{\lambda}{2}\bigg((r-\delta) - \dfrac{1}{\Delta{t}}\bigg)x_i\Delta{x} $
  \EndFor
  \For{$n = N, \dots, 0$}
    \For{$i = 1, \dots, M$}
      \State $D^{n+1}_i = \dfrac{x_i}{2\Delta{x}}\dfrac{v^{n+1}_{i+1} - v^{n+1}_{i-1}}{\bar{S}^{n+1}}$
    \EndFor
    \State $\bar{S}^n = \dfrac{K - (A_{1}v^{n+1}_{0} + B_{1}v^{n+1}_{1} + C_{1}v^{n+1}_{2})}{D^{n+1}_{1} + (1+\Delta{x})}$
    \State $v^{n}_{0} = K - \bar{S}^{n}$
    \State $v^{n}_{1} = K - (1+\Delta{x})\bar{S}^n$
    \State $v^{n}_{M+1} = 0$
    \For{$i = 2, \dots, M$}
      \State $v^{n}_{i} = A_i v^{n+1}_{i-1} + B_{i}v^{n+1}_{i} + C_{i}v^{n+1}_{i+1} + D^{n+1}_{i}\bar{S}^n$
    \EndFor
  \EndFor
\end{algorithmic}
\end{algorithm}
\newpage
\subsection{Implicit scheme}
Analogously to the previous section, implicit methods approximate the temporal partial derivative using backward difference and the spatial partial derivative using a central difference at time $t_n$ and position $x_i$. Since the PDE in \eqref{eq:blackscholes:frontfixingmethod:inversetransform:american_options_bs_pde} is written backward in time, we use a forward difference instead.
\begin{figure}[H]
  \centering
  \includegraphics[scale=.8]{chapters/chapter3/ImplicitStencil.pdf}
  \caption{Stencil diagram of the implicit scheme.}
  \label{fig:finitedifferencesschemes:implicit_stencil}
\end{figure}
Therefore, the central difference for the first and second order spatial partial derivative at time $t_n$ is 
\begin{align}
  \label{eq:finitedifferencesschemes:implicit:spatial_first_order_central_finite_difference}
  \dfrac{v^{n}_{i+1} - v^{n}_{i-1}}{2 \Delta{x}} =& \dfrac{\partial{v}}{\partial{x}}+ O(\Delta{x}^2) \\
  \label{eq:finitedifferencesschemes:implicit:spatial_second_order_central_finite_difference}
  \dfrac{v^{n}_{i+1} - 2v^{n}_{i} + v^{n}_{i-1}}{\Delta{x}^2} =& \dfrac{\partial^2{v}}{\partial{x^2}}+ O(\Delta{x}^2)
\end{align}
for $i = 1, \dots, M$. Likewise, the forward difference of $v(x, t)$ and $\bar{S}(t)$ at position $x_i$ is  
\begin{align}
  \label{eq:finitedifferencesschemes:implicit:temporal_backward_finite_difference}
  \dfrac{v^{n+1}_{i} - v^{n}_{i}}{\Delta{t}} &= \dfrac{\partial{v}}{\partial{t}}+ O(\Delta{t}) \\
  \label{eq:finitedifferencesschemes:implicit:front_temporal_backward_finite_difference}
  \dfrac{\bar{S}^{n+1}-\bar{S}^{n}}{\Delta t} &= \bar{S}'(t) + O(\Delta{t}) \qquad & \text{ }
\end{align}
for $n = N,\dots,0$. Hence, combining \eqref{eq:finitedifferencesschemes:implicit:spatial_first_order_central_finite_difference}, \eqref{eq:finitedifferencesschemes:implicit:spatial_second_order_central_finite_difference}, \eqref{eq:finitedifferencesschemes:implicit:temporal_backward_finite_difference} and \eqref{eq:finitedifferencesschemes:implicit:front_temporal_backward_finite_difference},we obtain the implicit approximation of the PDE \eqref{eq:blackscholes:frontfixingmethod:inversetransform:american_options_bs_pde} as
\begin{equation*}
  \begin{split}
    \dfrac{v^{n+1}_{i} - v^{n}_{i}}{\Delta{t}} & + \dfrac{1}{2}\sigma^2 x_i^2 \dfrac{v^{n}_{i-1} - 2v^{n}_{i} + v^{n}_{i+1}}{(\Delta{x})^2} \\ 
     & + x_i\bigg( (r-\delta) - \dfrac{1}{\bar{S}^{n}}\dfrac{\bar{S}^{n+1} - \bar{S}^{n}}{\Delta{t}} \bigg)\dfrac{v^{n}_{i+1} - v^{n}_{i-1}}{2\Delta{x}} - rv^{n}_{i} = 0
  \end{split}
\end{equation*}
for $i=1,\dots,M$ and $n=N,\dots,0$. Similar to the explicit method, the approximation error is second order in space and first order in time. Again, to make the implicit  approximation more manageable, we introduce the following terms  
\begin{align}
  \alpha^{n}_{i} &:= -\dfrac{\lambda}{2}\sigma^2x^{2}_{i} + \dfrac{\lambda\Delta{x}}{2}x_{i}\bigg(r-\delta+\dfrac{\bar{S}^{n+1}-\bar{S}^n}{\Delta{t}\bar{S}^{n}}\bigg) \\
  \beta^{n}_{i} &:= 1 + \lambda\sigma^2x^{2}_{i} + r\Delta{t} \\
  \gamma^{n}_{i} &:= -\dfrac{\lambda}{2}\sigma^2x^{2}_{i} + \dfrac{\lambda\Delta{x}}{2}x_{i}\bigg(r-\delta+\dfrac{\bar{S}^{n+1}-\bar{S}^n}{\Delta{t}\bar{S}^{n}}\bigg)
\end{align}
and rearrange the PDE as
\begin{equation}
  \label{eq:finitedifferencesschemes:implicit:implicit_scheme_simplified}
  \alpha^{n}_{i}v^{n}_{i-1} + \beta^{n}_{i}v^{n}_{i} + \gamma^{n}_{i}v^{n}_{i+1} = v^{n+1}_{i}
\end{equation}
The boundary and terminal conditions are given by \eqref{eq:finitedifferencesschemes:explicit:boundary_conditions} and \eqref{eq:finitedifferencesschemes:explicit:terminal_conditions}. Likewise, the approximation of $v^{n}_{M}$ or $v^{n}_{1}$ for put and call, respectively, is given by \eqref{eq:finitedifferencesschemes:explicit:contact_point_approximation_2}. Similar to the explicit method, the approximation $v^{n}_{M}$ and $v^{n}_{0}$ given by the contact point condition is first order in space. Hence, the global approximation error of implicit scheme might be degraded to first order in space. Contrary to the explicit method, there is not an explicit expression for $\bar{S^n}$. Now, we formulate the system of equations of the problem \eqref{eq:blackscholes:frontfixingmethod:inversetransform:american_options_bs_pde}
using \eqref{eq:finitedifferencesschemes:explicit:boundary_conditions}, \eqref{eq:finitedifferencesschemes:explicit:terminal_conditions}, \eqref{eq:finitedifferencesschemes:explicit:contact_point_approximation_2} and \eqref{eq:finitedifferencesschemes:implicit:implicit_scheme_simplified}
\begin{subequations}
  \label{eq:finitedifferencesschemes:implicit:nielsen_system_of_equation}
  \begin{align}
    \text{\textbf{Call:}} \quad& \begin{cases}
      \alpha^{n}_{i}v^{n}_{i-1} + \beta^{n}_{i}v^{n}_{i} + \gamma^{n}_{i}v^{n}_{i+1} = v^{n+1}_{i} & \text{for $i = 1, \dots, M-1$ and $n = N,\dots,0$} \\
      v^{N+1}_i = 0 & \text{for $i = 0, \dots, M+1$}\\
      \bar{S}^{N+1} = K \\
      v^{n}_0 = 0 & \text{for $n = N, \dots 0$}\\
      v^{n}_{M} = (1-\Delta{x})\bar{S}^n - K & \text{for $n = N, \dots 0$}\\
      v^{n}_{M+1} = \bar{S}^n - K  & \text{for $n = N, \dots 0$}
    \end{cases}\\
    \text{\textbf{Put:}} \quad& \begin{cases}
      \alpha^{n}_{i}v^{n}_{i-1} + \beta^{n}_{i}v^{n}_{i} + \gamma^{n}_{i}v^{n}_{i+1} = v^{n+1}_{i} & \text{for $i = 2, \dots, M$ and $n = N,\dots,0$} \\
      v^{N+1}_i = 0 & \text{for $i = 0, \dots, M+1$} \\
      \bar{S}^{N+1} = K \\
      v^{n}_{0} = K - \bar{S}^{n} & \text{for $n = N, \dots 0$}\\
      v^{n}_{1} =  K - (1+\Delta{x})\bar{S}^{n} & \text{for $n = N, \dots 0$}\\
      v^{n}_{M} = 0 & \text{for $n = N, \dots 0$}
    \end{cases}
  \end{align}
\end{subequations}
Since there is not an explicit formula for $v^{n}_{i}$ and $\bar{S}^n$, we will 
have to solve a non-linear system of equation. Let's define the vector $\mathbf{v}^n \in \mathbb{R}^{M-1}$ 
\begin{subequations}
  \begin{align}
    \text{\textbf{Call:}} \qquad \mathbf{v}^{n} :=& \begin{bmatrix}
      v^{n}_{1}, & v^{n}_{2}, & \cdots, & v^{n}_{M-1}
    \end{bmatrix}^{\text{T}}\\
    \text{\textbf{Put:}} \qquad \mathbf{v}^{n} :=& \begin{bmatrix}
      v^{n}_{2}, & v^{n}_{3}, & \cdots, & v^{n}_{M}
    \end{bmatrix}^{\text{T}}
  \end{align}    
\end{subequations}
the matrix $\Lambda^{n} \in \mathbb{R}^{M-1,M-2}$ 
\begin{subequations}
  \begin{align}
    \text{\textbf{Call:}} \qquad \Lambda^{n} =& \begin{bmatrix}
      \beta^{n}_{1} & \gamma^{n}_1 \\
      \alpha^{n}_{2} & \beta^{n}_{2} & \gamma^{n}_{2} \\
      & \ddots & \ddots & \ddots  \\
      & & \ddots & \ddots & \ddots  \\
      & & & \alpha^{n}_{M-2} & \beta^{n}_{M-2} & \gamma^{n}_{M-2} \\
      & & & & \alpha^{n}_{M-1} & \beta^{n}_{M-1} \\
      & & & & & \alpha^{n}_{M} \\
    \end{bmatrix}\\
    \text{\textbf{Put:}} \qquad  \Lambda^{n} :=& \begin{bmatrix}
      \gamma^{n}_1 \\
      \beta^{n}_2 & \gamma^{n}_2 \\
      \alpha^{n}_3 & \beta^{n}_3 & \gamma^{n}_3 \\
      & \ddots & \ddots & \ddots \\
      & & \ddots & \ddots & \ddots \\
      & & & \alpha^{n}_{M-1} & \beta^{n}_{M-1} & \gamma^{n}_{M-1} \\
      & & & & \alpha^{n}_{M} & \beta^{n}_{M} \\
    \end{bmatrix}
  \end{align}
\end{subequations}
and the vector $\mathbf{f}^{n} \in \mathbb{R}^{M-1}$
\begin{subequations}
  \begin{align}
    \text{\textbf{Call:}} \qquad \mathbf{f}^n :=& \begin{bmatrix}
      v^{n+1}_{1} \\
      \vdots \\
      v^{n+1}_{M-1} - \gamma^{n}_{M-1}[(1-\Delta{x})\bar{S}^{n} - K] \\
      v^{n+1}_{M} - \gamma^{n}_{M}(\bar{S}^n - K) - \beta^{n}_{M}[(1-\Delta{x})\bar{S}^{n} - K]
    \end{bmatrix}\\
    \text{\textbf{Put:}} \qquad \mathbf{f}^n :=& \begin{bmatrix}
      v^{n+1}_{1} - \alpha^{n}_{1}(K - \bar{S}^{n}) - \beta^{n}_{1}[K - (1+\Delta{x})\bar{S}^{n}] \\
      v^{n+1}_{2} - \beta^{n}_2[K - (1+\Delta{x})\bar{S}^{n}] \\
      v^{n+1}_{3} \\
      \vdots \\
      v^{n+1}_{M-1}
    \end{bmatrix}
  \end{align}
\end{subequations}
Thus, the non-linear system of equations that we need to solve is
\begin{equation}
  F(\mathbf{v}^{n}, \bar{S}^{n}) = \Lambda^{n}\mathbf{v}^{n} - \mathbf{f}^n = 0
\end{equation}

By computing the Jacobian of the system, we con solve the non-linear system 
using the newton's method

\begin{equation}
  \mathbf{y}_{k+1} = \mathbf{y}_{k} - J^{-1}(\mathbf{y}_{k})F(\mathbf{y}_{k})
\end{equation}

where $y_k$ is some approximation of the solution
\begin{equation}
  \mathbf{y} = \begin{bmatrix}
    \mathbf{v}^{n} | \bar{S}^{n}
  \end{bmatrix}^{\text{T}}
\end{equation}

\subsection{Numerical results}
Generally, Datasets for American options are hard to get and often require paying substantial amount of money. Therefore, to validate our implementation, we mainly relied on the data available in Company, 
et al. \cite*{company_egorova_jodar_2014}, Nielsen, et al. \cite*{nielsen_2001}, Seydel \cite*{seydel_2009}, and Wilmott, et al. \cite*{wilmott_howison_dewynne_1995}. Moreover, we used the approximations produced by the binomial model introduced by Cox et al. \cite{cox_1979} as benchmark for assessing the consistency of our method. We chose the binomial model because it uses a completely different approach to price options than the one considered in our work, is widely used in the industry, and is simple to implement. First, 
we want to assess the correct functionality of our implementation. With that objective, we define the set of parameters taken from \cite{nielsen_2001}

\begin{equation}
  \label{eq:numericaresults:parameters_set_1}
  K = 1, \quad T = 1, \quad r=0.2, \quad \sigma=0.2, \quad \delta = 0 
\end{equation}

Moreover, Nielsen \cite{nielsen_2001} suggests that the optimal exercise boundary $\bar{S}(t) \approx 0.86$ for parameters \eqref{eq:numericaresults:parameters_set_1}. Then, we validated that the binomial model produces a similar approximation for $\bar{S}(t)$. Finally, we priced a put option using our implementation of the explicit \eqref{eq:finitedifferencesschemes:explicit:nielsen_system_of_equation} and implicit \eqref{eq:finitedifferencesschemes:implicit:nielsen_system_of_equation} method for the Nielsen transformation, and the explicit method for the Company transformation \eqref{alg:appendix:companytransformation:explicits:put_explicit_method_algorithm}. In tables \eqref{tab:rsme_explicit_nielsen_transformation} and \eqref{tab:rsme_explicit_company_transformation}, you can find the RMSE error produced by the explicit method for Nielsen and Company transformation.

% Please add the following required packages to your document preamble:
% \usepackage{booktabs}
\begin{table}[H]
    \centering
    \begin{tabular}{@{}ccccccc@{}}
    \toprule
    \textbf{Asset price} & \textbf{BOPM} & 0.125    & 0.0625   & 0.03125  & 0.015625 & 0.0078125 \\ \midrule
    0.8                  & 0.200000      & 0.200000 & 0.200000 & 0.200000 & 0.200000 & 0.200000  \\
    1.0                  & 0.048167      & 0.046048 & 0.047748 & 0.048176 & 0.048160 & 0.048155  \\
    1.2                  & 0.008666      & 0.008661 & 0.008638 & 0.008705 & 0.008674 & 0.008662  \\
    1.4                  & 0.001285      & 0.001467 & 0.001369 & 0.001313 & 0.001292 & 0.001286  \\
    1.6                  & 0.000167      & 0.000237 & 0.000196 & 0.000176 & 0.000170 & 0.000168  \\
    1.8                  & 0.000020      & 0.000038 & 0.000027 & 0.000022 & 0.000021 & 0.000020  \\
    2.0                  & 0.000002      & 0.000006 & 0.000004 & 0.000003 & 0.000002 & 0.000002  \\
                         & \textbf{RMSE} & 0.000080 & 0.000016 & 0.000002 & 0.00000  & 0.000000  \\ \bottomrule
    \end{tabular}
    \caption{\label{tab:rsme_explicit_nielsen_transformation}RSME error produced by the explicit scheme for the Nielsen transformation for $\Delta{t}=1/8,1/16,\dots,1/128$ and $\Delta{t}=\Delta{x}^2/2$.}
\end{table}


% Please add the following required packages to your document preamble:
% \usepackage{booktabs}
\begin{table}[H]
    \centering
    \begin{tabular}{@{}ccccccc@{}}
    \toprule
    \textbf{Asset price} & \textbf{BOPM} & 0.125    & 0.0625   & 0.03125  & 0.015625 & 0.0078125 \\ \midrule
    0.8                  & 0.200000      & 0.200000 & 0.200000 & 0.200000 & 0.200000 & 0.200000  \\
    1.0                  & 0.048167      & 0.049286 & 0.049274 & 0.048465 & 0.048256 & 0.048174  \\
    1.2                  & 0.008666      & 0.010736 & 0.009108 & 0.008829 & 0.008686 & 0.008667  \\
    1.4                  & 0.001285      & 0.001950 & 0.001501 & 0.001349 & 0.001295 & 0.001287  \\
    1.6                  & 0.000167      & 0.000354 & 0.000224 & 0.000185 & 0.000172 & 0.000168  \\
    1.8                  & 0.000020      & 0.000076 & 0.000035 & 0.000024 & 0.000021 & 0.000020  \\
    2.0                  & 0.000002      & 0.000017 & 0.000006 & 0.000003 & 0.000003 & 0.000002  \\
    & \textbf{RSME} & 0.000092 & 0.000045 & 0.000013 & 0.000003 & 0.00000   \\ \bottomrule
    \end{tabular}
    \caption{\label{tab:rsme_explicit_company_transformation}RSME error produced by the explicit scheme for Company transformation for $\Delta{t}=1/8,1/16,\dots,1/128$ and $\Delta{t}=\Delta{x}^2/2$.}
\end{table}

Moreover, figure \eqref{fig:finitedifferencesschemes:numericaresults:test_case_2_explicit} show the $V(S, 0)$ curve obtained by the Nielsen and Company explicit method. In each plot, we have listed the optimal exercise boundary $\bar{S}(0)$. Also, we have listed the correspondent value curve produced using the binary option pricing model using $2^500$ nodes. As you see, Nielsen and Company approximated optimal exercise boundary within 2 decimal places. 

\begin{figure}[H]
  \centering
  \begin{subfigure}{0.435\textwidth}
    \label{fig:finitedifferencesschemes:numericaresults:test_case_2_bopm}
    \centering
    \includegraphics[width=\textwidth]{chapters/chapter3/TestCase2BOPM.pdf}
    \caption{Binary model of $2^{500}$ nodes.}
  \vspace{0.5cm}
  \end{subfigure}
  \begin{subfigure}{0.4\textwidth}
    \label{fig:finitedifferencesschemes:numericaresults:test_case_2_explicit_nielsen}
    \centering
    \includegraphics[width=\textwidth]{chapters/chapter3/TestCase2NielsenExplicit.pdf}
    \caption{Nielsen transformation with $\Delta{x}=\expnumber{1}{-3}$ and $\Delta{t}=0.5\times\expnumber{1}{-6}$}
  \end{subfigure}
  \begin{subfigure}{0.435\textwidth}
    \label{fig:finitedifferencesschemes:numericaresults:test_case_2_explicit_company}
    \centering
    \includegraphics[width=\textwidth]{chapters/chapter3/TestCase2CompanyExplicit.pdf}
    \caption{Company transformation $\Delta{x}=\expnumber{1}{-3}$ and $\Delta{t}=0.5\times\expnumber{1}{-6}$}
  \end{subfigure}
  \caption{American put option value $V(S, 0)$ curve.}
  \label{fig:finitedifferencesschemes:numericaresults:test_case_2_explicit}
\end{figure}

Additionally, table \eqref{tab:rsme_implicit_nielsen_transformation} present the RSME error produced by the implicit method for the Nielsen transformation and figure \eqref{fig:finitedifferencesschemes:numericaresults:test_case_2_implicit} show the value curve obtained by using the implicit method for the Nielsen approximation.

\begin{figure}[H]
  \centering
  \begin{subfigure}{0.435\textwidth}
    \label{fig:finitedifferencesschemes:numericaresults:test_case_2_implicit_bopm}
    \centering
    \includegraphics[width=\textwidth]{chapters/chapter3/TestCase2BOPM.pdf}
    \caption{Binary model of $2^{500}$ nodes.}
    \vspace{0.5cm}
  \end{subfigure}
  \begin{subfigure}{0.4\textwidth}
    \label{fig:finitedifferencesschemes:numericaresults:test_case_2_implicit_nielsen}
    \centering
    \includegraphics[width=\textwidth]{chapters/chapter3/TestCase2NielsenExplicit.pdf}
    \caption{Implicit method for Nielsen transformation with $\Delta{x}=\Delta{t}=\expnumber{1}{-3}$.}
  \end{subfigure}
  \caption{American put option value $V(S, 0)$ curve.}
  \label{fig:finitedifferencesschemes:numericaresults:test_case_2_implicit}
\end{figure}

% Please add the following required packages to your document preamble:
% \usepackage{booktabs}
\begin{table}[H]
  \centering
  \begin{tabular}{@{}ccccccc@{}}
  \toprule
  \textbf{Asset price} & \textbf{BOPM}        & 0.125      & 0.0625     & 0.03125    & 0.015625   & 0.0078125  \\ \midrule
  0.8                  & 0.200000             & 0.224810   & 0.227045   & 0.200000   & 0.200000   & 0.200000   \\
  1.0                  & 0.048167             & 0.156141   & 0.159308   & 0.080886   & 0.053942   & 0.048944   \\
  1.2                  & 0.008666             & 0.114815   & 0.117140   & 0.041319   & 0.013783   & 0.009290   \\
  1.4                  & 0.001285             & 0.080638   & 0.081485   & 0.023604   & 0.003694   & 0.001519   \\
  1.6                  & 0.000167             & 0.047914   & 0.048026   & 0.014269   & 0.001085   & 0.000229   \\
  1.8                  & 0.000020             & 0.019130   & 0.020665   & 0.008591   & 0.000353   & 0.000033   \\
  2.0                  & 0.000002             & 0.000022   & 0.003408   & 0.004695   & 0.000125   & 0.000005   \\
                       & \textbf{Asset price} & 0.06812077 & 0.06970396 & 0.02045601 & 0.00307761 & 0.00038755 \\ \bottomrule
  \end{tabular}
  \caption{\label{tab:rsme_implicit_nielsen_transformation}RSME error produced by the implicit scheme for the Nielsen transformation for $\Delta{x}=\Delta{t}=1/8,1/16,\dots,1/128$.}
\end{table}

In the figures  \eqref{fig:finitedifferencesschemes:numericaresults:test_case_2_explicit} and \eqref{fig:finitedifferencesschemes:numericaresults:test_case_2_implicit}, we observe that the approximation follows the geometry of American options. Specifically, within the continuation region $S>\bar{S}(0)$, the value is larger than the payoff function and as S gets larger, the value goes to zero. In addition, the value curve is exactly the payoff. Also notice that the RSME of the implicit method for Nielsen transformation is larger than the explicit method for both Company and Nielsen transformation. The explanation we have for this is that for the implicit method we have to solve a non-linear system of equation. Therefore, the error produced by the implicit version of Nielsen will be affected by the error produced by non-linear solver. 

We realized numerous testings to validate the correct functionality of our implementation. As Merton \cite{merton_1973} showed, pricing American call options without dividends is equivalent to solve the price of European options. Therefore, the most natural test to conduct is to price American call options without dividends ($\delta=0$) using our implementation and compare the results to the one obtain using the Black-Scholes formula for  \cite{merton_1973}. 

For the following experiments, we use the following set of parameters.

Moreover, \eqref{eq:numericaresults:parameters_set_1}, we tested each scheme for the case of put options. For the setup of the experiments, we priced a put option using the binomial model \cite{cox_1979} with parameters \eqref{eq:numericaresults:parameters_set_1}. Then, we proceed to compute the RSME error 

where $V^{\text{BM}}_i$ is the price produced by the binomial model, and $\hat{V}_i$ is the price produced by our numerical method. In Tables \eqref{tab:rsme_explicit_nielsen_transformation} and \eqref{tab:rsme_explicit_company_transformation} contain the results obtained for Nielsen and Company transformation, respectively, for $\Delta{x}=1/8, 1/16,\dots,1/128$ and $\Delta{t}=\Delta{x}^2$

% Please add the following required packages to your document preamble:
% \usepackage{booktabs}
\begin{table}[H]
    \begin{tabular}{@{}ccccccc@{}}
        \toprule
        \textbf{Asset price} & \textbf{BOPM} & 0.125    & 0.0625   & 0.03125  & 0.015625 & 0.0078125 \\ \midrule
        0.8                  & 0.200000      & 0.200000 & 0.200000 & 0.200000 & 0.200000 & 0.200000  \\
        1.0                  & 0.048155      & 0.046048 & 0.047748 & 0.048176 & 0.048160 & 0.048155  \\
        1.2                  & 0.008662      & 0.008661 & 0.008638 & 0.008705 & 0.008674 & 0.008662  \\
        1.4                  & 0.001286      & 0.001467 & 0.001369 & 0.001313 & 0.001292 & 0.001286  \\
        1.6                  & 0.000168      & 0.000237 & 0.000196 & 0.000176 & 0.000170 & 0.000168  \\
        1.8                  & 0.000020      & 0.000038 & 0.000027 & 0.000022 & 0.000021 & 0.000020  \\
        2.0                  & 0.000002      & 0.000006 & 0.000004 & 0.000003 & 0.000002 & 0.000002  \\
                             & \textbf{RMSE} & 0.000081 & 0.000017 & 0.000002 & 0.000000 & 0.000000  \\ \bottomrule
    \end{tabular}
\end{table}

% Please add the following required packages to your document preamble:
% \usepackage{booktabs}
\begin{table}[H]
    \begin{tabular}{@{}ccccccc@{}}
        \toprule
        \textbf{Asset price} & \textbf{BOPM} & 0.125    & 0.0625   & 0.03125  & 0.015625 & 0.0078125 \\ \midrule
        0.8                  & 0.200000      & 0.224810 & 0.227045 & 0.200000 & 0.200000 & 0.200000  \\
        1.0                  & 0.048180      & 0.156141 & 0.159308 & 0.080886 & 0.053942 & 0.048944  \\
        1.2                  & 0.008661      & 0.114815 & 0.117140 & 0.041319 & 0.013783 & 0.009290  \\
        1.4                  & 0.001283      & 0.080638 & 0.081485 & 0.023604 & 0.003694 & 0.001519  \\
        1.6                  & 0.000167      & 0.047914 & 0.048026 & 0.014269 & 0.001085 & 0.000229  \\
        1.8                  & 0.000020      & 0.019130 & 0.020665 & 0.008591 & 0.000353 & 0.000033  \\
        2.0                  & 0.000002      & 0.000022 & 0.003408 & 0.004695 & 0.000125 & 0.000005  \\
                             & \textbf{RMSE} & 0.068119 & 0.069703 & 0.020455 & 0.003076 & 0.000385  \\ \bottomrule
    \end{tabular}
\end{table}

% Please add the following required packages to your document preamble:
% \usepackage{booktabs}
\begin{table}[H]
    \begin{tabular}{@{}ccccccc@{}}
        \toprule
        \textbf{Asset price} & \textbf{BOPM} & 0.125    & 0.0625   & 0.03125  & 0.015625 & 0.0078125 \\ \midrule
        0.8                  & 0.200000      & 0.200000 & 0.200000 & 0.200000 & 0.200000 & 0.200000  \\
        1.0                  & 0.048180      & 0.054188 & 0.053469 & 0.052440 & 0.052340 & 0.052259  \\
        1.2                  & 0.008661      & 0.012658 & 0.010936 & 0.010331 & 0.010294 & 0.010271  \\
        1.4                  & 0.001283      & 0.002423 & 0.001855 & 0.001694 & 0.001650 & 0.001640  \\
        1.6                  & 0.000167      & 0.000462 & 0.000292 & 0.000250 & 0.000234 & 0.000229  \\
        1.8                  & 0.000020      & 0.000107 & 0.000048 & 0.000034 & 0.000030 & 0.000029  \\
        2.0                  & 0.000002      & 0.000028 & 0.000009 & 0.000005 & 0.000004 & 0.000004  \\
                             & \textbf{RMSE} & 0.002763 & 0.002187 & 0.001737 & 0.001695 & 0.001663  \\ \bottomrule
    \end{tabular}
\end{table}

% Please add the following required packages to your document preamble:
% \usepackage{booktabs}
\begin{table}[H]
    \begin{tabular}{@{}lllllll@{}}
        \toprule
        \textbf{Asset Price} & \textbf{BOPM} & 0.125    & 0.0625   & 0.03125  & 0.015625 & 0.0078125 \\ \midrule
        0.8                  & 0.170000      & 0.170000 & 0.170000 & 0.170000 & 0.170000 & 0.170000  \\
        1.0                  & 0.037785      & 0.038365 & 0.038260 & 0.037619 & 0.037754 & 0.037773  \\
        1.2                  & 0.006576      & 0.007278 & 0.006645 & 0.006602 & 0.006573 & 0.006573  \\
        1.4                  & 0.000953      & 0.001132 & 0.001013 & 0.000969 & 0.000953 & 0.000952  \\
        1.6                  & 0.000122      & 0.000185 & 0.000142 & 0.000128 & 0.000124 & 0.000123  \\
        1.8                  & 0.000015      & 0.000038 & 0.000021 & 0.000016 & 0.000015 & 0.000015  \\
        2.0                  & 0.000002      & 0.000008 & 0.000003 & 0.000002 & 0.000002 & 0.000002  \\
                             & \textbf{RMSE} & 0.000035 & 0.000018 & 0.000006 & 0.000001 & 0.000000  \\ \bottomrule
    \end{tabular}
\end{table}
% \section{Numerical results}


For the setup of the experiments, we utilize the binomial option pricing model \cite{cox_1979}

We realized numerous testings to validate the correct functionality of our implementation. As Merton \cite{merton_1973} showed, pricing American call options without dividends is equivalent to solve the price of European options. Therefore, the most natural test to conduct is to price American call options without dividends ($\delta=0$) using our implementation and compare the results to the one obtain using the Black-Scholes formula for  \cite{merton_1973}. 

For the following experiments, we use the following set of parameters.
\begin{equation}
    \label{eq:numericaresults:parameters_set_1}
    K = 1, \quad T = 1, \quad r=0.2, \quad \sigma=0.2, \quad \delta = 0 
\end{equation}

Moreover, using the same set of parameters as in \eqref{eq:numericaresults:parameters_set_1}, we tested each scheme for the case of put options. For the setup of the experiments, we priced a put option using the binomial model \cite{cox_1979} with parameters \eqref{eq:numericaresults:parameters_set_1}. Then, we proceed to compute the RSME error 

\begin{equation}
    \label{eq:numericaresults:rsme}
    \text{RSME} := \bigg(\sum_{i}{(V_{\text{BOPM}} - \hat{V}_i)^2}\bigg)^{1/2}
\end{equation}

where $V^{\text{BM}}_i$ is the price produced by the binomial model, and $\hat{V}_i$ is the price produced by our numerical method. In Tables \ref{tab:rsme_explicit_nielsen_transformation} and \ref{tab:rsme_explicit_company_transformation} contain the results obtained for Nielsen and Company transformation, respectively, for $\Delta{x}=1/8, 1/16,\dots,1/128$ and $\Delta{t}=\Delta{x}^2$

% Please add the following required packages to your document preamble:
% \usepackage{booktabs}
\begin{table}[H]
    \centering
    \begin{tabular}{@{}ccccccc@{}}
    \toprule
    \textbf{Asset price} & \textbf{BOPM} & 0.125    & 0.0625   & 0.03125  & 0.015625 & 0.0078125 \\ \midrule
    0.8                  & 0.200000      & 0.200000 & 0.200000 & 0.200000 & 0.200000 & 0.200000  \\
    1.0                  & 0.048167      & 0.046048 & 0.047748 & 0.048176 & 0.048160 & 0.048155  \\
    1.2                  & 0.008666      & 0.008661 & 0.008638 & 0.008705 & 0.008674 & 0.008662  \\
    1.4                  & 0.001285      & 0.001467 & 0.001369 & 0.001313 & 0.001292 & 0.001286  \\
    1.6                  & 0.000167      & 0.000237 & 0.000196 & 0.000176 & 0.000170 & 0.000168  \\
    1.8                  & 0.000020      & 0.000038 & 0.000027 & 0.000022 & 0.000021 & 0.000020  \\
    2.0                  & 0.000002      & 0.000006 & 0.000004 & 0.000003 & 0.000002 & 0.000002  \\
                         & \textbf{RMSE} & 0.000080 & 0.000016 & 0.000002 & 0.00000  & 0.000000  \\ \bottomrule
    \end{tabular}
    \caption{\label{tab:rsme_explicit_nielsen_transformation}RSME error produced by the explicit scheme for the Nielsen transformation for $\Delta{t}=1/8,1/16,\dots,1/128$ and $\Delta{t}=\Delta{x}^2/2$.}
\end{table}

% Please add the following required packages to your document preamble:
% \usepackage{booktabs}
\begin{table}[H]
    \centering
    \begin{tabular}{@{}ccccccc@{}}
    \toprule
    \textbf{Asset price} & \textbf{BOPM} & 0.125    & 0.0625   & 0.03125  & 0.015625 & 0.0078125 \\ \midrule
    0.8                  & 0.200000      & 0.200000 & 0.200000 & 0.200000 & 0.200000 & 0.200000  \\
    1.0                  & 0.048167      & 0.049286 & 0.049274 & 0.048465 & 0.048256 & 0.048174  \\
    1.2                  & 0.008666      & 0.010736 & 0.009108 & 0.008829 & 0.008686 & 0.008667  \\
    1.4                  & 0.001285      & 0.001950 & 0.001501 & 0.001349 & 0.001295 & 0.001287  \\
    1.6                  & 0.000167      & 0.000354 & 0.000224 & 0.000185 & 0.000172 & 0.000168  \\
    1.8                  & 0.000020      & 0.000076 & 0.000035 & 0.000024 & 0.000021 & 0.000020  \\
    2.0                  & 0.000002      & 0.000017 & 0.000006 & 0.000003 & 0.000003 & 0.000002  \\
    & \textbf{RSME} & 0.000092 & 0.000045 & 0.000013 & 0.000003 & 0.00000   \\ \bottomrule
    \end{tabular}
    \caption{\label{tab:rsme_explicit_company_transformation}RSME error produced by the explicit scheme for Company transformation for $\Delta{t}=1/8,1/16,\dots,1/128$ and $\Delta{t}=\Delta{x}^2/2$.}
\end{table}

% Please add the following required packages to your document preamble:
% \usepackage{booktabs}
\begin{table}[H]
    \centering
    \begin{tabular}{@{}ccccccc@{}}
    \toprule
    \textbf{Asset price} & \textbf{BOPM}        & 0.125      & 0.0625     & 0.03125    & 0.015625   & 0.0078125  \\ \midrule
    0.8                  & 0.200000             & 0.224810   & 0.227045   & 0.200000   & 0.200000   & 0.200000   \\
    1.0                  & 0.048167             & 0.156141   & 0.159308   & 0.080886   & 0.053942   & 0.048944   \\
    1.2                  & 0.008666             & 0.114815   & 0.117140   & 0.041319   & 0.013783   & 0.009290   \\
    1.4                  & 0.001285             & 0.080638   & 0.081485   & 0.023604   & 0.003694   & 0.001519   \\
    1.6                  & 0.000167             & 0.047914   & 0.048026   & 0.014269   & 0.001085   & 0.000229   \\
    1.8                  & 0.000020             & 0.019130   & 0.020665   & 0.008591   & 0.000353   & 0.000033   \\
    2.0                  & 0.000002             & 0.000022   & 0.003408   & 0.004695   & 0.000125   & 0.000005   \\
                         & \textbf{Asset price} & 0.06812077 & 0.06970396 & 0.02045601 & 0.00307761 & 0.00038755 \\ \bottomrule
    \end{tabular}
    \caption{RSME error produced by the implicit scheme for the Nielsen transformation for $\Delta{x}=\Delta{t}=1/8,1/16,\dots,1/128$.}
\end{table}

% Please add the following required packages to your document preamble:
% \usepackage{booktabs}
\begin{table}[H]
    \begin{tabular}{@{}ccccccc@{}}
        \toprule
        \textbf{Asset price} & \textbf{BOPM} & 0.125    & 0.0625   & 0.03125  & 0.015625 & 0.0078125 \\ \midrule
        0.8                  & 0.200000      & 0.200000 & 0.200000 & 0.200000 & 0.200000 & 0.200000  \\
        1.0                  & 0.048155      & 0.046048 & 0.047748 & 0.048176 & 0.048160 & 0.048155  \\
        1.2                  & 0.008662      & 0.008661 & 0.008638 & 0.008705 & 0.008674 & 0.008662  \\
        1.4                  & 0.001286      & 0.001467 & 0.001369 & 0.001313 & 0.001292 & 0.001286  \\
        1.6                  & 0.000168      & 0.000237 & 0.000196 & 0.000176 & 0.000170 & 0.000168  \\
        1.8                  & 0.000020      & 0.000038 & 0.000027 & 0.000022 & 0.000021 & 0.000020  \\
        2.0                  & 0.000002      & 0.000006 & 0.000004 & 0.000003 & 0.000002 & 0.000002  \\
                             & \textbf{RMSE} & 0.000081 & 0.000017 & 0.000002 & 0.000000 & 0.000000  \\ \bottomrule
    \end{tabular}
\end{table}

% Please add the following required packages to your document preamble:
% \usepackage{booktabs}
\begin{table}[H]
    \begin{tabular}{@{}ccccccc@{}}
        \toprule
        \textbf{Asset price} & \textbf{BOPM} & 0.125    & 0.0625   & 0.03125  & 0.015625 & 0.0078125 \\ \midrule
        0.8                  & 0.200000      & 0.224810 & 0.227045 & 0.200000 & 0.200000 & 0.200000  \\
        1.0                  & 0.048180      & 0.156141 & 0.159308 & 0.080886 & 0.053942 & 0.048944  \\
        1.2                  & 0.008661      & 0.114815 & 0.117140 & 0.041319 & 0.013783 & 0.009290  \\
        1.4                  & 0.001283      & 0.080638 & 0.081485 & 0.023604 & 0.003694 & 0.001519  \\
        1.6                  & 0.000167      & 0.047914 & 0.048026 & 0.014269 & 0.001085 & 0.000229  \\
        1.8                  & 0.000020      & 0.019130 & 0.020665 & 0.008591 & 0.000353 & 0.000033  \\
        2.0                  & 0.000002      & 0.000022 & 0.003408 & 0.004695 & 0.000125 & 0.000005  \\
                             & \textbf{RMSE} & 0.068119 & 0.069703 & 0.020455 & 0.003076 & 0.000385  \\ \bottomrule
    \end{tabular}
\end{table}

% Please add the following required packages to your document preamble:
% \usepackage{booktabs}
\begin{table}[H]
    \begin{tabular}{@{}ccccccc@{}}
        \toprule
        \textbf{Asset price} & \textbf{BOPM} & 0.125    & 0.0625   & 0.03125  & 0.015625 & 0.0078125 \\ \midrule
        0.8                  & 0.200000      & 0.200000 & 0.200000 & 0.200000 & 0.200000 & 0.200000  \\
        1.0                  & 0.048180      & 0.054188 & 0.053469 & 0.052440 & 0.052340 & 0.052259  \\
        1.2                  & 0.008661      & 0.012658 & 0.010936 & 0.010331 & 0.010294 & 0.010271  \\
        1.4                  & 0.001283      & 0.002423 & 0.001855 & 0.001694 & 0.001650 & 0.001640  \\
        1.6                  & 0.000167      & 0.000462 & 0.000292 & 0.000250 & 0.000234 & 0.000229  \\
        1.8                  & 0.000020      & 0.000107 & 0.000048 & 0.000034 & 0.000030 & 0.000029  \\
        2.0                  & 0.000002      & 0.000028 & 0.000009 & 0.000005 & 0.000004 & 0.000004  \\
                             & \textbf{RMSE} & 0.002763 & 0.002187 & 0.001737 & 0.001695 & 0.001663  \\ \bottomrule
    \end{tabular}
\end{table}

% Please add the following required packages to your document preamble:
% \usepackage{booktabs}
\begin{table}[H]
    \begin{tabular}{@{}lllllll@{}}
        \toprule
        \textbf{Asset Price} & \textbf{BOPM} & 0.125    & 0.0625   & 0.03125  & 0.015625 & 0.0078125 \\ \midrule
        0.8                  & 0.170000      & 0.170000 & 0.170000 & 0.170000 & 0.170000 & 0.170000  \\
        1.0                  & 0.037785      & 0.038365 & 0.038260 & 0.037619 & 0.037754 & 0.037773  \\
        1.2                  & 0.006576      & 0.007278 & 0.006645 & 0.006602 & 0.006573 & 0.006573  \\
        1.4                  & 0.000953      & 0.001132 & 0.001013 & 0.000969 & 0.000953 & 0.000952  \\
        1.6                  & 0.000122      & 0.000185 & 0.000142 & 0.000128 & 0.000124 & 0.000123  \\
        1.8                  & 0.000015      & 0.000038 & 0.000021 & 0.000016 & 0.000015 & 0.000015  \\
        2.0                  & 0.000002      & 0.000008 & 0.000003 & 0.000002 & 0.000002 & 0.000002  \\
                             & \textbf{RMSE} & 0.000035 & 0.000018 & 0.000006 & 0.000001 & 0.000000  \\ \bottomrule
    \end{tabular}
\end{table}
\section{Linear complementary problem}
\subsection{Overview}
A linear complementary problem (LCP) is an optimization problem in which goal is to solve a system of linear equations subject to a complementary condition. In other words, given the matrix $\mathbf{A}\in\mathbb{R}^{d \times d}$ and vector $\mathbf{r}\in\mathbb{R}^{d}$, we want to find the vectors $\mathbf{v}\in\mathbb{R}^{d}$ and $\mathbf{w}\in\mathbb{R}^{d}$ such as
\begin{align*}
  \mathbf{A}\mathbf{v} - \mathbf{w} = \mathbf{b}
\end{align*}
subject to the complementary conditions 
\begin{align*}
  v_i &\ge 0 \qquad \text{for $i = 1,\dots,d$} \\
  w_i &\ge 0 \qquad \text{for $i = 1,\dots,d$} \\
  \mathbf{v}^{T}\mathbf{w} &= 0
\end{align*}
Note that the complementary conditions require that either $v_i=0$ or $w_i=0$ for each component. As you will see later, the pricing problem for American options can be reformulated as a linear complementary problem. But first, we need to reconsider the results obtained from applying the Black-Scholes to pricing problem in section \ref{sec:blackscholes}. Recall that the value curve of American options $V(S,t)$ is bounded from below by the payoff function 
\begin{align*}
  V(S, t) - H(S, t) \ge 0 \qquad \text{for all $(S,t)$}
\end{align*}
Moreover, the value function can be divided in two complementary regions: the exercise region \eqref{eq:blackscholes:preliminaries:exercise_region} where  
\begin{align}
  \label{eq:lcp:overview:value_in_continuation_region}
  &V(S, t) - H(S, t) > 0 \qquad \text{for $(S,t) \in \mathcal{C}$} \\ 
\end{align}
and continuation region \eqref{eq:blackscholes:preliminaries:continuation_region} where 
\begin{align}
  \label{eq:lcp:overview:value_in_exercise_region}
  &V(S, t) - H(S, t) = 0 \qquad \text{for $(S,t) \in \mathcal{S}$}
\end{align}
and the value function $V(S,t)$ is the solution to the Black-Scholes PDE  within the continuation region
\begin{align*}
    \frac{\partial{V}}{\partial{t}} + \mathcal{L}_{\text{BS}}(V) = 0 \qquad \text{for $(S,t) \in \mathcal{C}$}
\end{align*}
where $\mathcal{L}_{\text{BS}}(V)$ is the linear parabolic operator \eqref{eq:blackscholes:preliminaries:linear_parabolic_operator} applied to the function $V(S,t)$. Finally, by plugin $V(S,t)=H(S,t)$ into the Black-Scholes PDE, we obtain the bound for PDE in the exercise region
\begin{align}
    \label{eq:lcp:overview:blackscholes_in_exercise_region}
    \frac{\partial{V}}{\partial{t}} + \mathcal{L}_{\text{BS}} < 0 \qquad \text{for $(S,t) \in \mathcal{S}$}
\end{align}
By grouping \eqref{eq:lcp:overview:value_in_continuation_region}, \eqref{eq:lcp:overview:value_in_exercise_region}, \eqref{eq:lcp:overview:blackscholes_in_exercise_region}, we reformulate the problem of pricing American options as system of variational inequalities
\begin{align}
  \begin{cases}
    \big[\frac{\partial V}{\partial \tau} - \mathcal{L}_{\text{BS}}(V)\big] \cdot [V(S,T-\tau) - H(S,T-\tau)] = 0 & \text{for all $(S,t)$} \\
    V(S, T-\tau) - H(S, T-\tau) \ge 0 & \text{for $(S, t) \ $}\\
    \frac{\partial V}{\partial \tau} - \mathcal{L}_{\text{BS}}(V) \ge 0 &  \text{for all $(S, t)$}\\
    V(S, T) = H(S, T) \\  
  \end{cases}
  \label{eq:lcp:overview:variational_inequalities}
\end{align}
The benefit of the variational inequalities is that there is no explicit dependence on the optimal exercise boundary defined in \eqref{eq:blackscholes:preliminaries:optimal_exercise_boundary}. Also, note that we rewrote the Black-Scholes PDE forward in time by introducing the transformation $\tau = T - t$ deliberately so that the variational inequalities' formulation looks similar in structure to LCP problem. But before elaborating more about the relationship between the variational inequalities and LCP problems, we introduce the transformations
\begin{align}
\label{eq:lcp:overview:heat_diffusion_domain_transformation}
S = Ke^x, \quad t = T - \dfrac{2\tau}{\sigma^2},\quad h(x,\tau) := e^{(\alpha x + \beta \tau)}\dfrac{H(S,t)}{K}, \quad v(x, \tau) := V(S, t) =: Ke^{-(\alpha x + \beta \tau)}y(x, \tau)
\end{align}
where
\begin{align}
  \label{eq:lcp:overview:heat_diffusion_domain_transformation_2}
  q &:= \dfrac{2r}{\sigma^2}, \quad q_{\delta} := \dfrac{2(r-\delta)}{\sigma^2}, \quad \alpha := \dfrac{1}{2}(q_{\delta} - 1) \quad \beta := \dfrac{1}{4}(q_{\delta} - 1)^2 + q 
\end{align}
so that the Black-Scholes PDE \eqref{eq:chapter2:european_option_pde_with_dividens} to transform to the heat diffusion equation
\begin{align}
  \label{eq:lcp:overview:heat_diffusion_equation_PDE}
  \dfrac{\partial{y}}{\partial{\tau}} - \dfrac{\partial^2{y}}{\partial{x^2}} = 0
\end{align}
with boundary conditions, its solution $y(x,\tau)$ lies within the continuous region 
\begin{align}
  \label{eq:lcp:overview:heat_diffusion_equation_solution_region}
  \mathcal{F}: \mathbb{R} \times [0, \sigma^2T/2]
\end{align}
Although we might solve \eqref{eq:lcp:overview:variational_inequalities} without transforming the problem, Dewynne et al. \cite{dewynne_howison_rupf_wilmott_1993} and Seydel \cite{seydel_2009} suggest that transformation to the heat diffusion equation \eqref{eq:lcp:overview:heat_diffusion_equation_PDE} introduces desirable numerical properties when applying finite difference schemes. Hence, under given transformation, the system of variational inequalities \eqref{eq:lcp:overview:variational_inequalities} becomes
\begin{align}
  \begin{cases}
    \bigg[\dfrac{\partial{y}}{\partial{\tau}} - \dfrac{\partial^2{y}}{\partial{x^2}}\bigg] \cdot [y(x, \tau) - h(x, \tau)] = 0 & \text{for all $(x,\tau)$} \\
    y(x, \tau) - h(x, \tau) \ge 0 & \text{for all $(x, \tau)$}\\
    \dfrac{\partial{y}}{\partial{\tau}} - \dfrac{\partial^2{y}}{\partial{x^2}} \ge 0 &  \text{for all $(x, \tau)$}\\
    y(x, 0) = h(x, 0) \\  
  \end{cases}
  \label{eq:lcp:overview:variational_inequalities_heat_equation}
\end{align}
Now that we have formulated the linear complementary problem in terms of the heat equation, we proceed to apply the finite difference scheme framework presented in section \ref{sec:finitedifferencesschemes}. Specifically, we want to approximate $y(x, \tau)$ at each of the nodes within the grid 
\begin{align}
  \mathcal{G} := \{(x_i, \tau_n): (i, n) \in \{0, \dots, M+1\} \times \{0, \dots, N+1\}\}
\end{align}
where 
\begin{align}
  \label{eq:lcp:overview:grid_2}
  x_i &:= x_{\text{min}} + i\Delta{x} &  \qquad \text{for $i = 0,\dots, M+1$} \\
  \tau_n &:= t_{\text{min}} + i{\Delta{\tau}} & \qquad \text{for $i = 0,\dots, N+1$} \\
  \Delta{x} &:= \dfrac{x_{\text{max}} - x_{\text{min}}}{M+1} \\ 
  \Delta{t} &:= \dfrac{t_{\text{max}} - t_{\text{min}}}{N+1}
\end{align}

where $x_{\text{min}}$ is set to a sufficiently small value, $x_{\text{max}}$ to a sufficiently large value, $\tau_{\text{min}}$ to 0 and $\tau_{\text{max}}$ to $\dfrac{\sigma^2}{2}T$

We dedicate the Crank-Nicholson scheme. Moreover, the Crank-Nicholson approximation is given by 
\begin{equation}
  \dfrac{y^{n+1}_{i} - y^{n}_{i}}{\Delta \tau} = \dfrac{1}{2}\bigg(\dfrac{y^{n}_{i-1} - 2y^{n}_{i} + y^{n}_{i+1}}{(\Delta x)^2} + \dfrac{y^{n+1}_{i-1} - 2y^{n+1}_{i} + y^{n+1}_{i+1}}{(\Delta x)^2}\bigg)
\end{equation}

\begin{figure}[H]
  \label{fig:lcp:thetamethod:stencil}
  \centering
  \includegraphics[scale=0.75]{chapters/chapter5/CrankNicholsonStencil.pdf}
  \caption{Stencil diagram for the Crank-Nicholson scheme.}
\end{figure}

\subsection{Theta method}
The theta method is a combination of the explicit and implicit method controlled by parameter $\theta\in[0,1]$
\begin{equation}
  \label{eq:lcp:thetamethod:finitedifference_approximation}
  \dfrac{y^{n+1}_{i} - y^{n}_{i}}{\Delta \tau} = (1-\theta)\dfrac{y^{n}_{i-1} - 2y^{n}_{i} + y^{n}_{i+1}}{(\Delta x)^2} +  \theta\dfrac{y^{n+1}_{i-1} - 2y^{n+1}_{i} + y^{n+1}_{i+1}}{(\Delta x)^2}
\end{equation}
for $i = 1,\dots,M$, and $n = 0,\dots,N$. The PDE approximation is fully explicit for $\theta = 0$, fully implicit for $\theta = 1$, and Crank-Nicholson for $\theta=1/2$. Introducing $\lambda = \Delta{t} / \Delta{x}^2$, and rearranging \eqref{eq:lcp:thetamethod:finitedifference_approximation}, we have
\begin{equation}
  \label{eq:lcp:thetamethod:finitedifference_approximation_2}
  y^{n+1}_{i} - \lambda\theta(y^{n+1}_{i-1} - 2y^{n+1}_{i} + y^{n+1}_{i+1}) =  y^{n}_{i} + (1-\theta)\lambda(y^{n}_{i-1} - 2y^{n}_{i} + y^{n}_{i+1})
\end{equation}
for $i = 1,\dots,M$, and $n = 0,\dots,N$. Also, let us define $b^n_i$
\begin{equation}
  b^{n}_i := y^{n}_{i} + (1-\theta)\lambda(y^{n}_{i-1} - 2y^{n}_{i} + y^{n}_{i+1})
\end{equation}
for $i = 1,\dots,M$, and $n = 0,\dots,N$. Then, we define the vectors $\mathbf{b}^n\in\mathbb{R}^{M}$, $\mathbf{y}^{n+1}\in\mathbb{R}^{M}$ and $\mathbf{h}^n\in\mathbb{R}^{M}$ ad
\begin{align}
  \mathbf{b}^{n} :=& \begin{bmatrix}
    b^{n}_1, \dots, b^{n}_{M}
  \end{bmatrix}^{T}\\
  \mathbf{y}^{n+1} :=& \begin{bmatrix}
    y^{n+1}_1, \dots, y^{n+1}_{M}
  \end{bmatrix}^{T}\\
  \mathbf{h}^{n+1} :=& \begin{bmatrix}
    h(x_1, \tau_{n+1}), \dots, h(x_M, \tau_{n+1})
  \end{bmatrix}^{T}
\end{align}
and the matrix $\mathbf{A}\in\mathbb{R}^{M \times M}$ as
\begin{align}
  A :=& \begin{bmatrix}
    1+2\theta & -\lambda\theta & & 0 \\ 
   -\lambda\theta & \ddots & \ddots \\
      & \ddots & \ddots & \ddots \\
    0 & & \ddots & \ddots & \\
  \end{bmatrix} 
\end{align}
Hence, we can write the equations in \eqref{eq:lcp:overview:variational_inequalities_heat_equation} as  
\begin{align}
  \begin{cases}
    (\mathbf{A}\mathbf{y}^{n+1} + \mathbf{b}^{n})^{T}(\mathbf{y}^{n+1}- \mathbf{h}^{n+1}) = 0\\
    \mathbf{A}\mathbf{y}^{n+1} + \mathbf{b}^{n} \ge 0\\
    \mathbf{y}^{n+1}- \mathbf{h}^{n+1} \ge 0 \\
    \mathbf{y}^{0} = \mathbf{h}^{0} \\  
  \end{cases}
  \label{eq:lcp:overview:variational_inequalities_heat_equation_2}
\end{align}
which is the LCP formulation. Note that if we define the vectors
\begin{align*}
  \mathbf{v} := \mathbf{y}^{n+1} - \mathbf{h}^{n+1} \\
  \mathbf{w} := \mathbf{A}\mathbf{y}^{n+1} - \mathbf{b}^{n} \\
  \mathbf{r} := \mathbf{b}^n - \mathbf{A}\mathbf{h}^{n+1}
\end{align*}

\begin{align}
    b^{n}_{i} := y^{n}_{i} + (1-\theta)\lambda(y^{n}_{i-1} - 2y^{n}_{i} + y^{n}_{i+1})
\end{align}

\begin{align}
  h^{n}_{i} := h(x_i, \tau_n)
\end{align}

\subsection{PSOR method}

\subsection{Numerical results}
\section{Conclusions}

An option provides the right to buy or sell an asset at a predetermined strike price in the future. Generally, investment firms are responsible for writing these contracts and selling them to investors. Subsequently, investors hold these contracts to hedge against potential changes in price. When a holder already owns an asset and wants to hedge against a potential price drop, they enter into a put option. On the contrary, if a holder aims to hedge against an increase in the price of an asset they intend to acquire, they enter into a call option contract. A wide variety of options are available in the market. Among the most notorious contracts, we have European and American options. European options are contracts that can exercise at the expiration date only. Likewise, American options are contracts that can be exercised before or at the maturity date. 

Investment firms charge premiums to investors for entering an option contract. Then, the writer uses the premium to hedge the possible claims that the holder will have in the future. Charging the correct premium is important because it minimizes the arbitrage opportunities for either the holder and writer. Clearly, pricing schemes depend on the type of the contract. In general, The Black-Scholes formula is used to price European options. Sadly, no formula is available for American options. Therefore, firms rely on numerical methods to come up with some approximation of the price. Numerous numerical methods for pricing American options derive from the Black-Scholes PDE. 

In this report we have implemented, and analyzed numerical schemes derived from the free boundary and the variational inequalities' formulation of the pricing problem. Specifically, we have discussed: An explicit and implicit front fixing schemes for solving the free boundary problem based on the Nielsen transformation, an explicit front fixing scheme based on the Company transformation, and the explicit, implicit and Crank-Nicholson PSOR schemes.

The explicit front fixing schemes were derived from applying central finite difference and forward/backward difference. However, they differ in how the contact point condition is approximated. Specifically, The Nielsen front fixing schemes use forward difference (or backward difference for call options) to approximate the contact point condition while the Company explicit scheme use central finite difference. This explains why Nielsen front fixing schemes yielded first order convergence with respect to the spatial discretization parameter $\Delta{x}$ while Company explicit scheme yielded second order. Moreover, as it is normally the case for explicit and implicit central finite differences, both Nielsen and Company schemes exhibits first order convergence with respect to the temporal discretization parameter $\Delta{t}$. While the explicit front fixing schemes for Nielsen and Company transformation are conditionally stable, they both proved to be substantially faster and much more accurate than the implicit scheme. 

We discourage the use of the Nielsen implicit scheme. As we already told, the implicit scheme is less accurate by far. The reason behind this is that the implicit scheme requires to solve a nonlinear system of equation at each time step in the grid. Moreover, the approximation errors produced by the nonlinear solver get accumulated over time, affecting the overall accuracy of the method. We could decrease the approximation error in the nonlinear solver by decreasing its tolerance, and increasing its maximum number of iteration. However, we found that as we do that, the overall performance of the method reduces substantially. To make things even worse, the size of the nonlinear equations is inversely proportional to the spatial discretization parameter. Therefore, the computational resources required by the implicit method grows substantially as we decrease the spatial discretization parameters. For instance, for a grid of $M$ nodes in the spatial direction, the non-linear solver needs to invert a Jacobian matrix of $M\times M$ entries. In other words, decreasing the spatial discretization parameters by a decimal point, requires 100 times more memory. To summarize, we can increase the accuracy of implicit method by decreasing the tolerance of the non-linear solver and by decreasing the discretization parameters of the grid but by sacrificing the performance of the method and increasing the memory consumption substantially.

Similar to the Company front fixing schemes, the PSOR schemes showed to have second order convergence with respect to the spatial discretization parameter $\Delta{x}$. Moreover, it showed to have first order convergence for the explicit ($\theta=0$) and implicit ($\theta=1$) schemes, and second order convergence for the Crank-Nicholson ($\theta=0.5$) in with respect to the temporal discretization parameter $\Delta{t}$. Analogous to the explicit front fixing schemes, the PSOR explicit scheme is conditionally stable. When comparing the performance of the explicit PSOR to the explicit front fixing schemes, the explicit front fixing schemes showed to be substantially faster. In that regard, the issue with explicit PSOR is that as you decrease $\Delta{x}$, the complementary equations grow, hence, taking more time to solve the linear complementary problem. Similar argument can be done when comparing explicit front fixing schemes to the implicit and Crank-Nicholson PSOR schemes. In spite of that, the Crank-Nicholson PSOR scheme is second order convergence in time, therefore, by choosing $\Delta{x}$ to be smaller than $\Delta{t}$, we could have greater performance maintaining the accuracy. 

Concluding, the numerical experiments conducted showed that the explicit front-fixing schemes offers superior performance and accuracy than the implicit front fixing scheme and the all the PSOR schemes. Between the Company explicit front fixing scheme and the Nielsen explicit front fixing scheme, we recommend using the Company transformation because it has second order convergence in space, hence, yielding to smaller approximation errors. Similarly, the explicit PSOR scheme exhibited smaller approximation errors and computational than the implicit and Crank-Nicholson PSOR schemes and the implicit front fixing scheme. However, it is still rather slow compared to the explicit front fixing schemes.

\section{Further research}

Further research opportunities arise from this report. Firstly, we saw that generally, people transform the Black-Scholes PDE to the heat diffusion equation. Although this was done for the LCP problem, the front fixing schemes were derived within the financial domain which might led to worse approximations or slower performance. Moreover, we might modify Nielsen front fixing scheme so that it approximates the contact point condition using central finite differences. Likewise, for the Nielsen implicit front fixing scheme, we could explore using nonlinear methods for large scale-scale nonlinear systems such as the one proposed by\cite{lacruz_2006}. Also, we might also derive Crank-Nicholson schemes for the Nielsen and Company front fixing schemes. Finally, we might consider using real market data to evaluate how good are the numerical schemes presented under real market conditions.

% \section{Front fixing method}

\subsection{Inverse transformation}

\begin{equation}
    \frac{\partial{V}}{\partial{t}} + \frac{1}{2}\sigma^{2} S^2 \frac{\partial^2{V}}{\partial{S^2}} + (r - \delta) S \frac{\partial{V}}{\partial{S}} - rV = 0 \quad \text{for $S > \bar{S}(t)$ and $0 \le t < T$}
\end{equation}

\begin{equation}
    V(S, t) = K - S \quad  \text{for $0 \le S \le \bar{S}(t)$ and $0 \le t < T$}
\end{equation}

\begin{equation}
    V(S, T) = \max(K - S, 0) \quad \text{for $S \ge 0$}
\end{equation}

\begin{equation}
    \frac{\partial{V}}{\partial{S}}(\bar{S}(t), t) = -1
\end{equation}

\begin{equation}
    \lim_{S\rightarrow \infty} V(S, t) = 0
\end{equation}

\begin{equation}
    \bar{S}(T) = K
\end{equation}

In order for remove the free boundary in the system of equation, the following 
transformation is used:

\begin{equation}
    x = \frac{S}{\bar{S}(t)}
\end{equation}

Next,

\begin{equation}
    v(x, t) := V(x\bar{S}(t), t) = V(S, t)
\end{equation}


By computing the partial derivatives of V with respect to S and t

\begin{equation}
    \frac{\partial{V}}{\partial{t}} =  \frac{\partial{v}}{\partial{t}} + \frac{\partial{v}}{\partial{x}} \frac{\partial{x}}{\partial{t}} 
    = \frac{\partial{v}}{\partial{t}} - x\frac{\bar{S}^\prime(t)}{\bar{S}(t)}\frac{\partial{v}}{\partial{x}} 
\end{equation}

\begin{equation}
    \frac{\partial{V}}{\partial{S}} = \frac{\partial{v}}{\partial{x}} 
    \frac{\partial{x}}{\partial{S}} = 
    \frac{1}{\bar{S}(t)} \frac{\partial{v}}{\partial{x}}
\end{equation}

\begin{equation}
    \frac{\partial^2{V}}{\partial{S^2}} =
    \frac{1}{\bar{S}(t)^2} \frac{\partial^2{v}}{\partial{x}^2}
\end{equation}

an expression for (3.1) with respect to $x$ is derived:

\begin{equation}
    \frac{\partial{v}}{\partial{t}} + \frac{1}{2}\sigma^{2} x^2 \frac{\partial^2{v}}{\partial{x}^2} + \bigg[(r - \delta) - \frac{\bar{S}^\prime(t)}{\bar{S}(t)}\bigg]x\frac{\partial{v}}{\partial{x}} - rv = 0 \quad \text{for $x > 1$ and $0 \le t < T$}
\end{equation}

Similarly, (3.2) is reformulated in term of x to:

\begin{equation}
    v(x, t) = K - x\bar{S}(t) \quad  \text{for $0 \le x \le 1$ and $0 \le t < T$}
\end{equation}

Next, the terminal condition (3.3) is re-written with respect of x:

\begin{equation}
    v(x, T) = \max(K - x\bar{S}(T), 0) = K \max(1 - x, 0) = 0 \quad \text{for $x \ge 1$}
\end{equation}

Finally, the left and right boundary conditions are given with respect to x:

\begin{equation}
    \frac{\partial{v}}{\partial{x}}(x, t) = -\bar{S}(t)
\end{equation}

\begin{equation}
    \lim_{x \rightarrow \infty} v(x, t) = 0
\end{equation}

In summary, a non linear system of PDEs is obtained:

\begin{equation}
    \frac{\partial{v}}{\partial{t}} + \frac{1}{2}\sigma^{2} x^2 \frac{\partial^2{v}}{\partial{x}^2} + \bigg[(r - \delta) - \frac{\bar{S}^\prime(t)}{\bar{S}(t)}\bigg]x\frac{\partial{v}}{\partial{x}} - rv = 0 \quad \text{for $x > 1$ and $0 \le t < T$}
\end{equation}

\begin{equation}
    v(x, t) = K - x\bar{S}(t) \quad  \text{for $0 \le x \le 1$ and $0 \le t < T$}
\end{equation}

\begin{equation}
    v(x, T) = 0 \quad \text{for $x \ge 1$}
\end{equation}

\begin{equation}
    \frac{\partial{v}}{\partial{x}}(x, t) = -\bar{S}(t)
\end{equation}

\begin{equation}
    \lim_{x \rightarrow \infty} v(x, t) = 0
\end{equation}

\begin{equation}
    \bar{S}(T) = K
\end{equation}
\newpage

\appendix


\section{Explicit scheme for Company transformation} 
The explicit scheme is given by
\begin{equation*}
    \begin{split}
      \dfrac{v^{n+1}_{i} - v^{n}_{i}}{\Delta{t}} & - \dfrac{1}{2}\sigma^2 \dfrac{v^{n}_{i-1} - 2v^{n}_{i} + v^{n}_{i+1}}{(\Delta{x})^2} \\ 
       & - \bigg( (r-\delta) - \dfrac{\sigma^2}{2} - \dfrac{\bar{S}^{n+1}_{f} - \bar{S}^{n}_{f}}{\Delta{t}\bar{S}^{n}_{f}} \bigg)\dfrac{v^{n}_{i+1} - v^{n}_{i-1}}{2\Delta{x}} + rv^{n}_{i} = 0
    \end{split}
\end{equation*}
for $i=1\dots,M$ and $n = 0,\dots,N$. 
\begin{equation*}
    \begin{split}
        v^{n+1}_{i} = av^{n}_{i-1} + bv^{n}_{i} + cv^{n}_{i+1} + \dfrac{\bar{S}^{n+1}_{f} - \bar{S}^{n}_{f}}{2\Delta{x}\bar{S}^{n}_{f}}(v^{n}_{i+1} - v^{n}_{i-1})
    \end{split}
\end{equation*}
where 
\begin{equation*}
    \begin{split}
        \lambda =& \dfrac{\Delta{t}}{\Delta{x}^2}\\
        a =& \dfrac{\lambda}{2}\bigg( \sigma^2 - \bigg(r - \delta - \dfrac{\sigma^2}{2}\bigg)\Delta{x} \bigg) \\
        b =& 1 - \sigma^2\lambda- r\Delta{t} \\
        c =& \dfrac{\lambda}{2}\bigg(\sigma^2 + \bigg(r - \delta - \dfrac{\sigma^2}{2}\bigg)\Delta{x}\bigg)
    \end{split}
\end{equation*}
Moreover, the boundary conditions 
\begin{equation*}
    \begin{split}
        \text{\textbf{Call:}} \qquad & v^{n}_{0} = 0, \quad v^{n}_{M+1} = \bar{S}^{n}_{f} - 1 \\
        \text{\textbf{Put:}} \qquad & v^{n}_{0} = 1 - \bar{S}^{n}_{f}, \quad v^{n}_{M+1} = 0 \\
    \end{split}
\end{equation*}
for $n=0,\dots,N$. Moreover, the contact point condition is approximated using central finite difference
\begin{align*}
    \text{\textbf{Call:}} \qquad & \dfrac{v^{n}_{M+2} - v^{n}_{M}}{2\Delta{x}^2} = \dfrac{\partial{v}}{\partial{x}}(0, t) + O(\Delta{x}^2) \\
    \text{\textbf{Put:}} \qquad & \dfrac{v^{n}_{1} - v^{n}_{-1}}{2\Delta{x}} = \dfrac{\partial{v}}{\partial{x}}(0, t)+ O(\Delta{x}^2) 
\end{align*}

for $n=0,\dots,N$. Moreover, the contact point condition is approximated using central finite difference
\begin{align}
    \label{eq:appendix:explicitmethodcompany:contact_point_approximation}
    \text{\textbf{Call:}} \qquad & \dfrac{v^{n}_{M+2} - v^{n}_{M}}{2\Delta{x}^2} = \bar{S}^{n}_{f} \\
    \text{\textbf{Put:}} \qquad & \dfrac{v^{n}_{1} - v^{n}_{-1}}{2\Delta{x}}  = -\bar{S}^{n}_{f}  
\end{align}

By combining the central difference approximation for the PDE, the boundary conditions and the contact point, it is obtained 

\begin{align*}
    \text{\textbf{Call:}} \qquad & v^{n}_{M} =  \\
    \text{\textbf{Put:}} \qquad & v^{n}_{1} = \alpha - \beta\bar{S}^{n}_{f}  
\end{align*}

\begin{algorithm}[H]
    \caption{Explicit method for put options}\label{alg:appendix:companytransformation:explicits:put_explicit_method_algorithm}
    \begin{algorithmic}
    \For{$i = 0,\dots,M+1$} 
      \State $v^{0}_i = 0 $
    \EndFor
    \State $\bar{S}_{f}^{0} = K$

    \State $a = \dfrac{\lambda}{2}\bigg(\sigma^2 - \bigg(r - \delta - \dfrac{\sigma^2}{2}\bigg)\Delta{x}\bigg)$
    \State $b = 1 - \sigma^2\lambda- r\Delta{t} $
    \State $c = \dfrac{\lambda}{2}\bigg(\sigma^2 + \bigg(r - \delta - \dfrac{\sigma^2}{2}\bigg)\Delta{x}\bigg)$
    \State $\alpha = 1 + \dfrac{r\Delta{x}^2}{\sigma^2}$
    \State $\beta = 1 + \Delta{x} + \dfrac{1}{2}\Delta{x}^2$

    \For{$n = 0, \dots, N$}
      \State $d^n = \dfrac{\alpha - (av^{n}_{0} + bv^{0}_{1} + cv^{n}_{2} - (v^{n}_{2} - v^{n}_{0})/(2\Delta{x}))}{(v^{n}_{2} - v^{n}_{0})/(2\Delta{x}) + \beta\bar{S}^{n}_f}$
      \State $\bar{S}^{n+1}_{f}=d^{n}\bar{S}^{n}_{f}$
      \State $a^{n} = a - \dfrac{\bar{S}^{n+1}_{f} - \bar{S}^{n}_{f}}{2\Delta{x}\bar{S}^{n}_{f}}$
      \State $c^{n} = c - \dfrac{\bar{S}^{n+1} - \bar{S}^{n}}{2\Delta{x}\bar{S}^{n}_{f}}$

      \State $v^{n+1}_{0}=1 - \bar{S}^{n+1}_{f}$
      \State $v^{n+1}_{1}=\alpha - \beta\bar{S}^{n+1}_{f}$
      \State $v^{n+1}_{M+1} = 0$
      \For{$i = 2, \dots, M$}
        \State $v^{n+1}_{i} = a^{n} v^{n}_{i-1} + b v^{n}_{i} + c^{n}v^{n}_{i+1}$
      \EndFor
    \EndFor
  \end{algorithmic}
  \end{algorithm}
\section{Python implementation}
The following python code is simplified version of the final implementation. A complete version can be found at \url{https://github.com/Alcruz/math4062-dissertation}.

\begin{lstlisting}[language=Python, caption=Base classses for options solver.]
    class Option(ABC):
        def __init__(self, type: OptionType, K: float, T: float):
            self.type = type
            self.K = K # Strike price
            self.T = T # Maturity

        @abstractmethod
        def payoff(self, S: float):
            pass
    class CallOption(Option):
        def __init__(self, K: float, T: float):
            super().__init__(OptionType.CALL, K, T)
        def payoff(self, S: float):
            return np.maximum(S - self.K, 0) 
    class PutOption(Option):
        def __init__(self, K: float, T: float):
            super().__init__(OptionType.PUT, K, T)
        def payoff(self, S: float):
            return np.maximum(self.K - S, 0)
    class Solver(ABC): 
        def __init__(self, 
            option: Option,
            r: float,  # risk-free interest rate
            sigma: float,  # sigma price volatitliy
            dx: float,  # grid resolution along x-axis
            dt: float,  # grid resolution along t-axis
            delta # dividends
        ) -> None:
            self.option = option
            self.r = r
            self.sigma = sigma
            self.dx = dx
            self.dt = dt
            self.delta = delta
\end{lstlisting}
\begin{lstlisting}[language=Python, caption=Explicit solver for Nielsen transformation.]
    class ExplicitSolver(Solver):
        def __init__(self, 
            option: Option,
            r: float,  # risk-free interest rate
            sigma: float,  # sigma price volatitliy
            dx: float,  # grid resolution along x-axis
            dt: float,  # grid resolution along t-axis
            delta # dividends
        ) -> None:
            self.option = option
            self.r = r
            self.sigma = sigma
            self.dx = dx
            self.dt = dt
            self.delta = delta
            lambd = dt / np.power(dx, 2)
            alpha = dt / dx
            self.A = 0.5 * np.power(sigma*self.x_axis, 2) * lambd - 0.5 * self.x_axis * ((r-delta) - (1/dt)) * alpha
            self.B = 1 - np.power(sigma*self.x_axis, 2) * lambd - r*dt
            self.C = 0.5 * np.power(sigma*self.x_axis, 2) * lambd + 0.5 * self.x_axis * ((r-delta) - (1/dt)) * alpha
        def solve(self):
            V = np.zeros_like(self.x_axis)
            S_bar = self.option.K
            for _ in np.arange(0, self.option.T, self.dt):
                D = 0.5*self.x_axis/self.dx
                D[1:-1] *= (V[2:]-V[:-2]) * (1/S_bar)
                
                S_bar = self.compute_time_iteration(V, D)
            return S_bar*self.x_axis, V, S_bar
        @abstractmethod
        def compute_time_iteration(self, V: np.ndarray, D: np.ndarray):
            pass
    class CallOptionExplicitSolver(ExplicitSolver):
        def __init__(self, 
            option: CallOption,
            r: float,  # risk-free interest rate
            sigma: float,  # sigma price volatitliy
            dx: float,  # grid resolution along x-axis
            dt: float,  # grid resolution along t-axis
            delta=0  # dividends
        ):
            self.x_axis = np.arange(0, 1+dx, dx)
            super().__init__(option, r, sigma, dx, dt, delta)
        def compute_time_iteration(self, V: np.ndarray, D: np.ndarray):
            S_bar = self.option.K + self.A[-2]*V[-3] + self.B[-2]*V[-2] + self.C[-2]*V[-1]
            S_bar /= 1 - self.dx - D[-2]
    
            V[1:-2] = self.A[1:-2]*V[:-3] + self.B[1:-2]*V[1:-2] \
                + self.C[1:-2]*V[2:-1] + D[1:-2]*S_bar
            V[-2] = (1-self.dx)*S_bar - self.option.K
            V[-1] = S_bar - self.option.K
            return S_bar
    class PutOptionExplicitSolver(ExplicitSolver):
        def __init__(self, 
            option: PutOption,
            r: float,  # risk-free interest rate
            sigma: float,  # sigma price volatitliy
            dx: float,  # grid resolution along x-axis
            dt: float,  # grid resolution along t-axis
            x_max: 2.,  # sufficiently large value for x
            delta=0  # dividends
        ):
            self.x_axis = np.arange(1, x_max+dx, dx)
            super().__init__(option, r, sigma, dx, dt, delta)
        def compute_time_iteration(self, V: np.ndarray, D: np.ndarray) -> float:
            S_bar = self.option.K - (self.A[1]*V[0] + self.B[1]*V[1] + self.C[1]*V[2])
            S_bar /= D[1] + 1 + self.dx
            V[2:-1] = self.A[2:-1]*V[1:-2] + self.B[2:-1] * \
                V[2:-1] + self.C[2:-1]*V[3:] + D[2:-1]*S_bar
            V[0] = self.option.payoff(S_bar)
            V[1] = self.option.payoff((1+self.dx)*S_bar)
            return S_bar
\end{lstlisting}

\begin{lstlisting}[language=Python, caption=Implicit solver for Nielsen transformation]
    class ImplicitSolver(Solver):
        def __init__(self, 
            option: Option,
            r: float,  # risk-free interest rate
            sigma: float,  # sigma price volatitliy
            dx: float,  # grid resolution along x-axis
            dt: float,  # grid resolution along t-axis
            delta=0, # dividends
        ) -> None:
            self.option = option
            self.r = r
            self.sigma = sigma
            self.dx = dx
            self.dt = dt
            self.delta = delta
            self.lambd = self.dt/np.power(self.dx, 2)
            self.kappa = self.dt/self.dx
            self.alpha = 1 + self.lambd*np.power(self.sigma*self.x_axis, 2) + self.r*self.dt
            self.M = self.x_axis.size 
        def beta(self, S, S_bar):
            return -0.5*self.lambd*np.power(self.sigma, 2)*np.power(self.x_axis, 2) + 0.5*self.kappa*self.x_axis*((self.r-self.delta) - (S_bar - S)/(self.dt*S))
        def gamma(self, S, S_bar):
            return -0.5*self.lambd*np.power(self.sigma, 2)*np.power(self.x_axis, 2) - 0.5*self.kappa*self.x_axis*((self.r-self.delta) - (S_bar - S)/(self.dt*S))
        @abstractmethod
        def solve_non_linear_system(self, V, S_bar) -> float:
            pass
        def solve(self):
            S_bar = self.option.K
            V = np.zeros_like(self.x_axis)
            for _ in np.arange(0, self.option.T, self.dt):
                S_bar = self.solve_non_linear_system(V, S_bar)
            return S_bar*self.x_axis, V, S_bar
    class CallOptionImplicitSolver(ImplicitSolver):
        def __init__(self, 
            option: Option,
            r: float,  # risk-free interest rate
            sigma: float,  # sigma price volatitliy
            dx: float,  # grid resolution along x-axis
            dt: float,  # grid resolution along t-axis
            delta=0, #dividends
            maxiter=1000,
            tolerance=1e-24,
            method='lm'
        ) -> None:
            self.x_axis = np.arange(0, 1+dx, dx)
            super().__init__(option, r, sigma, dx, dt, delta)
            self.maxiter=maxiter
            self.tolerance=tolerance
            self.method=method
        def jacobian(self, y, S_bar):
            p, s, = y[:-1], y[-1]
            beta = self.beta(s, S_bar)
            gamma = self.gamma(s, S_bar)
            dgamma_ds = - 0.5 * (1/self.dx) * self.x_axis * S_bar / s**2
            dbeta_ds = 0.5 * (1/self.dx) * self.x_axis * S_bar / s**2
            retVal = diags([self.alpha[2:], beta[1:-1], gamma[1:-1]],
                        [-1, 0, 1], shape=(self.M-2, self.M-2)).toarray()
            retVal[-1, :] = 0
            retVal[:, -1] = 0
            retVal[0, -1] = dbeta_ds[1]*p[0] + dgamma_ds[1]*p[1]
            retVal[1:-2, -1] = dbeta_ds[2:-3]*p[1:-1] + dgamma_ds[2:-3]*p[2:]
            retVal[-2, -1] = dbeta_ds[-3]*p[-2]
            retVal[-2, -1] -= dgamma_ds[-3]*self.option.payoff((1-self.dx)*s) + gamma[-3]*(1 - self.dx)
            retVal[-1, -2] = self.alpha[-2]
            retVal[-1, -1] -= dgamma_ds[-2]*self.option.payoff(s) + gamma[-2] + dbeta_ds[-2]*self.option.payoff((1-self.dx)*s) - beta[1]*(1-self.dx)
            return retVal
        def system(self, y, b, S_bar):
            *v, s = y
            beta = self.beta(s, S_bar)
            gamma = self.gamma(s, S_bar)
            A = diags([self.alpha[2:-1], beta[1:-2], gamma[1:-3]],
                    [-1, 0, 1], shape=(self.M-2, self.M-3)).toarray()
            f = b[1:-1]
            f[-2] -= gamma[-3]*((1-self.dx)*s-self.option.K)
            f[-1] -= gamma[-2]*(s-self.option.K) + beta[-2]*((1-self.dx)*s - self.option.K)
            res = A@v - f
            return res
        def solve_non_linear_system(self, V: np.ndarray, S_bar: np.ndarray):
            *V[1:-2], S_bar = root(
                lambda y: self.system(y, np.copy(V[:]), S_bar),
                # jac=lambda y: self.jacobian(y, S_bar),
                x0=np.concatenate([V[1:-2], [S_bar]]),
                method=self.method,
                options=dict(xtol=self.tolerance, maxiter=self.maxiter)
            )['x']
            V[-2] = (1-self.dx)*S_bar-self.option.K
            V[-1] = S_bar-self.option.K
            return S_bar
    class PutOptionImplicitSolver(ImplicitSolver):
        def __init__(self, 
            option: Option,
            r: float,  # risk-free interest rate
            sigma: float,  # sigma price volatitliy
            dx: float,  # grid resolution along x-axis
            dt: float,  # grid resolution along t-axis
            delta=0, # dividends
            x_max=2,
            maxiter=1000,
            tolerance=1e-24,
            method='lm'
        ) -> None:
            self.x_axis = np.arange(1, x_max+dx, dx)
            super().__init__(option, r, sigma, dx, dt, delta)
            self.maxiter=maxiter
            self.tolerance=tolerance
            self.method=method
        def jacobian(self, y, S_bar):
            p, s, = y[:-1], y[-1]
            beta = self.beta(s, S_bar)
            gamma = self.gamma(s, S_bar)
    
            dgamma_ds = - 0.5 * (1/self.dx) * self.x_axis * S_bar / s**2
            dbeta_ds = 0.5 * (1/self.dx) * self.x_axis * S_bar / s**2
    
            retVal = diags([beta[3:-1], self.alpha[2:-1], gamma[1:-1]],
                        [-2, -1, 0], shape=(self.M-2, self.M-2)).toarray()
            retVal[-1, :] = 0
            retVal[:, -1] = 0
            retVal[0, -1] = dgamma_ds[1]*p[0] + dbeta_ds[1] * \
                (self.option.K - S_bar) - beta[1] - self.alpha[1]*(1+self.dx)
            retVal[1, -1] = dgamma_ds[2]*p[1] + dbeta_ds[2] * \
                (self.option.K - (1+self.dx)*S_bar) - beta[2]*(1+self.dx)
            retVal[2:-1, -1] = dbeta_ds[3:-2]*p[:-2] + dgamma_ds[3:-2]*p[2:]
            retVal[-1, -3] = beta[-2]
            retVal[-1, -2] = self.alpha[-2]
            retVal[-1, -1] = dbeta_ds[-2]*p[-2]
            return retVal
        def system(self, y, b, S_bar):
            p, s, = y[:-1], y[-1]
            _beta = self.beta(s, S_bar)
            _gamma = self.gamma(s, S_bar)
            A = diags([_beta[3:-1], self.alpha[2:-1], _gamma[1:-2]],
                    [-2, -1, 0], shape=(self.M-2, self.M-3)).toarray()
            f = b[1:-1]
            f[0] -= _beta[1]*(self.option.K-s) + self.alpha[1]*((self.option.K-(1+self.dx)*s))
            f[1] -= _beta[2]*(self.option.K-(1+self.dx)*s)
            res = A@p - f
            return res
        def solve_non_linear_system(self, V: np.ndarray, S_bar: np.ndarray):
            *V[2:-1], S_bar = root(
                lambda y: self.system(y, np.copy(V[:]), S_bar),
                # jac=lambda y: self.jacobian(y, S_bar),
                x0=np.concatenate([V[2:-1], [S_bar]]),
                method=self.method,
                options=dict(xtol=self.tolerance, maxiter=self.maxiter)
            )['x']
            V[0] = self.option.payoff(S_bar)
            V[1] = self.option.payoff((1+self.dx)*S_bar)    
            return S_bar
\end{lstlisting}

\section{Code for numerical experiments} \label{sec:numericalexperiments}
\section{Code for numerical experiments} \label{sec:numericalexperiments}

\section{Code for convergence analysis} \label{sec:convergenceanalysis}
\vspace{2cm}
\includegraphics[scale=0.9, clip, trim={2.5cm, 2.5cm 0mm 7cm}]{chapters/appendix/convergence_analysis}
\includepdf[pages=2-, scale=0.8, pagecommand={\thispagestyle{plain}}, clip,trim=2.5cm 25mm 0mm 0mm]{chapters/appendix/convergence_analysis}
\newpage
\printbibliography

\end{document}
