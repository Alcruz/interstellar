\section{Finite difference schemes}

\subsection{Overview}

In this section, we present explicit and implicit finite difference schemes
for solving the non-linear PDE problem in
\eqref{eq:blackscholes:frontfixingmethod:inversetransform:american_options_bs_pde}. 
The explicit and implicit schemes of \eqref{eq:blackscholes:frontfixingmethod:logtransform:american_options_bs_pde}
are provided in appendix A.

Previously, we  considered the pricing problem of American options which requires
solving the free boundary problem defined in 
\eqref{eq:blackscholes:preliminaries:american_options_pde_free_boundary_problem_full}.
Then, we presented the front fixing method used to rewrite the PDE problem using
a fixed boundary by applying the inverse transformation presented in \cite{nielsen_2001}
and the log transformation presented in \cite{company_egorova_jodar_2014} resulting
in the systems \eqref{eq:blackscholes:frontfixingmethod:inversetransform:american_options_bs_pde}
and \eqref{eq:blackscholes:frontfixingmethod:logtransform:american_options_bs_pde}. 

Recall that the solution $v(x,t)$ of 
\eqref{eq:blackscholes:frontfixingmethod:inversetransform} 
is defined in the continuous region 
\begin{equation*}
  \mathcal{F}: \mathcal{X} \times \mathcal{T}
\end{equation*}
where
\begin{equation*}
  \mathcal{T}: [0, T]
\end{equation*}
\begin{equation*}
   \mathcal{X}: [0, \infty)
\end{equation*}
Now, we want to discretize $\mathcal{F}$ using a grid with resolution $\Delta x$
and $\Delta t$. Let's define the bound of the grid as:
\begin{equation*}
  x_{\text{min}} := 0
\end{equation*}
\begin{equation*}
  x_{\text{max}} := x_{\infty}
\end{equation*}
\begin{equation*}
  t_{\text{min}} := 0 
\end{equation*}
\begin{equation*}
  t_{\text{max}} := T
\end{equation*}

where $x_{\infty}$ means an arbitrary large value in this context. Moreover, the grid
dimension will be given by

\begin{align}
  M &:= \dfrac{x_{\text{max}} - x_{\text{min}}}{\Delta x} \\ 
  N &:= \dfrac{t_{\text{max}} - t_{\text{min}}}{\Delta t} \\ 
  x_i &:= x_{\text{min}} + i\Delta x & \text{for $i = 0,\dots, M$} \\
  t_i &:= t_{\text{min}} + i{\Delta t} & \qquad \text{for $i = 0,\dots, N$}
\end{align}
Hence, the grid $\mathcal{G}$ is defined as
\begin{align}
  \mathcal{G} := \{(x_i, t_n): (i, n) \in \{0,\dots,M\}\times\{0,\dots,N\}\}
\end{align}
Thus, solving equation \eqref{eq:blackscholes:frontfixingmethod:inversetransform:american_options_bs_pde}
numerically is the find approximations 
\begin{align*}
  v^{n}_i \approx v(x_i,t_n)
\end{align*}
and
\begin{align*}
  \bar{S}^{n} \approx \bar{S}(t_n)
\end{align*}
Since \eqref{eq:blackscholes:frontfixingmethod:inversetransform:american_options_bs_pde}
and \eqref{eq:blackscholes:frontfixingmethod:logtransform:american_options_bs_pde}
are written backward in time, all information of the previous time step is contained
at $t_{n+1}$. Therefore, as an explicit scheme, we will use backward central difference 
to approximate the (partial) derivates of $v(x, t)$ and $\bar{S}(t)$. Conversely,
we will use forward central difference as an implicit scheme. 

\subsection{Explicit scheme}

In the explicit scheme, we use the information available in the previous time 
step $t_{n+1}$ when computing the central finite difference. 
Therefore, we approximate the spatial partial derivatives of $v(x,t)$ at the nodes
$(x_{i},t_{n})$ as
\begin{equation}
  \dfrac{v^{n+1}_{i+1} - v^{n+1}_{i-1}}{2 \Delta{x}} = \dfrac{\partial{v}}{\partial{x}}+ O((\Delta{x})^2) \qquad \text{for $i = 1, \dots, M - 1$}
\end{equation}

\begin{equation}
  \dfrac{v^{n+1}_{i+1} - 2v^{n+1}_{i} + v^{n+1}_{i-1}}{\Delta{x}^2} = \dfrac{\partial^2{v}}{\partial{x^2}}+ O(h^2) \qquad \text{for $i = 1, \dots, M - 1$} \\
\end{equation}

Analogously, we approximate the time partial derivative of $v(x, t)$ and the time derivative of $\bar{S}(t)$
as
\begin{equation}
  \dfrac{v^{n+1}_{i} - v^{n}_{i-1}}{\Delta{t}} = \dfrac{\partial{v}}{\partial{t}}+ O(\Delta{t}) \qquad \text{for $n = N-1,\dots,0$ }
\end{equation}

\begin{equation}
  \dfrac{\bar{S}^{n+1}-\bar{S}^{n}}{\Delta t} = \bar{S}'(t) + O(\Delta{t}) \qquad \text{for $n = N-1,\dots,0$ }
\end{equation}

Using the finite difference approximations above, and approximation of the PDE in \eqref{eq:blackscholes:frontfixingmethod:inversetransform:american_options_bs_pde} 
is given by 
\begin{equation*}
  \begin{split}
    \dfrac{v^{n+1}_{i} - v^{n}_{i}}{\Delta{t}} & + \dfrac{1}{2}\sigma^2 x_i^2 \dfrac{v^{n+1}_{i-1} - 2v^{n+1}_{i} + v^{n+1}_{i+1}}{(\Delta{x})^2} \\ 
     & + x_i\bigg( (r-\delta) - \dfrac{1}{\bar{S}^{n+1}}\dfrac{\bar{S}^{n+1} - \bar{S}^{n}}{\Delta{t}} \bigg)\dfrac{v^{n+1}_{i+1} - v^{n+1}_{i-1}}{2\Delta{x}} - rv^{n+1}_{i} = 0
  \end{split}
\end{equation*}

for $i = 1, \dots, M-1$ and $t = N-1, \dots, 0$. To simplify the expression above, we introduce the terms
\begin{align*}
  \lambda &:= \dfrac{\Delta{t}}{(\Delta{x})^2} \\
  A_i &:= \dfrac{\lambda}{2}\sigma^2x^{2}_i - \dfrac{\lambda}{2}\bigg((r-\delta) - \dfrac{1}{\Delta{t}}\bigg)x_i\Delta{x} & \text{for $i = 1, \dots, M - 1$} \\ 
  B_i &:= 1 - \lambda\sigma^2x_i^2 - r\Delta{t} & \text{for $i = 1, \dots, M - 1$} \\
  C_i &:= \dfrac{\lambda}{2}\sigma^2x^{2}_i + \dfrac{\lambda}{2}\bigg((r-\delta) - \dfrac{1}{\Delta{t}}\bigg)x_i\Delta{x} &  \text{for $i = 1, \dots, M - 1$} \\
  D^{n+1}_{i} &:= \dfrac{x_i}{2\Delta{x}}\dfrac{v^{n+1}_{i+1} - v^{n+1}_{i-1}}{\bar{S}^{n+1}} &  \text{for $i = 1, \dots, M - 1$}
\end{align*}

Then, we rearrange the finite difference approximation of the PDE as 

\begin{equation*}
  v^{n}_{i} - D^{n+1}_{i}\bar{S}^n = A^{n+1}_i v^{n+1}_{i-1} + B^{n+1}_{i}v^{n+1}_{i} + C^{n+1}_{i}v^{n+1}_{i} 
\end{equation*}

for $i = 1, \dots, M-1$ and $t = N-1, \dots, 0$. Moreover, we have 
well-defined boundary conditions 

\begin{subequations}
  \begin{equation*}
    v^{n}_{0} = 0 \qquad v^{n}_{M} = \bar{S}^{n} - K \qquad \text{for $n=N-1,\dots,0$}
  \end{equation*}
  \begin{equation*}
    v^{n}_{0} = K - \bar{S}^{n} \qquad v^{n}_{M} = 0 \qquad \text{for $n=N-1,\dots,0$}
  \end{equation*}
\end{subequations}

and terminal condition

\begin{subequations}
  \begin{equation*}
    v^{N}_{i} = 0 \qquad \text{for $i=0,\dots,M$}
  \end{equation*}
\end{subequations}


Moreover, using the contact point condition,

\begin{subequations}
  \begin{align}
    v^{n}_{M-1} &= v^{n}_{M} - \Delta{x}\bar{S}^{n} = (1-\Delta{x})\bar{S}^n - K & \qquad \text{for $n = N-1,\dots,0$ }\\
    v^{n}_{1} &= v^{n}_{0} - \Delta{x}\bar{S}^{n} = K - (1+\Delta{x})\bar{S}^n & \qquad \text{for $n = N-1,\dots,0$ }
  \end{align}    
\end{subequations}


we obtain an explicit expression for $\bar{S}^n$ 

\begin{subequations}
  \begin{equation}
    \bar{S}^{n} = \dfrac{K + A_{M-2}v^{n+1}_{M-2} + B_{M-1}v^{n+1}_{M-1} + C_{M-1}v^{n+1}_{M}}{(1-\Delta{x}) - D^{n+1}_{M-1}} \qquad \text{for $n = N-1,\dots,0$ } \\
  \end{equation}
  \begin{equation}    
    \bar{S}^{n} = \dfrac{K - (A_{1}v^{n+1}_{0} + B_{1}v^{n+1}_{1} + C_{1}v^{n+1}_{2})}{D^{n+1}_1 + (1+\Delta{x})} \qquad \text{for $n = N-1,\dots,0$ }
  \end{equation}
\end{subequations}

Thus, the non-linear PDE problem \eqref{eq:blackscholes:frontfixingmethod:inversetransform:american_options_bs_pde}
can be rewritten as:
\begin{subequations}
  \begin{align}
    \begin{cases}
      v^{n}_{i} - D^{n+1}_{i}\bar{S}^n = A_i v^{n+1}_{i-1} + B_{i}v^{n+1}_{i} + C_{i}v^{n+1}_{i+1} & \text{for $i = 1, \dots, M-2$ and $n = N-1,\dots,0$} \\
      v^{N}_i = 0 \qquad \bar{S}^N = K & \text{for $i = 0, \dots, M$} \\
      v^{n}_0 = 0 \quad v^{n}_{M-1} = (1-\Delta{x})\bar{S}^n - K \quad v^{n}_M = \bar{S}^n - K  & \text{for $n = N-1, \dots 0$} \\
      x_{\text{min}} := 0 \quad x_{\text{max}} := 1 \\
      t_{\text{min}} := 0 \quad t_{\text{max}} := T
    \end{cases}
  \end{align}
  \begin{align}
    \begin{cases}
      v^{n}_{i} - D^{n+1}_{i}\bar{S}^n = A_i v^{n+1}_{i-1} + B_{i}v^{n+1}_{i} + C_{i}v^{n+1}_{i+1} & \text{for $i = 2, \dots, M-1$ and $n = N-1,\dots,0$} \\
      v^{N}_i = 0 \qquad \bar{S}^N = K & \text{for i = 0, \dots, M} \\
      v^{n}_{0} = K - \bar{S}^{n} \quad v^{n}_{1} =  K - (1+\Delta{x})\bar{S}^{n} \quad v^{n}_{M} = 0 & \text{for $n = N-1, \dots 0$} \\
      x_{\text{min}} := 1 \quad x_{\text{max}} := x_{\infty} \\
      t_{\text{min}} := 0 \quad t_{\text{max}} := T
    \end{cases}
  \end{align}
\end{subequations}

Therefore, we can formulate the following algorithm

\begin{algorithm}
  \caption{Explicit method for call options} \label{alg:cap}
  \begin{algorithmic}
  \Ensure $\lambda \le 0.5$
  
  \For{$i = 0,\dots,M$} 
    \State $v^{N}_i = 0 $
  \EndFor
  
  \State $\bar{S}^{N} = K$

  \For{$i = 1,\dots,M-2$} 
    \State $A_i = \dfrac{\lambda}{2}\sigma^2x^{2}_i - \dfrac{\lambda}{2}\bigg((r-\delta) - \dfrac{1}{\Delta{t}}\bigg)x_i\Delta{x}$
    \State $B_i = 1 - \lambda\sigma^2x_i^2 - r\Delta{t} $
    \State $C_i = \dfrac{\lambda}{2}\sigma^2x^{2}_i + \dfrac{\lambda}{2}\bigg((r-\delta) - \dfrac{1}{\Delta{t}}\bigg)x_i\Delta{x} $
  \EndFor
  
  \For{$n = N-1, \dots, 0$}
    \For{$i = 1, \dots, M-1$}
      \State $D^{n+1}_i = \dfrac{x_i}{2\Delta{x}}\dfrac{v^{n+1}_{i+1} - v^{n+1}_{i-1}}{\bar{S}^{n+1}}$
    \EndFor
    \State $\bar{S}^n = \dfrac{K - (A_{M-2}v^{n+1}_{M-2} + B_{M-1}v^{n+1}_{M-1} + C_{M-1}v^{n+1}_{M})}{D^{n+1}_{M-1} + (1-\Delta{x})}$
    \State $v^{n}_{0} = 0$
    \State $v^{n}_{M-1} = (1-\Delta{x})\bar{S}^{n} - K$
    \State $v^{n}_{M} = \bar{S}^{n} - K$
    \For{$i = 1, \dots, M-2$}
      \State $v^{n}_{i} = A^{n+1}_i v^{n+1}_{i-1} + B^{n+1}_{i}v^{n+1}_{i} + C^{n+1}_{i}v^{n+1}_{i} + D^{n+1}_{i}\bar{S}^n$
    \EndFor
  \EndFor
\end{algorithmic}
\end{algorithm}

\newpage

\begin{algorithm}[H]
  \caption{Explicit method for put options}\label{alg:cap}
  \begin{algorithmic}
  \Require $n \geq 0$
  \Ensure $\lambda \le 0.5$
  
  \For{$i = 0,\dots,M$} 
    \State $v^{N}_i = 0 $
  \EndFor
  
  \State $\bar{S}^{N} = K$

  \For{$i = 2,\dots,M-1$} 
    \State $A_i = \dfrac{\lambda}{2}\sigma^2x^{2}_i - \dfrac{\lambda}{2}\bigg((r-\delta) - \dfrac{1}{\Delta{t}}\bigg)x_i\Delta{x}$
    \State $B_i = 1 - \lambda\sigma^2x_i^2 - r\Delta{t} $
    \State $C_i = \dfrac{\lambda}{2}\sigma^2x^{2}_i + \dfrac{\lambda}{2}\bigg((r-\delta) - \dfrac{1}{\Delta{t}}\bigg)x_i\Delta{x} $
  \EndFor
  
  \For{$n = N-1, \dots, 0$}
    \For{$i = 1, \dots, M-1$}
      \State $D^{n+1}_i = \dfrac{x_i}{2\Delta{x}}\dfrac{v^{n+1}_{i+1} - v^{n+1}_{i-1}}{\bar{S}^{n+1}}$
    \EndFor
    \State $\bar{S}^n = \dfrac{K - (A_{1}v^{n+1}_{0} + B_{1}v^{n+1}_{1} + C_{1}v^{n+1}_{2})}{D^{n+1}_{1} + (1+\Delta{x})}$
    \State $v^{n}_{0} = K - \bar{S}^{n}$
    \State $v^{n}_{1} = K - (1+\Delta{x})\bar{S}^n$
    \State $v^{n}_{M} = 0$
    \For{$i = 2, \dots, M-1$}
      \State $v^{n}_{i} = A^{n+1}_i v^{n+1}_{i-1} + B^{n+1}_{i}v^{n+1}_{i} + C^{n+1}_{i}v^{n+1}_{i} + D^{n+1}_{i}\bar{S}^n$
    \EndFor
  \EndFor
\end{algorithmic}
\end{algorithm}

\subsection{Implicit scheme}

Analogously to the previous section, the PDE in \eqref{eq:blackscholes:frontfixingmethod:inversetransform:american_options_bs_pde} 
is approximated using forward central difference  
\begin{equation*}
  \begin{split}
    \dfrac{v^{n+1}_{i} - v^{n}_{i}}{\Delta{t}} & + \dfrac{1}{2}\sigma^2 x_i^2 \dfrac{v^{n}_{i-1} - 2v^{n}_{i} + v^{n}_{i+1}}{(\Delta{x})^2} \\ 
     & + x_i\bigg( (r-\delta) - \dfrac{1}{\bar{S}^{n}}\dfrac{\bar{S}^{n+1} - \bar{S}^{n}}{\Delta{t}} \bigg)\dfrac{v^{n}_{i+1} - v^{n}_{i-1}}{2\Delta{x}} - rv^{n}_{i} = 0
  \end{split}
\end{equation*}

Similarly to the previous section, the terms are introduced  
\begin{align}
  \alpha^{n}_{i} &:= -\dfrac{\lambda}{2}\sigma^2x^{2}_{i} + \dfrac{\lambda\Delta{x}}{2}x_{i}\bigg(r-\delta+\dfrac{\bar{S}^{n+1}-\bar{S}^n}{\Delta{t}\bar{S}^{n}}\bigg) \\
  \beta^{n}_{i} &:= 1 + \lambda\sigma^2x^{2}_{i} + r\Delta{t} \\
  \gamma^{n}_{i} &:= -\dfrac{\lambda}{2}\sigma^2x^{2}_{i} + \dfrac{\lambda\Delta{x}}{2}x_{i}\bigg(r-\delta+\dfrac{\bar{S}^{n+1}-\bar{S}^n}{\Delta{t}\bar{S}^{n}}\bigg)
\end{align}

to make the expression above more manageable. Next, the PDE approximation is rearranged as
\begin{equation*}
  \alpha^{n}_{i}v^{n}_{i-1} + \beta^{n}_{i}v^{n}_{i} + \gamma^{n}_{i}v^{n}_{i+1} = v^{n+1}_{i}
\end{equation*}

Contrary to the explicit method, there is not an explicit expression for $\bar{S^n}$. Therefore,
the PDE problem in \eqref{eq:blackscholes:frontfixingmethod:inversetransform:american_options_bs_pde}
is rewritten as
\begin{subequations}
  \begin{align}
    \begin{cases}
      \alpha^{n}_{i}v^{n}_{i-1} + \beta^{n}_{i}v^{n}_{i} + \gamma^{n}_{i}v^{n}_{i+1} = v^{n+1}_{i} & \text{for $i = 1, \dots, M-2$ and $n = N-1,\dots,0$} \\
      v^{N}_i = 0 \qquad \bar{S}^N = K & \text{for $i = 0, \dots, M$} \\
      v^{n}_0 = 0 \quad v^{n}_{M-1} = (1-\Delta{x})\bar{S}^n - K \quad v^{n}_M = \bar{S}^n - K  & \text{for $n = N-1, \dots 0$} \\
      x_{\text{min}} := 0 \quad x_{\text{max}} := 1 \\
      t_{\text{min}} := 0 \quad t_{\text{max}} := T
    \end{cases}
  \end{align}
  \begin{align}
    \begin{cases}
      \alpha^{n}_{i}v^{n}_{i-1} + \beta^{n}_{i}v^{n}_{i} + \gamma^{n}_{i}v^{n}_{i+1} = v^{n+1}_{i} & \text{for $i = 2, \dots, M-1$ and $n = N-1,\dots,0$} \\
      v^{N}_i = 0 \qquad \bar{S}^N = K & \text{for i = 0, \dots, M} \\
      v^{n}_{0} = K - \bar{S}^{n} \quad v^{n}_{1} =  K - (1+\Delta{x})\bar{S}^{n} \quad v^{n}_{M} = 0 & \text{for $n = N-1, \dots 0$} \\
      x_{\text{min}} := 1 \quad x_{\text{max}} := x_{\infty} \\
      t_{\text{min}} := 0 \quad t_{\text{max}} := T
    \end{cases}
  \end{align}
\end{subequations}

Since there is not an explicit formula for $v^{n}_{i}$ and $\bar{S}^n$, we will 
have to solve a non-linear system of equation. Let's define the vector $\mathbf{v}^n \in \mathbb{R}^{M-2}$ 
 
\begin{subequations}
  \begin{equation}
    \mathbf{v}^{n} := \begin{bmatrix}
      v^{n}_{1}, & v^{n}_{2}, & \cdots, & v^{n}_{M-2}
    \end{bmatrix}^{\text{T}}
  \end{equation}
  \begin{equation}
    \mathbf{v}^{n} := \begin{bmatrix}
      v^{n}_{2}, & v^{n}_{3}, & \cdots, & v^{n}_{M-1}
    \end{bmatrix}^{\text{T}}
  \end{equation}    
\end{subequations}

the matrix $\Lambda^{n} \in \mathbb{R}^{M-1,M-2}$ 

\begin{subequations}
  \begin{equation}
    \Lambda^{n} = \begin{bmatrix}
      \beta^{n}_{1} & \gamma^{n}_1 \\
      \alpha^{n}_{2} & \beta^{n}_{2} & \gamma^{n}_{2} \\
      & \ddots & \ddots & \ddots  \\
      & & \ddots & \ddots & \ddots  \\
      & & & \alpha^{n}_{M-3} & \beta^{n}_{M-3} & \gamma^{n}_{M-3} \\
      & & & & \alpha^{n}_{M-2} & \beta^{n}_{M-2} \\
      & & & & & \alpha^{n}_{M-2} \\
    \end{bmatrix}
  \end{equation}
  \begin{equation}
    \Lambda^{n} = \begin{bmatrix}
      \gamma^{n}_1 \\
      \beta^{n}_2 & \gamma^{n}_2 \\
      \alpha^{n}_3 & \beta^{n}_3 & \gamma^{n}_3 \\
      & \ddots & \ddots & \ddots \\
      & & \ddots & \ddots & \ddots \\
      & & & \alpha^{n}_{M-2} & \beta^{n}_{M-2} & \gamma^{n}_{M-2} \\
      & & & & \alpha^{n}_{M-1} & \beta^{n}_{M-1} \\
    \end{bmatrix}
  \end{equation}
\end{subequations}

and the vector $\mathbf{f}^{n} \in \mathbb{R}^{M-1}$

\begin{subequations}
  \begin{equation}
    \mathbf{f}^n= \begin{bmatrix}
      v^{n+1}_{1} \\
      \vdots \\
      v^{n+1}_{M-2} - \gamma^{n}_{M-2}[(1-\Delta{x})\bar{S}^{n} - K] \\
      v^{n+1}_{M-1} - \gamma^{n}_{M-1}(\bar{S}^n - K) - \beta^{n}_{M-1}[(1-\Delta{x})\bar{S}^{n} - K]
    \end{bmatrix}
  \end{equation}
  \begin{equation}
    \mathbf{f}^n= \begin{bmatrix}
      v^{n+1}_{1} - \alpha^{n}_{1}(K - \bar{S}^{n}) - \beta^{n}_{1}[K - (1+\Delta{x})\bar{S}^{n}] \\
      v^{n+1}_{2} - \beta^{n}_2[K - (1+\Delta{x})\bar{S}^{n}] \\
      v^{n+1}_{3} \\
      \vdots \\
      v^{n+1}_{M-1}
    \end{bmatrix}
  \end{equation}
\end{subequations}

Thus, the non-linear system of equations that we need to solve is

\begin{equation}
  F(\mathbf{v}^{n}, \bar{S}^{n}) = \Lambda^{n}\mathbf{v}^{n} - \mathbf{f}^n = 0
\end{equation}

By computing the Jacobian of the system, we con solve the non-linear system 
using the newton's method

\begin{equation}
  \mathbf{y}_{k+1} = \mathbf{y}_{k} - J^{-1}(\mathbf{y}_{k})F(\mathbf{y}_{k})
\end{equation}

where $y_k$ is some approximation of the solution
\begin{equation}
  \mathbf{y} = \begin{bmatrix}
    \mathbf{v}^{n} | \bar{S}^{n}
  \end{bmatrix}^{\text{T}}
\end{equation}