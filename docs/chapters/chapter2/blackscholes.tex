
\section{Black-Scholes equation} \label{sec:blackscholes}

\subsection{Preliminaries}

A common problem in finance is pricing financial derivatives, often referred to 
simply as derivatives. In essence, derivatives are contracts set between parties 
whose value over time derives from the price of their underlying assets. A 
notorious family of derivatives in financial markets are "options". Options are 
contracts set between two parties in which the holder has the right to sell or 
buy, commonly referred to as exercising, an underlying asset at a 
pre-established price, also known as the "strike price", in the future. Options 
are referred to as "call options" or "put options" if the exercise position is 
to buy or to sell, respectively. Similarly, options are classified depending on 
their exercise style. In that regard, the simplest options are European options. 
European options give the right to exercise on the expiration date of the contract. 
Another well-known type of option is the American option. American options work 
similarly to European options, with the difference that they can be exercised at
any point in time between the beginning and the expiration date of the contract.

Let's define the payoff as

\begin{subequations} \label{eq:blackscholes:preliminaries:payoff_function}
\begin{equation}
H(S, t) = \max(S - K, 0)
\end{equation}
\begin{equation}
H(S, t) = \max(K - S, 0)
\end{equation}
\end{subequations}

where $K$ is the strike price, $S \in [0, \infty]$ is the asset price, and
$t \in [0, T]$ is the current time. Note that $t$ is measured in years, where 
$t=0$ and $t=T$ denote the beginning of the contract and the expiration date, 
respectively, and the interval $[0, T]$ represents the lifespan of the option. 
While an American option's payoff is defined for all $(S, t)$, a European 
option's payoff is only defined at $t=T$.

Obviously, options provide greater flexibility to holders by eliminating their 
exposure to negative payoffs. This is why the writers charge a premium to the holders
at the time they enter the contract. The premium is often referred to as the price 
or value of the option, and the problem of determining 
this value is called option pricing. When pricing options, it is crucial to find
the fair price; otherwise, the writer or holder of the option could devise a 
scheme in which the option will always be profitable for them. In other words,
options pricing must adhere to the principle of no-arbitrage. Therefore, we 
assume that the writer of the option uses the premium to construct a portfolio 
consisting of $\phi_0$ units of the asset and invests $\psi_0$ units of cash in 
a risk-free asset, such as US Treasury bills, certificates of deposit, or a bank 
account. Then, the writer rebalances the portfolio $(\phi_0, \psi_0)$ to hedge 
against any potential claims from the holder of the option at any future time 
$0 < t \le T$. Consequently, at any time $t$, the writer holds a portfolio 
$(\phi_t, \psi_t)$ with a value

\begin{equation}
  \Pi_t = \phi_t S_t + \psi_t B_t
\end{equation}

Moreover, the portfolio is self-financing. In other words, the changes in
portfolio depend on the changes in $S_t$ and $B_t$, 
and the rebalancing of portfolio $(\phi_t, \psi_t)$

\begin{align*}
  & d\Pi_t = \phi_tdS_t + \psi_t dB_t \\
  & S_t d\phi_t + B_t d\psi_t = 0
\end{align*}

Finally, the portfolio value matches the option value

\begin{equation}
  \Pi_t = V_t
\end{equation}

at any time $0 \le t \le T$. Using the self-financing portfolio hedging 
strategy, The Black-Scholes model presents a mathematical model for the dynamics of an option's price. 
The model makes certain assumptions about the market. A complete list of all the
assumptions can be found in \cite{seydel_2009} and \cite{wilmott_howison_dewynne_1995}. 
In the next part, we enumerate some them. 
First, the 
asset price $S_t$ is distributed as a log-normally,

{
\color{red}
\begin{equation}
  S_t = S_0 \exp\bigg\{\int_{0}^{t} \big(r(s) - \dfrac{1}{2}\sigma(s)\big)ds + \sqrt{t}Z\bigg\}
\end{equation}
}

where the risk-free interest $r(t)$ and the price volatility $\sigma(t)$ are 
deterministic functions of time during the life of the option. Secondly, the 
bank account $B(t)$ is a deterministic function

\begin{equation}
  dB = r(t)B(t)dt
\end{equation}

Finally, the asset does not pay dividends. Additionally, we will assume that 
the risk-free interest rate and asset price volatility are constant during the 
life of the option. Later on, we will address the assumption about dividends.

By applying the Black-Scholes model to price European options, the famous Black-Scholes 
PDE is obtained

\begin{equation}
  \begin{cases}
    \dfrac{\partial{V}}{\partial{t}} + \dfrac{1}{2}\sigma^{2} S^2 \dfrac{\partial^2{V}}{\partial{S^2}} + r S \dfrac{\partial{V}}{\partial{S}} - rV = 0 & \text{for $t\in[0,T)$ and $S\in[0, \infty)$} \\
    V(S, T) = H(S, T) & \text{for $S\in[0, \infty)$}
  \end{cases}
  \label{eq:blackscholes:preliminaries:european_option_pde}
\end{equation}

where $V(S, t)$ is a deterministic function. A derivation of
\eqref{eq:blackscholes:preliminaries:european_option_pde} can be found in \cite{seydel_2009}
and \cite{wilmott_howison_dewynne_1995}.

We previously mention that the one of the assumptions of the Black-Scholes model is
that the underlying asset does not pay dividends. In most cases, assets such as stocks
pay out dividends just a few times at year. In this case, dividends are to be 
modelled discretely. However, there are certain assets that pay out a proportion
of the current price during and interval of time. Thus, in such cases, it is
useful to model dividends as a continuous yield. \cite{wilmott_howison_dewynne_1995}
shows as reasonable model of the asset price with volatility $\sigma(t)$, 
rate of return $r(t)$, and continuous dividend yield $\delta(S,t)$ paid at instant
of time $dt$ is modeled as
\begin{align}
  dS = (r(t) - \delta(S, t))Sdt + \sigma(t) S dW
  \label{eq:blackscholes:preliminaries:bs_price_model_with_dividends}
\end{align}

Similarly to
as we did for the risk-free interest rate and the volatility of the asset price,
we will assume that continuous dividends yield is as constant from now on.
In \cite{wilmott_howison_dewynne_1995}, it is shown by applying the Black-Scholes 
model under \eqref{eq:blackscholes:preliminaries:bs_price_model_with_dividends} 
to price European options, the slightly modified version of \eqref{eq:blackscholes:preliminaries:european_option_pde}
is obtained

\begin{equation}
  \begin{cases}
    \dfrac{\partial{V}}{\partial{t}} + \dfrac{1}{2}\sigma^{2} S^2 \dfrac{\partial^2{V}}{\partial{S^2}} + (r - \delta) S \dfrac{\partial{V}}{\partial{S}} - rV = 0 & \text{for $t\in[0,T)$ and $S\in[0, \infty)$} \\
    V(S, T) = H(S, T) & \text{for $S\in[0, \infty)$}
  \end{cases}
  \label{eq:chapter2:european_option_pde_with_dividens}
\end{equation}
 
Similarly, the Black-Scholes model is applied to price American options.
The work of \cite{wilmott_howison_dewynne_1995} derives some important facts 
about the value surface of American options. Firstly, the value function is 
bounded from below by the payoff function:
\begin{align}
V_{\text{Am}}(S, t) \ge H(S, t) \qquad \text{for $t \in [0, T]$}
\label{eq:blackscholes:american_options_price_lower_bound}
\end{align}

Moreover, the domain of $V(S, t)$ can be separated into the exercise region, 
where it is profitable for the holder to exercise the option,
\begin{equation}
\mathcal{S} := {(S, t) : V(S, t) = H(S, t)}
\label{eq:blackscholes:preliminaries:exercise_region}
\end{equation}

The continuation region, where it is preferable to continue holding the option 
because exercising is not profitable:
\begin{equation}
\mathcal{C} := {(S, t) : V(S, t) > H(S, t)}
\end{equation}

The boundary of the continuation region, where it is most optimal for the 
holder to exercise the option:
\begin{equation}
\partial \mathcal{C} := {(S, t) : S = \bar{S}(t)}
\label{eq:blackscholes:preliminaries:continuation_region}
\end{equation}

where $\bar{S}(t)$ is the optimal exercise price. Lastly, the price dynamics of 
American options behaves as European options within
the continuation region. Since we know $V(S, t)$ at the stopping region, we only
need to solve $V(S,t)$ at continuation region and determine its boundary $\partial\mathcal{C}$
at the same time. Therefore, this is known as the free boundary problem formulation
of the American option pricing problem and is equivalent to solve:
\begin{align}
  \begin{cases}
  \dfrac{\partial{V}}{\partial{t}} + \dfrac{1}{2}\sigma^{2} S^2 \dfrac{\partial^2{V}}{\partial{S^2}} + (r - \delta)S \dfrac{\partial{V}}{\partial{S}} - rV = 0 & \text{for $(S, t) \in \mathcal{C}$} \\
  V(S, t) = H(S, t) & \text{for $(S,t)\in \partial\mathcal{C}$}
  \end{cases}
  \label{eq:blackscholes:preliminaries:american_options_pde_free_boundary_problem}
\end{align}

at time $T$, the value of the option at exercise region will overlap with linear
segment of the payoff function $H(S, T)$, in that case, the optimal exercise 
$S(T)$ will be equal to the strike price $K$ payoff function. Therefore, 
we can use this information to define the terminal conditions of 
\eqref{eq:blackscholes:preliminaries:american_options_pde_free_boundary_problem}
\begin{align}
  V(S,T) = H(S, T) \quad \bar{S}(T) = K
  \label{eq:blackscholes:preliminaries:american_options_terminal_condition}
\end{align}

Next, we need to establish boundary conditions for the system 
\eqref{eq:blackscholes:preliminaries:american_options_pde_free_boundary_problem}. 
Generally, when pricing options, we would need two boundaries conditions 
for option pricing. However, the free boundary problem in 
\eqref{eq:blackscholes:preliminaries:american_options_pde_free_boundary_problem}
is expressed in terms on the moving boundary condition $\bar{S}(t)$. Therefore,
we only need to determine one extra boundary condition. In case of 
American put options, the left boundary condition would be given by $\bar{S}(t)$,
and the right boundary condition by $V(S, t) = 0$ for a sufficiently large $S$. 
Analogously, $\bar{S}(t)$ would be the right boundary condition for 
an American call option and its left boundary condition by $V(S, t) = 0$ for $S = 0$. 
Finally, $V(S,t)$ touches tangentially the exercise region $S$ at $(\bar{S}(t), t)$.
Since the exercise region is a linear segment, the derivative at that point is
\begin{subequations} \label{eq:blackscholes:preliminaries:smooth_passing_condition}
  \begin{equation}
    \dfrac{\partial{V}}{\partial{S}}(\bar{S}(t), t) = 1 
  \end{equation}
  \begin{equation}
    \dfrac{\partial{V}}{\partial{S}}(\bar{S}(t), t) = -1
  \end{equation}
\end{subequations}

which is called the smooth pasting condition in \cite{seydel_2009} and \cite{wilmott_howison_dewynne_1995}.
Grouping \eqref{eq:blackscholes:preliminaries:american_options_pde_free_boundary_problem}, \eqref{eq:blackscholes:preliminaries:american_options_terminal_condition}
and \eqref{eq:blackscholes:preliminaries:smooth_passing_condition} in one equation, we obtain the system.

\begin{subequations} \label{eq:blackscholes:preliminaries:american_options_pde_free_boundary_problem_full}
\begin{equation}
  \begin{cases}
    \dfrac{\partial{V}}{\partial{t}} + \dfrac{1}{2}\sigma^{2} S^2 \dfrac{\partial^2{V}}{\partial{S}^2} + (r - \delta)S\dfrac{\partial{V}}{\partial{S}} - rV = 0 & \text{for $0 \le S < \bar{S}(t)$ and $0 \le t < T$} \\
    V(S, T) = S - K \qquad \bar{S}(T) = K \\
    V(0, t) = 0 \quad \dfrac{\partial{V}}{\partial{S}}(\bar{S}(t), t) = 1
  \end{cases}
\end{equation}
\begin{equation}
  \begin{cases}
    \dfrac{\partial{V}}{\partial{t}} + \dfrac{1}{2}\sigma^{2} S^2 \dfrac{\partial^2{V}}{\partial{S}^2} + (r - \delta)S\dfrac{\partial{V}}{\partial{S}} - rV = 0 & \text{for $\bar{S}(t) < S < \infty$ and $0 \le t < T$} \\
    V(S, T) = K - S \qquad \bar{S}(T) = K \\
    \lim_{S\rightarrow\infty}V(S, t) = 0 \quad \dfrac{\partial{V}}{\partial{S}}(\bar{S}(t), t) = -1
  \end{cases}
\end{equation}

\end{subequations}

\subsection{Front-Fixing method}

In the previous section, we presented the pricing of American options problem.
By applying the Black-Scholes model, we derived the Black-Scholes PDE that describes 
the price dynamics in the continuation region $\mathcal{C}$ of call and put options.
Moreover, we presented the moving boundary condition $\bar{S}(t)$ for this PDE.
The moving boundary condition $\bar{S}(t)$ makes the Black-Scholes PDE more 
involved since we also need to determine this boundary as time changes. This type 
of problem are known as free boundary problems. The front fixing method
was first introduced by \cite{landau_1950_heat_ci} and is a strategy in which 
we define a map from the original domain to new domain where moving boundary
remains constant as time changes. In this section, we explore two transformation
based on the work of Nielsen and others \cite{nielsen_2001}, and the work of
Company and others \cite{company_egorova_jodar_2014}.

\subsubsection{Inverse transformation} \label{sec:blackscholes:frontfixingmethod:inversetransform}

This method proposes the transformation 
\begin{equation}
    x = \dfrac{S}{\bar{S}(t)}
    \label{eq:blackscholes:frontfixingmethod:inversetransform}
\end{equation}

which maps the boundary of the continuation region $\partial C$ defined in 
\eqref{eq:blackscholes:preliminaries:exercise_region} to the fixed boundary 
\begin{equation}
  \mathcal{\partial C}_x := \{(x, t): x = 1\} 
\end{equation}

which remain constant as t changes. Now, let us define the value $v(x,t)$ 
under this new map
\begin{equation}
  v(x, t) := V(S, t)
  \label{eq:blackscholes:frontfixingmethod:inversetransform:value_function}
\end{equation}

which fixes the moving boundary $\bar{S}(t)$ at $x=1$ when $S(t)$. Next, we compute
the partial derivatives of $V$ with respect of the partial derivatives of $v$ which
will allow us to rewrite the PDE in \eqref{eq:blackscholes:preliminaries:american_options_pde_free_boundary_problem_full} 
with respect of \eqref{eq:blackscholes:frontfixingmethod:inversetransform:value_function}

\begin{align*}
  \dfrac{\partial{x}}{\partial{S}} &= \dfrac{1}{\bar{S}(t)} \\\\
  \dfrac{\partial{x}}{\partial{t}} &= -x\dfrac{\bar{S}'(t)}{\bar{S}(t)} \\\\
  \dfrac{\partial{V}}{\partial{S}} &= \dfrac{\partial{v}}{\partial{x}}\dfrac{\partial{x}}{\partial{S}} = \dfrac{1}{\bar{S}(t)}\dfrac{\partial{v}}{\partial{x}} \\\\
  \dfrac{\partial^2{V}}{\partial{S^2}} &= \dfrac{1}{\bar{S}(t)} \dfrac{\partial{x}}{\partial{S}} \dfrac{\partial^2{v}}{\partial{x^2}} = \dfrac{1}{\bar{S}(t)^2} \dfrac{\partial^2{v}}{\partial{x}^2} \\\\
  \dfrac{\partial{V}}{\partial{t}} &=  \dfrac{\partial{v}}{\partial{t}} + \dfrac{\partial{v}}{\partial{x}} \dfrac{\partial{x}}{\partial{t}} = \dfrac{\partial{v}}{\partial{t}} - x\dfrac{\bar{S}^\prime(t)}{\bar{S}(t)}\dfrac{\partial{v}}{\partial{x}}
\end{align*}

Substituting these partial derivatives in the Black-Scholes PDE given by \eqref{eq:blackscholes:preliminaries:american_options_pde_free_boundary_problem_full},
we obtain the non-linear PDE
\begin{subequations} \label{eq:blackscholes:frontfixingmethod:american_options_pde}
  \begin{align}  
    \dfrac{\partial{v}}{\partial{t}} + \dfrac{1}{2}\sigma^{2} x^2 \dfrac{\partial^2{v}}{\partial{x}^2} + \bigg[(r - \delta) - \dfrac{\bar{S}^\prime(t)}{\bar{S}(t)}\bigg]x\dfrac{\partial{v}}{\partial{x}} - rv = 0 \quad & \text{for $x \in [0, 1)$ and $t \in [0, T)$}\\
    \dfrac{\partial{v}}{\partial{t}} + \dfrac{1}{2}\sigma^{2} x^2 \dfrac{\partial^2{v}}{\partial{x}^2} + \bigg[(r - \delta) - \dfrac{\bar{S}^\prime(t)}{\bar{S}(t)}\bigg]x\dfrac{\partial{v}}{\partial{x}} - rv = 0 \quad & \text{for $x > 1$ and $t \in (0, T]$}
  \end{align}  
\end{subequations}

Likewise, we rewrite the terminal condition as
\begin{subequations} \label{eq:blackscholes:frontfixingmethod:inversetransform:american_options_terminal_condition}
  \begin{equation}
    v(x, T) = \max(x\bar{S}(T) - K) = K \max(x - 1, 0) = 0
  \end{equation}
  \begin{equation}
    v(x, T) = \max(K - x\bar{S}(T)) = K \max(1 - x, 0) = 0
  \end{equation}
\end{subequations}

Note that $x$ is always less than one for put options. Analogously, $x$ is 
always greater than one for call options. Finally, we rewrite the boundary condition
given by the optimal exercise price $\bar{S}(t)$ as
\begin{subequations} \label{eq:blackscholes:frontfixingmethod:inversetransform:american_options_optimal_price_bc}
  \begin{equation}
    \dfrac{\partial v}{\partial x}(1, t) = 1
  \end{equation}  
  \begin{equation}
    \dfrac{\partial v}{\partial x}(1, t) = -1
  \end{equation}
\end{subequations}

and the boundary condition opposite to the optimal exercise price as  
\begin{equation} \label{eq:blackscholes:frontfixingmethod:inversetransform:american_option_opposite_bc}
  v(0, t) = 0
\end{equation} 

for both put and call options. In summary, by groping equations 
\eqref{eq:blackscholes:frontfixingmethod:american_options_pde},
\eqref{eq:blackscholes:frontfixingmethod:inversetransform:american_options_terminal_condition},
\eqref{eq:blackscholes:frontfixingmethod:inversetransform:american_options_optimal_price_bc},
and \eqref{eq:blackscholes:frontfixingmethod:inversetransform:american_options_optimal_price_bc},
we obtain the system
\begin{subequations} \label{eq:blackscholes:frontfixingmethod:inversetransform:american_options_bs_pde}
\begin{align}
  \begin{cases}
    \dfrac{\partial{v}}{\partial{t}} + \dfrac{1}{2}\sigma^{2} x^2 \dfrac{\partial^2{v}}{\partial{x}^2} + \bigg[(r - \delta) - \dfrac{\bar{S}^\prime(t)}{\bar{S}(t)}\bigg]x\dfrac{\partial{v}}{\partial{x}} - rv = 0 & \text{for $x \in (0, 1)$ and $t \in [0, T)$} \\
    v(x, T) = 0 \qquad \bar{S}(T) = K & \text{for $x\in[0, 1]$}\\
    v(0, t) = 0 \quad v(1, t) = \bar{S}(t) - K \quad \dfrac{\partial{v}}{\partial{x}}(1, t) = \bar{S}(t) & \text{for $t\in[0, T)$} \\
  \end{cases}
\end{align}
\begin{align}
  \begin{cases}
    \dfrac{\partial{v}}{\partial{t}} + \dfrac{1}{2}\sigma^{2} x^2 \dfrac{\partial^2{v}}{\partial{x}^2} + \bigg[(r - \delta) - \dfrac{\bar{S}^\prime(t)}{\bar{S}(t)}\bigg]x\dfrac{\partial{v}}{\partial{x}} - rv = 0 & \text{for $x > 1$ and $t \in [0, T)$} \\
    v(x, T) = 0 \qquad \bar{S}(T) = K & \text{for $x \ge 1$} \\
    v(1, t) = K - \bar{S}(t) \quad \lim_{x\rightarrow\infty}v(x, t) = 0 \quad \dfrac{\partial{v}}{\partial{x}}(1, t) = -\bar{S}(t) & \text{for $t\in[0, T)$}
  \end{cases}
\end{align}
\end{subequations}

\subsubsection{Log transformation}

In this section, we define the transformations
\begin{equation}
  x := \log \dfrac{KS}{\bar{S}(t)} \qquad v(x, t) := \dfrac{V(S, t)}{K}
\end{equation}

which maps the boundary $\partial \mathcal{C}$ to the region
\begin{equation}
  \partial{\mathcal{C}_x} := \{ (x, t): x = \log{K} \}  
\end{equation}

\newpage

Similarly to the previous section, we compute the partial derivatives of $v(x,t)$,
\begin{align*}
  \dfrac{\partial x}{\partial t} &= -\dfrac{\bar{S}'(t) }{\bar{S}(t)} \\\\
  \dfrac{\partial x}{\partial S} &= \dfrac{1}{S} \\\\ 
  \dfrac{\partial V}{\partial S} &= \dfrac{K}{S} \dfrac{\partial v}{\partial x} \\\\
  \dfrac{\partial^2 V}{\partial^2 S} &= \dfrac{K}{S^2}\dfrac{\partial^2 v}{\partial x^2} - \dfrac{K}{S^2} \dfrac{\partial v}{\partial x} \\\\
  \dfrac{\partial V}{\partial t} &= K \dfrac{\partial v}{\partial t} - K\dfrac{\bar{S}'(t)}{\bar{S}(t)}\dfrac{\partial v}{\partial x}
\end{align*}

Using the partial derivates, we rewrite the Black-Scholes PDE as follows
\begin{subequations}
  \begin{align}
      \dfrac{\partial v}{\partial t} + \dfrac{1}{2}\sigma^2\dfrac{\partial^2 v}{\partial x^2} + \bigg((r-\delta) - \dfrac{\sigma^2}{2} \bigg)\dfrac{\partial v}{\partial x} -\dfrac{\bar{S}'(t)}{\bar{S}(t)}\dfrac{\partial v}{\partial x} - rv = 0 \quad \text{for $x < \log{K}$ and $t \in [0, T)$} \\
      \dfrac{\partial v}{\partial t} + \dfrac{1}{2}\sigma^2\dfrac{\partial^2 v}{\partial x^2} + \bigg((r-\delta) - \dfrac{\sigma^2}{2} \bigg)\dfrac{\partial v}{\partial x} -\dfrac{\bar{S}'(t)}{\bar{S}(t)}\dfrac{\partial v}{\partial x} - rv = 0 \quad \text{for $x > \log{K}$ and $t \in [0, T)$}
  \end{align}
\end{subequations}

with terminal condition
\begin{subequations} \label{eq:blackscholes:frontfixingmethod:logtransform:american_options_terminal_condition}
  \begin{equation}
    v(x, T) = \max\bigg(\dfrac{e^x}{K} - 1, 0\bigg) = 0
  \end{equation}
  \begin{equation}
    v(x, T) = \max\bigg(1 - e^x, 0\bigg) = 0
  \end{equation}
\end{subequations}

\newpage

Next, we rewrite the boundary condition given by the optimal exercise price
\begin{subequations}
  \begin{equation}
    \dfrac{\partial v}{\partial x}(\log{K}, t) = \dfrac{\bar{S}(t)}{K} 
  \end{equation}
  \begin{equation}
    \dfrac{\partial v}{\partial x}(\log{K}, t) = -\dfrac{\bar{S}(t)}{K}
  \end{equation}
\end{subequations}

and the boundary condition opposite to the optimal exercise price
\begin{subequations}
  \begin{equation}
    \lim_{x\rightarrow-\infty} v(x, t) = 0
  \end{equation}
  \begin{equation}
    \lim_{x\rightarrow\infty} v(x, t) = 0
  \end{equation}
\end{subequations}

Finally, grouping equations together, we have the system
\begin{subequations} \label{eq:blackscholes:frontfixingmethod:logtransform:american_options_bs_pde}
  \begin{equation}
    \begin{cases}
      \dfrac{\partial v}{\partial t} + \dfrac{1}{2}\sigma^2\dfrac{\partial^2 v}{\partial x^2} + \bigg((r-\delta) - \dfrac{\sigma^2}{2} \bigg)\dfrac{\partial v}{\partial x} -\dfrac{\bar{S}'(t)}{\bar{S}(t)}\dfrac{\partial v}{\partial x} - rv = 0 \quad \text{for $x < \log{K}$ and $t \in [0, T)$} \\
      v(x, T) = 0 \qquad \bar{S}(T) = K \\
      \lim_{x\rightarrow-\infty} v(x, t) = 0 \quad \dfrac{\partial{v}}{\partial{x}}(\log{K}, t) = \dfrac{\bar{S}(t)}{K}
    \end{cases}
  \end{equation}
  \begin{equation}
    \begin{cases}
      \dfrac{\partial v}{\partial t} + \dfrac{1}{2}\sigma^2\dfrac{\partial^2 v}{\partial x^2} + \bigg((r-\delta) - \dfrac{\sigma^2}{2} \bigg)\dfrac{\partial v}{\partial x} -\dfrac{\bar{S}'(t)}{\bar{S}(t)}\dfrac{\partial v}{\partial x} - rv = 0 \quad \text{for $x > \log{K}$ and $t \in [0, T)$} \\
      v(x, T) = 0 \qquad \bar{S}(T) = K \\
      \lim_{x\rightarrow\infty} v(x, t) = 0 \quad \dfrac{\partial{v}}{\partial{x}}(\log{K}, t) = -\dfrac{\bar{S}(t)}{K}
    \end{cases}
  \end{equation}
\end{subequations}