
\section{Background} \label{sec:_background}

\subsection{Finance}

A common problem in finance is to price financial derivatives, often referred just
as derivative. In essence, a derivatives are contracts set between parties 
whose value in time derives from the price of their underlying assets. A notorious
family of derivatives in financial markets are options. Options are contracts set 
between two parties in which the buyer of option has the right to sell or buy, 
commonly referred as, exercise, an stock at a prestablished price, also known as, 
strike price, in the future. An option is referred as a call option or as a put 
option if the exercise position is to buy or to sell respectevely. Similarly, 
options are classified depending on their exercise style. In that regard,
the simplest of options are European options. European options gives buyer of 
the option can only exercise the option at given point in time in the future, 
often referred as the maturity or expiration date. The payoff of an European 
option is expressed using the following formula

\begin{align}
  P = \max(S(T) - K, 0) 
  \label{eq:section2:subsection1:payoff_call_european_option}\\
  P = \max(K - S(T), 0)
  \label{eq:section2:subsection1:payoff_put_european_option} 
\end{align}

where $T$ is the time elapsed between the starting and expiration date of the 
contract, $S(T)$ is the stock price at the expiration time, and $K$ is the strike 
price. Note that formula (\ref*{eq:section2:subsection1:payoff_call_european_option})
and (\ref*{eq:section2:subsection1:payoff_put_european_option}) are for an call and a 
put option respectively. Another well known option are the American options. American options
gives the right to exercise the option at any point in time between the beginning 
and expiration date of the contract. The payoff of an American option is

\begin{align}
  P(t) = \max(S(t) - K, 0) 
  \label{eq:section2:subsection1:payoff_call_european_option}\\
  P(t) = \max(K - S(t), 0)
  \label{eq:section2:subsection1:payoff_put_european_option} 
\end{align}

where $t$ is the time when the buyer decides to exercise the option. 
The value of both options $V(t)$ at the maturity time $T$ is 
equal to the payoff

\begin{align}
  V(T) = P
\end{align}

Obviously, options give greater flexibility 
to buyers because is removing their exposure of having a negative value.
It is because of this that the writer of the option charges a premium $V(0)$ to the buyer
at the time they enter the contract. The premium is often referred as the price or value 
of the option and the problem of finding this value is called option pricing. 
When pricing options, it is important to find the just price because 
otherwise the writer or buyer of the option could set some scheme in which option
will always be profitable to them. In other words, options pricing must follow
the principle of no-arbitrage. Therefore, we assume that the writer of the option
uses the premium to construct a portfolio consisting of $\phi_0$ units of the 
stock and invests $\psi_0$ units of cash into a risk-free asset such as US
tresury bill, certificate of deposit, or bank account. Then, the writer only 
uses wealth of the portfolio $(\phi_0, \psi_0)$ to hedge any 
possible claims from the buyer of the option at any future time $0 \le t \le T$. 
Therefore, at any time $t$, the writer holds a portfolio $(\phi(t), \psi(t))$ 
with value

\begin{equation}
  \Pi(t) = \phi(t)S(t) + \psi(t)B(t)
\end{equation}

Moreover, the portfolio is self-financing. In other words, the changes in the value
of the portfolio $V(t)$ depend on the changes in $S(t)$ and $B(t)$, and the current
portfolio $(\phi(t), \psi(t))$

\begin{equation}
  d\Pi(t) = \phi(t)dS(t) + \psi(t)dB(t)
\end{equation}

Finally, the value of an option must satisfy the following

\begin{equation}
  \Pi(t) = V(t)
\end{equation}

at any time $0 \le t \le T$.

The black schole model is built upon the self-financing portfolio hedging strategy 
and expresses a mathematical model for dynamics of option's price. 
The black schole model makes with some assumptions about the market. For complete
list of these assumptions look at (reference). We enumerates the one we believe 
are important for our task. First, the interest rate of the risk-free asset is 
a deterministic function of time
\begin{equation}
  dB = r(t)B(t)dt
\end{equation}


By applying the black schole model to European option, the famous black-schole 
PDE is obtained:

\begin{equation}
  \begin{cases}
    \frac{\partial{V}}{\partial{t}} + \frac{1}{2}\sigma^{2} S^2 \frac{\partial^2{V}}{\partial{S^2}} + r S \frac{\partial{V}}{\partial{S}} - rV = 0 \\
    V(S, T) = P
  \end{cases}
\end{equation}

Where $V(S,t)$ is a smooth and deterministic function of $S$ and $t$.

\begin{align}
  \begin{cases}
    \frac{\partial{V}}{\partial{t}} + \frac{1}{2}\sigma^{2} S^2 \frac{\partial^2{V}}{\partial{S^2}} + (r - \delta) S \frac{\partial{V}}{\partial{S}} - rV = 0 & \text{for $S > \bar{S}(t)$ and $0 \le t < T$} \\
    V(S, t) = K - S & \text{for $0 \le S \le \bar{S}(t)$ and $0 \le t < T$} \\
    V(S, T) = \max(K - S, 0) & \text{for $S \ge 0$} \\
    \frac{\partial{V}}{\partial{S}}(\bar{S}(t), t) = -1 \\
    \lim_{S\rightarrow \infty} V(S, t) = 0 \\
    \bar{S}(T) = K
  \end{cases}
  \label{eq:background:finance:free_boundary_problem}
\end{align}