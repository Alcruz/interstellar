
\section{Background} \label{sec:_background}

\subsection{Finance}

A common problem in finance is to price financial derivatives, often referred just
as derivative. In essence, a derivatives are contracts set between parties 
whose value in time derives from the price of their underlying assets. A notorious
family of derivatives in financial markets are options. Options are contracts set 
between two parties in which the buyer of option has the right to sell or buy, 
commonly referred as, exercise, an stock at a prestablished price, also known as, 
strike price, in the future. An option is referred as a call option or as a put 
option if the exercise position is to buy or to sell respectevely. Similarly, 
options are classified depending on their exercise style. In that regard,
the simplest of options are European options. European options gives buyer of 
the option can only exercise the option at given point in time in the future, 
often referred as the maturity or expiration date. The payoff of an European 
option is expressed using the following formula

\begin{equation}
  P = \max(S_T - K, 0)
  \label{eq:section2:subsection1:payoff_call_european_option}
\end{equation}

\begin{equation}
  P = \max(K - S_T, 0)
  \label{eq:section2:subsection1:payoff_put_european_option}
\end{equation}

where $T$ is the time elapsed between the starting and expiration date of the 
contract, $S_T$ is the stock price at the expiration time, and $K$ is the strike 
price. Note that formula (\ref*{eq:section2:subsection1:payoff_call_european_option})
and (\ref*{eq:section2:subsection1:payoff_put_european_option}) are for an call and a 
put option respectevely. Obviously, options give greater flexibility 
to the buyers because is removing their exposure of having a negative payoff.
It is because the writers of options charge premiums to the buyers at the time
they enter the contracts. The premium is often referred as the price or value 
of the option and the problem of finding this value is called options pricing or
valuing. When pricing options, it is important to find the just price because 
otherwise the writer or buyer of the option could set some scheme in which option
will always be profitable to them. In other words, options pricing must follow
the principle of no-arbitrage. Suppose that there exists a risk-free financial
instruments with a continous rate of return $r$. This financial instrument 
could be a US tresury bill or a certificate of deposit. By principle of non-arbitrage,
the price of an option must by 

That is we want to gi

