\section{Linear complementary problem}


Now, note that the bound of $V(S, t)$ in each region is
\begin{align*}
  &V(S, t) - H(S, t) > 0 \qquad \text{for all $(S,t) \in \mathcal{C}$} \\ 
  &V(S, t) - H(S, t) = 0 \qquad \text{for all $(S,t) \in \mathcal{S}$}
\end{align*}

Similarly, the bound of $\mathcal{L}_{\text{BS}}(V)$ is
\begin{align*}
  &\frac{\partial{V}}{\partial{t}} + \frac{1}{2}\sigma^{2} S^2 \frac{\partial^2{V}}{\partial{S^2}} + (r - \delta)S \frac{\partial{V}}{\partial{S}} - rV = 0 \qquad \text{for $(S,t) \in \mathcal{C}$} \\
  &\frac{\partial{V}}{\partial{t}} + \frac{1}{2}\sigma^{2} S^2 \frac{\partial^2{V}}{\partial{S^2}} + (r - \delta)S \frac{\partial{V}}{\partial{S}} - rV < 0 \qquad \text{for $(S,t) \in \mathcal{S}$}
\end{align*}

Therefore, grouping the bounds above we form a linear complementary system of equations
{
  \color{red}  
  \begin{align}
    \begin{cases}
      \big[\frac{\partial V}{\partial t} - \mathcal{L}_{\text{BS}}(V)\big] \cdot [V(S,t) - H(S,t)] = 0 & \text{for all $(S,t)$} \\
      V(S, t) - H(S, t) \ge 0 & \text{for all $(S, t)$}\\
      \frac{\partial V}{\partial t} - \mathcal{L}_{\text{BS}}(V) \le 0 &  \text{for all $(S, t)$}\\
      V(S, T) = H(S, T) \\  
    \end{cases}
    \label{eq:background:finance:linear_complementary_problem}
  \end{align}
}

The benefit of the reformulation (\ref*{eq:background:finance:linear_complementary_problem})
is that there is no dependence on the unknown boundary of the continuation region.
Later in section (XXX) and section (XXX), we will explore numerical methods for solving 
both type of problems.


\subsubsection{Theta method}

We discretize the system of equation containing (3.17) and (3.18) to solve it. 
Firstly, we define an uniform meshgrid within the region $[1, x_\text{max}]\times [0, T]$ and 
with resolution $\Delta x$ and $\Delta t$.

\begin{align*}
    M := \dfrac{x_{\text{max}} - 1}{\Delta x}
    \qquad
    N := \dfrac{T}{\Delta t}
\end{align*}

\begin{align*}
   & & x_i &:= 1 + i{\Delta x} &\text{for $i = 0,\dots,M$} & & \\ 
   & & t_n &:= n{\Delta t} &\text{for $n = 0,\dots,N$} & &
\end{align*}

Now we define the approximations

\begin{equation}
    v^{n}_{i} \approx v(x_{i}, t_{n}) \qquad \text{for $(x_{i}, t_{n}) \in \{x_k\}^{M}_{0} \times \{t_k\}^{N}_{0}$} 
\end{equation}
\begin{equation}
    \bar{S}^{n} \approx \bar{S}(t_{n}) \qquad \text{for $t_{n} \in \{t_k\}^{N}_{0}$} 
\end{equation}

By the boundary conditions, we can derive an expression for

\begin{align}
    v^{n}_{0} &= K - \bar{S}^{n} \qquad \text{for $n = 0, \dots, N - 1, N$} \\
    v^{n}_{M+1} &= 0 \qquad \text{for $n = 0, \dots, N - 1, N$}
\end{align}

Additionally by using the smothness condition, we get:

\begin{align}
    \dfrac{v^{n}_{1} - v^{n}_0}{\Delta x} = -\bar{S}^{n}
\end{align}

or 

\begin{align}
    v^{n}_{1}= K - (1 + {\Delta x})\bar{S}^{n}
\end{align}

Next, we equation (3.XX) discretize using centered finite difference. The discretization
method, we use is the theta method which a interpolation between an implicit and explicit scheme.

\begin{equation}
    \begin{split}
        v^{t+1}_{i} &- v^{t}_{i} + \theta \bigg\{ \dfrac{1}{2}\sigma^2 x^{2}_{i} \dfrac{\Delta t}{(\Delta x)^2} (v^{t}_{i-1} - 2 v^{t}_{i} + v^{t}_{i+1}) + \bigg[ (r - \delta) - \dfrac{1}{\bar{S}^t} \dfrac{\bar{S}^{t+1} - \bar{S}^{t}}{\Delta t} \bigg] \dfrac{\Delta t}{2\Delta x} (v_{i+1}^{t} - v_{i-1}^{t}) - r v^{t}_{i} \Delta t \bigg\}
        \\ & + (1-\theta) \bigg\{ \dfrac{1}{2}\sigma^2 x^{2}_{i} \dfrac{\Delta t}{(\Delta x)^2}(v^{t+1}_{i-1} - 2 v^{t+1}_{i} + v^{t+1}_{i+1}) 
        \\ &  + \bigg[ (r - \delta) - \dfrac{1}{\bar{S}^{t+1}} \dfrac{\bar{S}^{t+1} - \bar{S}^{t}}{\Delta t} \bigg]\dfrac{\Delta t}{2\Delta x}(v_{i+1}^{t+1} - v_{i-1}^{t+1}) - r v^{t+1}_{i} \Delta t \bigg\} = 0
    \end{split}
\end{equation}

To simplify the expression above, we introduce the following terms

\begin{equation}
    \lambda := \dfrac{\Delta t}{(\Delta x)^2}
\end{equation}

\begin{align}
    \alpha_i &:= 1 + \theta (\lambda \sigma^2 x_{i}^{2} + r{\Delta t}) \\
    \beta_i &:= - \dfrac{1}{2} \lambda \theta \bigg[ \sigma^{2} x_{i}^{2} - x_i \Delta x (r - \delta) \bigg]  - \dfrac{1}{2} \lambda \theta   x_i \Delta x \dfrac{\bar{S}^{n+1} - \bar{S}^{n}}{\Delta t \bar{S}^n}  \\
    \gamma_i &:= -\dfrac{1}{2} \lambda \theta \bigg[ \sigma^{2} x_{i}^{2} + x_i \Delta x (r - \delta) \bigg]  + \dfrac{1}{2} \lambda \theta  x_i \Delta x  \dfrac{\bar{S}^{n+1} - \bar{S}^{n}}{\Delta t \bar{S}^n}\\
    a_i &:= 1 - (1-\theta) (\lambda \sigma^2 x_{i}^{2} +  r{\Delta t}) \\
    b_i &:= \dfrac{1}{2} (1-\theta) \lambda \bigg[\sigma^{2} x_{i}^{2} - x_i \Delta x \bigg( (r - \delta) - \dfrac{1}{\Delta t} \bigg) \bigg] \\
    c_i &:= \dfrac{1}{2} (1-\theta) \lambda \bigg[ \sigma^{2} x_{i}^{2} +  x_i \Delta x \bigg( (r - \delta) - \dfrac{1}{\Delta t} \bigg) \bigg] \\
    d^{n+1}_i &:= (1-\theta) \dfrac{x_i}{2 \Delta x}  \dfrac{v^{n+1}_{i+1} - v^{n+1}_{i-1}}{\bar{S}^{n+1}}
\end{align}

Now, the expression above becomes

\begin{equation}
    \beta^{n}_{i} v^{n}_{i-1} + \alpha^{n}_{i} v^{n}_{i} + \gamma^{n}_{i} v^{n}_{i+1} = b_i v^{n+1}_{i-1} + a_i v^{n+1}_{i} + c_i v^{n+1}_{i+1} + d^{n+1}_{i}\bar{S}^{n}
\end{equation}

\begin{align}
    \gamma^{n}_{1} v^{n}_{2} = b_1 v^{n+1}_{0} + a_1 v^{n+1}_{1} + c_1 v^{n+1}_{2} + d^{n+1}_{1}\bar{S}^{n} - \beta^{n}_{1} (K - \bar{S}^n) - \alpha^{n}_{1} (K - (1+\Delta x)\bar{S}^n) 
\end{align}

\begin{equation}
    \alpha^{n}_{2} v^{n}_{2} + \gamma^{n}_{2} v^{n}_{3} = b_2 v^{n+1}_{1} + a_2 v^{n+1}_{2} + c_2 v^{n+1}_{3} + d^{n+1}_{2}\bar{S}^{n} - \beta^{n}_{2} (K - (1+\Delta x)\bar{S}^n) 
\end{equation}

\begin{equation}
    \beta^{n}_{M} v^{n}_{M-1} + \alpha^{n}_{M} v^{n}_{M} = b_i v^{n+1}_{i-1} + a_i v^{n+1}_{i} + c_i v^{n+1}_{i+1} + d^{n+1}_{i}\bar{S}^{n}
\end{equation}
