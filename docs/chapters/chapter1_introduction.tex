\section{Introduction}

\subsection{Overview}

Options are equity-based derivatives that are primarily used to mitigate risk. The options market is significantly larger compared to other derivatives. In fact, options were the most traded derivatives in 2019, with a volume of 18.55 billion contracts when combining index and individual equities contracts \cite{statista_2019}. An enormous market like the options market demands reliable pricing mechanisms to minimize arbitrage opportunities. Naturally, pricing American options is a broad research area because there is no closed-form solution to the PDE resulting from the Black-Scholes model. Merton\cite{merton_1973} was the first to consider the Black-Scholes\cite{black_scholes_1973} model to price European and American options. Although Merton derived a nice formula to price European options, he stated that, in general, a closed-form solution was not attainable for American options. In 1977, Schwartz \cite{schwartz_197779} and Brennan \cite{brennan_1997} proposed using finite difference schemes to solve the pricing problem for American options. The work of Merton and Schwartz served as a foundation for the free boundary problem formulation of the pricing problem. The motivation behind the free boundary problem formulation is that for American options, there exists an optimal exercise price that marks the boundary between the region where exercising the option is profitable and the region where it is not. Moreover, this optimal exercise price changes with time, making it impossible for the holder to determine when to exercise. Based on the work of Landau \cite{landau_1950_heat_ci}, Wu et al. \cite{wu1997front} formulated the front-fixing method as an approach to solve the free boundary problem for options, in which the Landau transformation is used to transform the moving boundary into a fixed boundary. Multiple transformations have been proposed by Huang et al. \cite{huang_2000}, Nielsen et al. \cite{nielsen_2001}, and Company et al. \cite{company_egorova_jodar_2014}. Around the same period, Dewynne et al. \cite{dewynne_howison_rupf_wilmott_1993} took a different approach to solve the pricing problem. The idea was to reformulate the free boundary problem as a problem with variational inequalities that, when finite difference is applied, transforms into a linear complementary problem \cite{dantzig_1968}.

\subsection{Aim} 

The goal of this work is to implement the numerical methods proposed by Nielsen et al. \cite{nielsen_2001} and Company et al.\cite{company_egorova_jodar_2014} to solve the free boundary formulation of the pricing problem for American options \cite{dewynne_howison_rupf_wilmott_1993}. Moreover, Company et al. \cite{company_egorova_jodar_2014} and Nielsen et al.\cite{nielsen_2001} proposed schemes for pricing American put options with underlying assets that have non-paying dividends. However, we aim to derive analogous schemes for pricing call contracts, and consider underlying assets that have a continuous dividend yield, such as the S\&P500.  Furthermore, we consider the PSOR method proposed by Dewynne\cite{dewynne_howison_rupf_wilmott_1993}\cite{wilmott_howison_dewynne_1995} to solve for the variational inequalities formulation of the pricing problem. Finally, we conduct a convergence analysis for each the schemes derived.

\subsection{Main Achievements}

In this work, we were able to derive the corresponding PDE problem resulting from applying the transformations proposed by Nielsen et al.\cite{nielsen_2001} and Company et al.\cite{company_egorova_jodar_2014} for call and put options with underlying assets featuring a continuous dividend yield. Moreover, we implemented explicit and implicit front fixing schemes for the Nielsen transformation method \cite{nielsen_2001}, an explicit front fixing scheme for Company transformation, and the theta PSOR scheme for the linear complementary problem proposed by Dewynne\cite{wilmott_howison_dewynne_1995}. Finally, we derived the order of convergence for each of the implemented methods.

\subsection{Outline}

The outline of this paper is as follows. In section 2, we explore the Black-Scholes model for American options for assets that pay dividends, resulting in the free boundary formulation of the pricing problem. Furthermore, we delve into the front-fixing method as a strategy for fixing the moving boundary by applying the changes of variables proposed by Company et al. \cite{company_egorova_jodar_2014} and Nielsen et al. \cite{nielsen_2001}. In section 3, we explore explicit and implicit schemes to solve the partial differential equations resulting from applying the front-fixing method and the Nielsen transformation to the free boundary problem obtained in section 1. We conclude section 3 with numerical experiments and convergence analysis for the numerical schemes presented in section 3, and appendix \ref{sec:company_explicit_scheme}. In section 4, we explore a reformulation of the pricing problem as a variational inequality. Additionally, we derived the PSOR method as way to solve the linear complementary problem that arises from applying the theta method to the variational inequality. Finally, in the same section, we discuss the results obtained by explicit, implicit and Crank-Nicholson PSOR schemes.
