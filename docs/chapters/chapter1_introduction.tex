\section{Introduction}

\subsection{Overview}

Options are equity-based derivatives that are primarily used to mitigate risk. The options market is significantly larger compared to other derivatives. In fact, options were the most traded derivatives in 2019, with a volume of 18.55 billion contracts when combining index and individual equities contracts \cite{statista_2019}. An enormous market like the options market demands reliable pricing mechanisms to minimize arbitrage opportunities. Naturally, pricing American options is a broad research area because there is no closed-form solution to the PDE resulting from the Black-Scholes model. Merton \cite{merton_1973} was the first to consider the Black-Scholes model proposed by Black and Scholes \cite{black_scholes_1973} to price European and American options. Although Merton derived a nice formula to price European options, he stated that, in general, a closed-form solution was not attainable for American options. In 1977, Schwartz \cite{schwartz_197779} and Brennan \cite{brennan_1997} proposed using finite difference schemes to solve the pricing problem for American options. The work of Merton and Schwartz served as a foundation for the free boundary problem for American options pricing. The motivation behind the free boundary problem formulation is that for American options, there exists an optimal exercise price that marks the boundary between the region where exercising the option is profitable and the region where it is not. Moreover, this optimal exercise price changes with time, making it impossible for the holder to determine when to exercise. Based on the work of Landau \cite{landau_1950_heat_ci}, Wu et al. \cite{wu1997front} formulated the front-fixing method as an approach to solve the free boundary problem for options, in which the Landau transformation is used to transform the optimal exercise price or the moving boundary into a fixed boundary. Multiple transformations have been proposed by Huang et al. \cite{huang_2000}, Nielsen et al. \cite{nielsen_2001}, and Company et al. \cite{company_egorova_jodar_2014}. Around the same period, Dewynne et al. \cite{dewynne_howison_rupf_wilmott_1993} took a different approach to solve the pricing problem. The idea was to reformulate the free boundary problem as a problem with variational inequalities that, when finite difference is applied, transforms into a linear complementary problem \cite{dantzig_1968}.

\subsection{Aim} 

The goal of this work is to implement the numerical methods proposed by Nielsen et al. \cite{nielsen_2001} and Company et al. \cite{company_egorova_jodar_2014} to solve the free boundary formulation of the pricing problem for American options \cite{dewynne_howison_rupf_wilmott_1993}. Moreover, Company et al. \cite{company_egorova_jodar_2014} and Nielsen et al. \cite{nielsen_2001} proposed schemes for pricing American put options with underlying assets that have non-paying dividends. However, we aim to derive analogous schemes for pricing call contracts as well, considering underlying assets that have a continuous dividend yield, such as the S\&P500. Additionally, we want to conduct a convergence analysis of each of the methods. Furthermore, we consider the linear complementary formulation proposed by Dewynne \cite{dewynne_howison_rupf_wilmott_1993} \cite{wilmott_howison_dewynne_1995}. Finally, we perform a convergence analysis of each of the methods implemented.

\subsection{Main Achievements}

In this work, we were able to derive the corresponding PDE problem resulting from applying the transformations proposed by Nielsen et al. \cite{nielsen_2001} and Company et al. \cite{company_egorova_jodar_2014} for call and put options with underlying assets featuring a continuous dividend yield. Moreover, we implemented explicit and implicit schemes for Nielsen et al.'s method \cite{nielsen_2001}, an explicit scheme for Company et al.'s method \cite{company_egorova_jodar_2014}, and the theta method for the linear complementary problem proposed by \cite{wilmott_howison_dewynne_1995} using Python. Finally, we derived the order of convergence for each of the implemented methods.

\subsection{Outline}

The outline of this paper is as follows. In Section 2, we explore the Black-Scholes model for American options for assets that pay dividends, resulting in the free boundary formulation of the pricing problem. Furthermore, we delve into the front-fixing method as a strategy for fixing the moving boundary by applying a change of variable. We consider the changes of variables proposed by Company et al. \cite{company_egorova_jodar_2014} and Nielsen et al. \cite{nielsen_2001}. In Section 3, we explore explicit and implicit schemes to solve the partial differential equations resulting from applying the front-fixing method and the Nielsen transformation to the free boundary problem obtained in Section 1. We conclude Section 3 with numerical experiments and convergence analysis for the numerical schemes presented based on the work of Nielsen and Company (see Appendix \ref{sec:company_explicit_scheme}). In Section 5, we explore a reformulation of the pricing problem as a variational inequality and the linear complementary system of equations resulting from it. Additionally, we delve into the theta method, which is a more general numerical method that yields explicit, implicit, and Crank-Nicholson schemes. Finally, in the same section, we discuss the results of solving the linear complementary problem reformulation of pricing American options using the theta method.
