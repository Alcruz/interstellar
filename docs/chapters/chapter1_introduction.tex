\section{Introduction}

\subsection{Overview}

Options are equity-based derivates which are mainly used to mitigate risk. The option market is significantly big compared to other derivatives. In fact, options were the most traded derivates in 2019 with a volume of 18.55 billion contracts when combining index and individual equities contracts \cite{statista_2019}. An enormous market, such as the option market, demands for reliable pricing mechanisms, so that arbitrage opportunities are minimized. Naturally, pricing American options is a wide research area because there is not a closed-form solution to PDE resulting from Black-Scholes model. Merton \cite{merton_1973} was the first to consider the Black-Scholes model proposed by Black, and Scholes \cite{black_scholes_1973} to price European and American options. Although, in his work, Merton derived a nice formula to price European options, he stated that, in general, a closed-from solution was not attainable for American options. In 1977, Schwartz \cite*{schwartz_197779} and \cite{brennan_1997} proposed using finite difference schemes  to solve the price problem for American options. The work of Merton and Schwartz served as foundation for the free boundary problem for American options pricing. The motivation behind the free boundary problem formulation is that for American options there exists the optimal exercise price that marks the boundary between the region where exercising the option is profitable and the region where is not. Moreover, this optimal exercise price changes with time making impossible for holder to determine when to exercise. Based on the work of Landau \cite {landau_1950_heat_ci}, Wu et al. \cite{wu1997front} formulated the front fixing method as an approach to solve the free boundary problem for options in which the Landau transformation is used to transform the optimal exercise price or the moving boundary to fixed boundary. Since, multiple transformation has been proposed by Huang et al. \cite*{huang_2000}, Nielsen et al. \cite*{nielsen_2001}, and Company et al. \cite*{company_egorova_jodar_2014}. Around the period, Dewynne et al. \cite*{dewynne_howison_rupf_wilmott_1993}, took a different approach to solve the pricing problem. The idea was to reformulate the free boundary problem as a problem with variational inequalities that, when finite difference is applied, transform to a linear complementary problem \cite{dantzig_1968}.

\subsection{Aim} 

The goal of this work is to implement the numerical methods proposed by Nielsen, et al. \cite{nielsen_2001}, and Company, et al. \cite{company_egorova_jodar_2014} to solve the free boundary formulation of the pricing problem for American options. \cite {dewynne_howison_rupf_wilmott_1993}. Moreover, Company et al. \cite {company_egorova_jodar_2014} and Nielsen et al. \cite*{nielsen_2001} proposed schemes for pricing American put options with non-paying dividends underlies. However, we would like derived analogous schemes for pricing call contracts, as well as, considering underlying assets that have a continuous dividend yield such as the S\&P500. Finally, we want to a convergence analysis of the  Additionally, we also consider linear complementary formulation proposed by Dewynne \cite{dewynne_howison_rupf_wilmott_1993} \cite{wilmott_howison_dewynne_1995}. Finally, we do convergence analysis of each of the methods

\subsection{Main Achievements}

In this work, we were able to derive corresponding the PDE problem resulting from applying the transformations proposed by Nielsen, et al. \cite{nielsen_2001} and Company, et al. \cite{company_egorova_jodar_2014} for call and put options with underlies with a continuous dividend yield. Moreover, we implement explicit and implicit schemes for Nielsen, et al. \cite{nielsen_2001}, explicit scheme for \cite{company_egorova_jodar_2014}, and theta method for the linear complementary problem proposed by \cite{wilmott_howison_dewynne_1995} using python. Finally, we were able derived order of convergence of each of the methods implemented.

\subsection{Outline}

The outline of this paper is the following. In section 2, we explore the  Black-Scholes model for American options for assets that pay dividends which results in the free boundary formulation of the pricing problem. Moreover, we explore the front fixing method as a strategy for fixing the moving boundary by applying a change of variable. We considered the changes of variable proposed by Company, et al. \cite*{company_egorova_jodar_2014} and Nielsen, et al. \cite*{nielsen_2001}. In section 3, we explore an explicit and implicit schemes to solve partial differential equations that  result from applying the front fixing method and the Nielsen transformation to the free boundary problem obtained in section 1. We conclude section 3 with numerical experiments and convergence analysis for the numerical schemes presented base on the work of Nielsen and Company (appendix \ref{sec:company_explicit_scheme}). In section 5, we explore a reformulation of the pricing problem as a variational inequality and the linear complementary system of equation resulted from it. Moreover, we explore the theta method which a more general numerical method that yields to explicit, implicit and Crank-Nicholson schemes. Finally, in the same section, we discuss the results of solving the linear complementary problem reformulation of pricing American options using the theta method.

