\section{Introduction}

\subsection{Overview}

Options are equity-based derivates which are mainly used to mitigate risk. 
The option market is significantly big compared to other derivatives. 
In fact, options were the most traded 
derivates in 2019 with a volume of 18.55 billion contracts when combining 
index and individual equities contracts \cite{statista_2019}. 
An enormous market, such as the option market, demands for reliable pricing 
mechanisms, so that arbitrage opportunities are minimized. Naturally, pricing 
American options is a wide research area because there is not a closed-form
solution to PDE resulting from Black-Scholes model. In this work,
we consider the free boundary and the linear complementary formulations of the 
pricing problem for American options. To solve the free boundary formulation,
we apply the front fixing method along with two transformations that allow to fix
the moving boundaries in the original problem. Alternatively, to solve the linear complementary problem, we apply the PSOR 
method. Moreover, for both the front-fixing and PSOR method, we present explicit 
and implicit central finite difference schemes. Finally, we conduct numerical 
experiments that allow us to compare the trade-offs and properties of each of 
the methods.

\subsection{Brief history}

Merton \cite{merton_1973} was the first to consider the Black-Scholes model 
proposed by Black, and Scholes \cite{black_scholes_1973} to price European
and American options. Although, in his work, Merton derived a nice formula 
to price European options, he stated that, in general, a closed-from solution 
was not attainable for American options. In 1977, Schwartz \cite*{schwartz_197779}
and \cite{brennan_1997} proposed using finite difference schemes 
to solve the price problem for American options. The work of Merton and Schwartz
served as foundation for the free boundary problem for American options pricing.
Based on the work of Landau \cite*{landau_1950_heat_ci}, 
Wu et al. \cite{wu1997front} formulated the front fixing mention as an approach 
to solve the free boundary problem for options in which the Landau transformation
is used to transform the moving boundary to fixed boundary. Since, multiple transformation has 
been proposed by Huang et al. \cite*{huang_2000}, Nielsen et al. \cite*{nielsen_2001},
and Company et al. \cite*{company_egorova_jodar_2014}. During the same period, 
Dewynne et al. \cite*{dewynne_howison_rupf_wilmott_1993} proposed the linear 
complementary formulation of the pricing problem and introduced the PSOR method.

\subsection{Motivation}

In their work, Company et al. \cite*{company_egorova_jodar_2014} and Nielsen 
et al. \cite*{nielsen_2001} proposed changes of variables for solving the free
boundary problems for American put options with non-paying dividends underlies.
 As part of our work, we consider a price model in which
the assets have continuous dividends yield. Such assets are found in index
equity derivates. Consequently, we present methods for solving the price problem
for both put and call options. Moreover, Company et al. \cite*{company_egorova_jodar_2014} 
proposed an explicit finite difference scheme. Therefore, we formulate an implicit 
finite difference scheme based on the transformation proposed in 
\cite*{company_egorova_jodar_2014}. Additionally, we solve the linear complementary
formulation for the pricing problem proposed in \cite*{seydel_2009} and 
\cite*{wilmott_howison_dewynne_1995}. Similarly, we consider a price model in which
assets pay out dividends.

\subsection{Aim} 

The goal of this work is to implement the numerical methods proposed by 
Nielsen, et al. \cite{nielsen_2001}, Company, et al. \cite{company_egorova_jodar_2014}
and Seydel \cite*{seydel_2009} to price American options for underlying assets 
that pay out with dividends. Additionally, we compare how well these methods approximate 
the price using as a reference the binomial model which is a widely used method 
for pricing American options in the industry. Finally, we explore the 
trade-off in terms of CPU time and memory, and do convergence analysis for each method.
\subsection{Main results}

\subsection{Outline}

The outline of this paper is the following. In section 2, we explore the 
Black-Scholes model for American options which results in the free boundary 
formulation of the pricing problem. Moreover, we explore the front fixing method
as a strategy for fixing the moving boundary by applying a change of variable. 
We considered the changes of variable proposed by Nielsen, et al. \cite*{nielsen_2001} 
and Company, et al. \cite*{company_egorova_jodar_2014}. In section 3, we explore
an explicit and implicit schemes to solve partial differential equations that 
result from applying the front fixing method to the free boundary problem. 
In section 4, we discuss the results obtained from pricing American options using
explicit and implicit methods given in section 3 and appendix 1. In section 5, 
we explore a reformulation of the pricing problem as the obstacle problem and 
the linear complementary system of equation resulted from it. Moreover, we 
explore the theta method that combines both explicit and implicit numerical 
scheme to solve obstacle problem. Finally, in the same section, we discuss the 
results of pricing American options using the theta method to solve the linear 
complementary problem.

