\section{Conclusions}

An option provides the right to buy or sell an asset at a predetermined strike price in the future. Generally, investment firms are responsible for writing these contracts and selling them to investors. Subsequently, investors hold these contracts to hedge against potential changes in price. When a holder already owns an asset and wants to hedge against a potential price drop, they enter into a put option. On the contrary, if a holder aims to hedge against an increase in the price of an asset they intend to acquire, they enter into a call option contract. A wide variety of options are available in the market. Among the most notorious contracts, we have European and American options. European options are contracts that can exercise at the expiration date only. Likewise, American options are contracts that can be exercised before or at the maturity date. 

Investment firms charge premiums to investors for entering an option contract. Then, the writer uses the premium to hedge the possible claims that the holder will have in the future. Charging the correct premium is important because it minimizes the arbitrage opportunities for either the holder and writer. Clearly, pricing schemes depend on the type of the contract. In general, The Black-Scholes formula is used to price European options. Sadly, no formula is available for American options. Therefore, firms rely on numerical methods to come up with some approximation of the price. Numerous numerical methods for pricing American options derive from the Black-Scholes PDE. 

In this report we have implemented, and analyzed numerical schemes derived from the free boundary and the variational inequalities' formulation of the pricing problem. Specifically, we have discussed: An explicit and implicit front fixing schemes for solving the free boundary problem based on the Nielsen transformation, an explicit front fixing scheme based on the Company transformation, and the explicit, implicit and Crank-Nicholson PSOR schemes.

The explicit front fixing schemes were derived from applying central finite difference and forward/backward difference. However, they differ in how the contact point condition is approximated. Specifically, The Nielsen front fixing schemes use forward difference (or backward difference for call options) to approximate the contact point condition while the Company explicit scheme use central finite difference. This explains why Nielsen front fixing schemes yielded first order convergence with respect to the spatial discretization parameter $\Delta{x}$ while Company explicit scheme yielded second order. Moreover, as it is normally the case for explicit and implicit central finite differences, both Nielsen and Company schemes exhibits first order convergence with respect to the temporal discretization parameter $\Delta{t}$. While the explicit front fixing schemes for Nielsen and Company transformation are conditionally stable, they both proved to be substantially faster and much more accurate than the implicit scheme. 

We discourage the use of the Nielsen implicit scheme. As we already told, the implicit scheme is less accurate by far. The reason behind this is that the implicit scheme requires to solve a nonlinear system of equation at each time step in the grid. Moreover, the approximation errors produced by the nonlinear solver get accumulated over time, affecting the overall accuracy of the method. We could decrease the approximation error in the nonlinear solver by decreasing its tolerance, and increasing its maximum number of iteration. However, we found that as we do that, the overall performance of the method reduces substantially. To make things even worse, the size of the nonlinear equations is inversely proportional to the spatial discretization parameter. Therefore, the computational resources required by the implicit method grows substantially as we decrease the spatial discretization parameters. For instance, for a grid of $M$ nodes in the spatial direction, the non-linear solver needs to invert a Jacobian matrix of $M\times M$ entries. In other words, decreasing the spatial discretization parameters by a decimal point, requires 100 times more memory. To summarize, we can increase the accuracy of implicit method by decreasing the tolerance of the non-linear solver and by decreasing the discretization parameters of the grid but by sacrificing the performance of the method and increasing the memory consumption substantially.

Similar to the Company front fixing schemes, the PSOR schemes showed to have second order convergence with respect to the spatial discretization parameter $\Delta{x}$. Moreover, it showed to have first order convergence for the explicit ($\theta=0$) and implicit ($\theta=1$) schemes, and second order convergence for the Crank-Nicholson ($\theta=0.5$) in with respect to the temporal discretization parameter $\Delta{t}$. Analogous to the explicit front fixing schemes, the PSOR explicit scheme is conditionally stable. When comparing the performance of the explicit PSOR to the explicit front fixing schemes, the explicit front fixing schemes showed to be substantially faster. In that regard, the issue with explicit PSOR is that as you decrease $\Delta{x}$, the complementary equations grow, hence, taking more time to solve the linear complementary problem. Similar argument can be done when comparing explicit front fixing schemes to the implicit and Crank-Nicholson PSOR schemes. In spite of that, the Crank-Nicholson PSOR scheme is second order convergence in time, therefore, by choosing $\Delta{x}$ to be smaller than $\Delta{t}$, we could have greater performance maintaining the accuracy. 

Concluding, the numerical experiments conducted showed that the explicit front-fixing schemes offers superior performance and accuracy than the implicit front fixing scheme and the all the PSOR schemes. Between the Company explicit front fixing scheme and the Nielsen explicit front fixing scheme, we recommend using the Company transformation because it has second order convergence in space, hence, yielding to smaller approximation errors. Similarly, the explicit PSOR scheme exhibited smaller approximation errors and computational than the implicit and Crank-Nicholson PSOR schemes and the implicit front fixing scheme. However, it is still rather slow compared to the explicit front fixing schemes.

\section{Further research}

Further research opportunities arise from this report. Firstly, in others work\cite{dewynne_howison_wilmott_howison_1995}\cite{seydel_2009}\cite{wu1997front}, it is common to transform the Black-Scholes PDE to the heat diffusion equation. Although this was done for the LCP problem, the front fixing schemes were derived within the financial domain which might led to worse approximations or slower performance. Moreover, we might modify the Nielsen front fixing schemes so that it approximates the contact point condition using central finite differences. Likewise, for the Nielsen implicit front fixing scheme, we could explore using nonlinear methods for large scale-scale nonlinear systems such as the one proposed by\cite{lacruz_2006}. Also, we might also derive Crank-Nicholson schemes for the Nielsen and Company front fixing schemes. Finally, we might consider using real market data to evaluate how good are the numerical schemes presented under real market conditions.
