
\section{Finite Difference schemes}

The Black-Scholes PDE can be transformed to heat diffusion PDE using the following
change of variables

\begin{align*}
  S &= Ke^x \\
  t &= T - \frac{2\tau}{\sigma^2} \\ 
  q &:= \frac{2r}{\sigma^2} \\
  q_{\delta} &:= \frac{2(r-\delta)}{\sigma^2} \\
  \alpha &:= \frac{1}{2}(q_{\delta} - 1) \\
  \beta &:= \frac{1}{4}(q_{\delta} - 1)^2 + q \\
  v(x, \tau) &:= e^{-(\alpha x + \beta \tau)}y(x, \tau)= V(S, t)
\end{align*}

The system (\ref{eq:background:finance:free_boundary_problem}) 
is the free boundary formulation for the pricing problem for American options.
A detailed derivation of (\ref{eq:background:finance:american_options_pde}) 
can be found at [REFERENCES].


The equation (\ref*{eq:background:finance:american_options_pde}) 
is a paraboblic PDE. Moreover, by applying the transformation,


the equation (\ref*{eq:background:finance:american_options_pde}) converts 
to the heat diffusion PDE.

\begin{equation}
  h(x, \tau) := \frac{H(S, t)}{K} = \begin{cases}
    \max(e^{x} - 1, 0)\\
    \max(1 - e^{x}, 0)
  \end{cases} 
\end{equation}

\begin{equation}
  \bar{x}(\tau) := \log{\bar{S}(t)} - \log{K} 
\end{equation}

\begin{align}
  \begin{cases}
    \frac{\partial y}{\partial \tau} = \frac{\partial^2 y}{\partial x^2} & \text{for $\tau\in[0,\frac{\sigma^2}{2}T)$ and $x\in(\bar{x}(t), \infty)$} \\
    y(x, \tau) = e^{(\alpha x + \beta \tau)}h(x, \tau) & \text{for $\tau\in[0, \frac{\sigma^2}{2}T]$ and $x\in(-\infty, \bar{x}(\tau)]$} \\
    \bar{x}(0) = 0
  \end{cases}
  \label{eq:background:finance:american_option_heat_equation}
\end{align}

We can reformulate equation (\ref*{eq:background:finance:american_option_heat_equation})
as:

\begin{equation}
  g := e^{\alpha x + \beta \tau}h(x, \tau)
\end{equation}

\begin{align}
  \begin{cases}
    \big(\frac{\partial y}{\partial \tau} - \frac{\partial^2 y}{\partial x^2}\big)(y  - g) =0 \\
    \frac{\partial y}{\partial \tau} - \frac{\partial^2 y}{\partial x^2} \ge 0 \quad y - g \ge 0 \\
    y(x, 0) = g(x, 0)
  \end{cases}
\end{align}


By exploring the geometric properties of the value function $V(S,t)$, 
we can determine useful conditions that will later help on in solving the equation 
(\ref*{eq:background:finance:american_options_pde}). Firstly, at
any given time $0 \le t \le T$, American options match the linear segment of the payoff
function within the stopping region. Therefore, we could say that 

\begin{align}
  \frac{\partial V}{\partial S}(S, t) =  \begin{cases}
    -1 & \text{(put)} \\ 
    1 & \text{(call)}
  \end{cases}
  \label{eq:background:finance:american_option_left_boundary}
\end{align}

Moreover, as the price goes to infinity the value of the option tends to zero

\begin{align}
  \lim_{S \rightarrow \infty}V(S, t) = 0 
  \label{eq:background:finance:american_option_stopping_right_boundary}
\end{align}



Pricing American options requires using numerical methods. The Black-Scholes PDE 
in (XXX) can be converted to the heat diffusion equation


Therefore, we focus on analyzing the numerical solution of the heat diffusion equation. 
Suppose the equation (XXX) is defined within the rectangular region $[x_{\text{min}}, x_{\text{max}}]\times[\tau_{\text{min}}, \tau_{\text{max}}]$.
By discretizing uniformly along the spatial direction $x$ and temporal direction $\tau$,

\begin{align}
  M &:= \frac{x_{\text{max}} - x_{\text{min}}}{\Delta x} \\ 
  N &:= \frac{\tau_{\text{max}} - \tau_{\text{min}}}{\Delta \tau} \\ 
  x_i &:= x_{\text{min}} + i\Delta x & \text{for $i = 0,\dots, M$} \\
  \tau_i &:= \tau_{\text{min}} + i{\Delta \tau} & \qquad \text{for $i = 0,\dots, N$}
\end{align}

where $\Delta x$ and $\Delta \tau$ are the distance between two consecutive points,
then, the grid is defined as the discrete region. 

\begin{align}
  \mathcal{G} := \{(x_i, \tau_j): (i, j) \in \{0,\dots,M\}\times\{0,\dots,N\}\}
\end{align}

Thus, solving numerically the equation (XXX) means finding an approximation for $y(x_i, \tau_n)$ at every 
$(x_i, \tau_n)$ within the grid $\mathcal{G}$,

\begin{align}
  y^{n}_i \approx y(x_i,\tau_n)
\end{align}

To obtain such approximation, we rely on central difference approximations (REFERENCE).

\begin{align}
  f'(x_i) &= \frac{f_{i+1} - f_{i}}{h} + O(h) \\
  f'(x_i) &= \frac{f_{i+1} - f_{i-1}}{2h} + O(h^2) \\
  f''(x_i) &= \frac{f_{i+1} - 2f_{i} + f_{i-1}}{h^2} + O(h^2)
\end{align}

\subsection{Explicit scheme}

An explicit scheme is one where we approximate the time partial derivative using
a forward difference approximation. Hence, the PDE in (XXX) is approximated as

\begin{equation}
  \frac{y^{n+1}_{i} - y^{n}_{i}}{\Delta \tau} = \frac{y^{n}_{i-1} - 2y^{n}_{i} + y^{n}_{i+1}}{(\Delta x)^2}
\end{equation}

By rearranging the terms,

\begin{equation}
  \lambda := \frac{\Delta \tau}{(\Delta x)^2}
\end{equation}

\begin{equation}
  y^{n+1}_i = \lambda y^{n}_{i-1} + (1 - 2\lambda)y^{n}_{i} + \lambda y^{n}_{i+1}
\end{equation}

It is shown by reference [REFERENCE] that method (XXX) is stable and consistent 
under the following condition

\begin{equation}
  0 < \Delta \tau \le \frac{(\Delta x)^2}{2}
\end{equation}

Moreover, the method has order of convergence $O(\Delta \tau, (\Delta x)^{2})$.

\subsection{Implicit scheme}

The implicit scheme approximates the time derivative using a backward difference

\begin{equation}
  \frac{y^{n+1}_{i} - y^{n}_{i}}{\Delta \tau} = \frac{y^{n+1}_{i-1} - 2y^{n+1}_{i} + y^{n+1}_{i+1}}{(\Delta x)^2}
\end{equation}

\begin{equation}
  y^{n+1}_{i} - \lambda (y^{n+1}_{i-1} - 2y^{n+1}_{i} + y^{n+1}_{i+1}) = y^{n}_{i}  
\end{equation}

\begin{equation}
  K := \begin{bmatrix}
    2 & -1     & & 0 \\ 
   -1 & \ddots & \ddots \\
      & \ddots & \ddots & \ddots \\
    0 & & \ddots & \ddots & \\
  \end{bmatrix} 
\end{equation}

\begin{equation}
  (I + \lambda K)\boldsymbol{y}^{n+1} = \boldsymbol{y}^{n}
\end{equation}

\subsection{Theta method}

\begin{equation}
  \frac{y^{n+1}_{i} - y^{n}_{i}}{\Delta \tau} = (1-\theta)\frac{y^{n}_{i-1} - 2y^{n}_{i} + y^{n}_{i+1}}{(\Delta x)^2} +  \theta\frac{y^{n+1}_{i-1} - 2y^{n+1}_{i} + y^{n+1}_{i+1}}{(\Delta x)^2}
\end{equation}

\begin{equation}
  y^{n+1}_{i} - \lambda\theta(y^{n+1}_{i-1} - 2y^{n+1}_{i} + y^{n+1}_{i+1}) =  y^{n}_{i} + (1-\theta)\lambda(y^{n}_{i-1} - 2y^{n}_{i} + y^{n}_{i+1})
\end{equation}

\begin{equation}
  (1 + \lambda\theta K)\boldsymbol{y}^{n+1} = (1-\lambda\theta K)\boldsymbol{y}^{n} 
\end{equation}