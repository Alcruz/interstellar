
\section{Explicit scheme for Company transformation} \label{sec:company_explicit_scheme}
In this section, we will briefly derive the explicit scheme for the Company approximation.\cite{company_egorova_jodar_2014}. Similar as in section \ref{sec:finitediferencesschemes:explicit_scheme}, the explicit front fixing Company scheme is obtained by approximating the second order spatial derivative using central finite difference at position $x_i$. Contrary to section \ref{sec:finitediferencesschemes:explicit_scheme},  the first order spatial derivative using forward difference at time step $t_n$
\begin{equation*}
    \begin{split}
      \dfrac{v^{n+1}_{i} - v^{n}_{i}}{\Delta{t}} & - \dfrac{1}{2}\sigma^2 \dfrac{v^{n}_{i-1} - 2v^{n}_{i} + v^{n}_{i+1}}{(\Delta{x})^2} \\ 
       & - \bigg( (r-\delta) - \dfrac{\sigma^2}{2} - \dfrac{\bar{S}^{n+1}_{f} - \bar{S}^{n}_{f}}{\Delta{t}\bar{S}^{n}_{f}} \bigg)\dfrac{v^{n}_{i+1} - v^{n}_{i-1}}{2\Delta{x}} + rv^{n}_{i} = 0
    \end{split}
\end{equation*}
for $i=1\dots,M$ and $n = 0,\dots,N$. Similarly, we rewrite the expression above by writing the approximation of time step $v^{n+1}_i$ with respect to the approximation at the current time step $t_{n}$
\begin{equation}
    \label{eq:appendix:comapany:explicit_scheme}
    \begin{split}
        v^{n+1}_{i} = av^{n}_{i-1} + bv^{n}_{i} + cv^{n}_{i+1} + \dfrac{\bar{S}^{n+1}_{f} - \bar{S}^{n}_{f}}{2\Delta{x}\bar{S}^{n}_{f}}(v^{n}_{i+1} - v^{n}_{i-1})
    \end{split}
\end{equation}
where 
\begin{equation}
    \label{eq:appendix:comapany:explicit_scheme_terms}
    \begin{split}
        \lambda =& \dfrac{\Delta{t}}{\Delta{x}^2}\\
        a =& \dfrac{\lambda}{2}\bigg( \sigma^2 - \bigg(r - \delta - \dfrac{\sigma^2}{2}\bigg)\Delta{x} \bigg) \\
        b =& 1 - \sigma^2\lambda- r\Delta{t} \\
        c =& \dfrac{\lambda}{2}\bigg(\sigma^2 + \bigg(r - \delta - \dfrac{\sigma^2}{2}\bigg)\Delta{x}\bigg)
    \end{split}
\end{equation}
The explicit scheme \eqref{eq:appendix:comapany:explicit_scheme} have boundary conditions (See section \ref{sec:blackscholes:company_transformation})
\begin{equation*}
    \begin{split}
        \text{\textbf{Call:}} \qquad & v^{n}_{0} = 0, \quad v^{n}_{M+1} = \bar{S}^{n}_{f} - 1 \\
        \text{\textbf{Put:}} \qquad & v^{n}_{0} = 1 - \bar{S}^{n}_{f}, \quad v^{n}_{M+1} = 0 \\
    \end{split}
\end{equation*}
for $n=0,\dots,N$, and initial condition 
\begin{equation*}
    \begin{split}
        \text{\textbf{Call:}} \qquad & v^0_i  = 0 \\
        \text{\textbf{Put:}} \qquad & v^0_i = 0 \\
    \end{split}
\end{equation*}
Moreover, the contact point condition \ref{eq:blackscholes:frontfixingmethod:company:contact_point_condition} is approximated using central finite difference
\begin{align*}
    \text{\textbf{Call:}} \qquad & \dfrac{v^{n}_{M+2} - v^{n}_{M}}{2\Delta{x}^2} = \dfrac{\partial{v}}{\partial{x}}(0^-, t) + O(\Delta{x}^2) \\
    \text{\textbf{Put:}} \qquad & \dfrac{v^{n}_{1} - v^{n}_{-1}}{2\Delta{x}} = \dfrac{\partial{v}}{\partial{x}}(0^+, t)+ O(\Delta{x}^2) 
\end{align*}
for $n=0,\dots,N$. Moreover, the contact point condition is approximated using central finite difference
\begin{align}
    \label{eq:appendix:explicitmethodcompany:contact_point_approximation}
    \text{\textbf{Call:}} \qquad & \dfrac{v^{n}_{M+2} - v^{n}_{M}}{2\Delta{x}^2} = \bar{S}^{n}_{f} \\
    \text{\textbf{Put:}} \qquad & \dfrac{v^{n}_{1} - v^{n}_{-1}}{2\Delta{x}}  = -\bar{S}^{n}_{f}  
\end{align}
Using central finite difference for approximating the contact point condition has the benefit that our scheme will be second order in space. However, we must come up with some way to approximate replace $v_{M+2}$ and $v_{-1}$ in \eqref{eq:appendix:explicitmethodcompany:contact_point_approximation}. By plug the contact condition in \eqref{eq:blackscholes:frontfixingmethod:nielsen:american_options_pde} we have that 
\begin{align}
    \label{eq:appendix:explicitmethodcompany:contact_point_approximation}
    \text{\textbf{Call:}} \qquad & \dfrac{\sigma^2}{2}\dfrac{\partial^2{v}}{\partial{x^2}}(0^-, \tau) - \dfrac{\sigma^2}{2}\bar{S}_f(\tau) + r = 0 \\
    \text{\textbf{Put:}} \qquad & \dfrac{\sigma^2}{2}\dfrac{\partial^2{v}}{\partial{x^2}}(0^+, \tau) + \dfrac{\sigma^2}{2}\bar{S}_f(\tau) - r = 0
\end{align}
and by applying using central finite difference we have
\begin{align}
    \label{eq:appendix:explicitmethodcompany:contact_point_approximation_2}
    \text{\textbf{Call:}} \qquad & \dfrac{\sigma^2}{2}\dfrac{v^{n}_{M} - 2v^{n}_{M+1} + v^{n}_{M+2}}{\Delta{x}^2} - \bigg(\dfrac{\sigma^2}{2}-\delta\bigg)\bar{S}^n_f + r = 0 \\
    \text{\textbf{Put:}} \qquad & \dfrac{\sigma^2}{2}\dfrac{v^{n}_{-1} - 2v^{n}_{0} + v^{n}_{1}}{\Delta{x}^2} + \bigg(\dfrac{\sigma^2}{2}-\delta\bigg)\bar{S}^n_f -r = 0
\end{align}
and by plug equation \eqref{eq:appendix:explicitmethodcompany:contact_point_approximation} in equation \eqref{eq:appendix:explicitmethodcompany:contact_point_approximation_2}, we have that 
\begin{align}
    \text{\textbf{Call:}} \qquad & v_{M} = \alpha - \beta\bar{S}^n_f \\
    \text{\textbf{Put:}} \qquad & v_{1} = \alpha - \beta\bar{S}^n_f \\
\end{align}
where
\begin{align}
    \label{eq:appendix:explicitmethodcompany:alpha_beta}
    \text{\textbf{Call:}} \qquad & \alpha := -\big(1 + \dfrac{r\Delta{x}^2}{\sigma^2}\big), \quad \beta := \Delta{x}-1-\dfrac{1}{2}\Delta{x}^2-\dfrac{\delta}{\sigma^2}\Delta{x}^2\\
    \text{\textbf{Put:}} \qquad & \alpha := 1 + \dfrac{r\Delta{x}^2}{\sigma^2}, \quad \beta := 1+\Delta{x}+\dfrac{1}{2}\Delta{x}^2+\dfrac{\delta}{\sigma^2}\Delta{x}^2
\end{align}
By combining all the equations above, we define algorithms for approximating $v(x,T)$ explicitly
\begin{algorithm}[H]
    \caption{Explicit method for call options}\label{alg:appendix:companytransformation:explicits:call_explicit_method_algorithm}
    \begin{algorithmic}
    \For{$i = 0,\dots,M+1$} 
      \State $v^{0}_i = 0 $
    \EndFor
    \State $\bar{S}_{f}^{0} = K$
    \State Define $a$, $b$, $c$ and $\lambda$ as in \eqref{eq:appendix:comapany:explicit_scheme_terms}
    \State Define $\alpha$ and $\beta$ as in \eqref{eq:appendix:explicitmethodcompany:alpha_beta}
    \For{$n = 0, \dots, N$}
      \State $d^n = \dfrac{\alpha - (av^{n}_{0} + bv^{0}_{1} + cv^{n}_{2} - (v^{n}_{M+1} - v^{n}_{M-1})/(2\Delta{x}))}{(v^{n}_{M+1} - v^{n}_{M-1})/(2\Delta{x}) + \beta\bar{S}^{n}_f}$
      \State $\bar{S}^{n+1}_{f}=d^{n}\bar{S}^{n}_{f}$
      \State $a^{n} = a - \dfrac{\bar{S}^{n+1}_{f} - \bar{S}^{n}_{f}}{2\Delta{x}\bar{S}^{n}_{f}}$
      \State $c^{n} = c - \dfrac{\bar{S}^{n+1} - \bar{S}^{n}}{2\Delta{x}\bar{S}^{n}_{f}}$
      \State $v^{n+1}_{0}= 0$
      \State $v^{n+1}_{M}=\alpha - \beta\bar{S}^{n+1}_{f}$
      \State $v^{n+1}_{M+1} =\bar{S}^{n+1}_{f} - 1$
      \For{$i = 1, \dots, M-1$}
        \State $v^{n+1}_{i} = a^{n} v^{n}_{i-1} + b v^{n}_{i} + c^{n}v^{n}_{i+1}$
      \EndFor
    \EndFor
  \end{algorithmic}
  \end{algorithm}

\begin{algorithm}[H]
    \caption{Explicit method for put options}\label{alg:appendix:companytransformation:explicits:put_explicit_method_algorithm}
    \begin{algorithmic}
    \For{$i = 0,\dots,M+1$} 
      \State $v^{0}_i = 0 $
    \EndFor
    \State $\bar{S}_{f}^{0} = K$
    \State Define $a$, $b$, $c$ and $\lambda$ as in \eqref{eq:appendix:comapany:explicit_scheme_terms}
    \State Define $\alpha$ and $\beta$ as in \eqref{eq:appendix:explicitmethodcompany:alpha_beta}
    \For{$n = 0, \dots, N$}
      \State $d^n = \dfrac{\alpha - (av^{n}_{0} + bv^{0}_{1} + cv^{n}_{2} - (v^{n}_{2} - v^{n}_{0})/(2\Delta{x}))}{(v^{n}_{2} - v^{n}_{0})/(2\Delta{x}) + \beta\bar{S}^{n}_f}$
      \State $\bar{S}^{n+1}_{f}=d^{n}\bar{S}^{n}_{f}$
      \State $a^{n} = a - \dfrac{\bar{S}^{n+1}_{f} - \bar{S}^{n}_{f}}{2\Delta{x}\bar{S}^{n}_{f}}$
      \State $c^{n} = c - \dfrac{\bar{S}^{n+1} - \bar{S}^{n}}{2\Delta{x}\bar{S}^{n}_{f}}$

      \State $v^{n+1}_{0}=1 - \bar{S}^{n+1}_{f}$
      \State $v^{n+1}_{1}=\alpha - \beta\bar{S}^{n+1}_{f}$
      \State $v^{n+1}_{M+1} = 0$
      \For{$i = 2, \dots, M$}
        \State $v^{n+1}_{i} = a^{n} v^{n}_{i-1} + b v^{n}_{i} + c^{n}v^{n}_{i+1}$
      \EndFor
    \EndFor
  \end{algorithmic}
  \end{algorithm}
\section{Python implementation}
The following python code is simplified version of the final implementation. A complete version can be found at \url{https://github.com/Alcruz/math4062-dissertation}.

\begin{lstlisting}[language=Python, caption=Base classses for options solver.]
    class Option(ABC):
        def __init__(self, type: OptionType, K: float, T: float):
            self.type = type
            self.K = K # Strike price
            self.T = T # Maturity

        @abstractmethod
        def payoff(self, S: float):
            pass
    class CallOption(Option):
        def __init__(self, K: float, T: float):
            super().__init__(OptionType.CALL, K, T)
        def payoff(self, S: float):
            return np.maximum(S - self.K, 0) 
    class PutOption(Option):
        def __init__(self, K: float, T: float):
            super().__init__(OptionType.PUT, K, T)
        def payoff(self, S: float):
            return np.maximum(self.K - S, 0)
    class Solver(ABC): 
        def __init__(self, 
            option: Option,
            r: float,  # risk-free interest rate
            sigma: float,  # sigma price volatitliy
            dx: float,  # grid resolution along x-axis
            dt: float,  # grid resolution along t-axis
            delta # dividends
        ) -> None:
            self.option = option
            self.r = r
            self.sigma = sigma
            self.dx = dx
            self.dt = dt
            self.delta = delta
\end{lstlisting}
\begin{lstlisting}[language=Python, caption=Explicit solver for Nielsen transformation.]
    class ExplicitSolver(Solver):
        def __init__(self, 
            option: Option,
            r: float,  # risk-free interest rate
            sigma: float,  # sigma price volatitliy
            dx: float,  # grid resolution along x-axis
            dt: float,  # grid resolution along t-axis
            delta # dividends
        ) -> None:
            self.option = option
            self.r = r
            self.sigma = sigma
            self.dx = dx
            self.dt = dt
            self.delta = delta
            lambd = dt / np.power(dx, 2)
            alpha = dt / dx
            self.A = 0.5 * np.power(sigma*self.x_axis, 2) * lambd - 0.5 * self.x_axis * ((r-delta) - (1/dt)) * alpha
            self.B = 1 - np.power(sigma*self.x_axis, 2) * lambd - r*dt
            self.C = 0.5 * np.power(sigma*self.x_axis, 2) * lambd + 0.5 * self.x_axis * ((r-delta) - (1/dt)) * alpha
        def solve(self):
            V = np.zeros_like(self.x_axis)
            S_bar = self.option.K
            for _ in np.arange(0, self.option.T, self.dt):
                D = 0.5*self.x_axis/self.dx
                D[1:-1] *= (V[2:]-V[:-2]) * (1/S_bar)
                
                S_bar = self.compute_time_iteration(V, D)
            return S_bar*self.x_axis, V, S_bar
        @abstractmethod
        def compute_time_iteration(self, V: np.ndarray, D: np.ndarray):
            pass
    class CallOptionExplicitSolver(ExplicitSolver):
        def __init__(self, 
            option: CallOption,
            r: float,  # risk-free interest rate
            sigma: float,  # sigma price volatitliy
            dx: float,  # grid resolution along x-axis
            dt: float,  # grid resolution along t-axis
            delta=0  # dividends
        ):
            self.x_axis = np.arange(0, 1+dx, dx)
            super().__init__(option, r, sigma, dx, dt, delta)
        def compute_time_iteration(self, V: np.ndarray, D: np.ndarray):
            S_bar = self.option.K + self.A[-2]*V[-3] + self.B[-2]*V[-2] + self.C[-2]*V[-1]
            S_bar /= 1 - self.dx - D[-2]
    
            V[1:-2] = self.A[1:-2]*V[:-3] + self.B[1:-2]*V[1:-2] \
                + self.C[1:-2]*V[2:-1] + D[1:-2]*S_bar
            V[-2] = (1-self.dx)*S_bar - self.option.K
            V[-1] = S_bar - self.option.K
            return S_bar
    class PutOptionExplicitSolver(ExplicitSolver):
        def __init__(self, 
            option: PutOption,
            r: float,  # risk-free interest rate
            sigma: float,  # sigma price volatitliy
            dx: float,  # grid resolution along x-axis
            dt: float,  # grid resolution along t-axis
            x_max: 2.,  # sufficiently large value for x
            delta=0  # dividends
        ):
            self.x_axis = np.arange(1, x_max+dx, dx)
            super().__init__(option, r, sigma, dx, dt, delta)
        def compute_time_iteration(self, V: np.ndarray, D: np.ndarray) -> float:
            S_bar = self.option.K - (self.A[1]*V[0] + self.B[1]*V[1] + self.C[1]*V[2])
            S_bar /= D[1] + 1 + self.dx
            V[2:-1] = self.A[2:-1]*V[1:-2] + self.B[2:-1] * \
                V[2:-1] + self.C[2:-1]*V[3:] + D[2:-1]*S_bar
            V[0] = self.option.payoff(S_bar)
            V[1] = self.option.payoff((1+self.dx)*S_bar)
            return S_bar
\end{lstlisting}

\begin{lstlisting}[language=Python, caption=Implicit solver for Nielsen transformation]
class ImplicitSolver(Solver):
    def __init__(self, 
        option: Option,
        r: float,  # risk-free interest rate
        sigma: float,  # sigma price volatitliy
        dx: float,  # grid resolution along x-axis
        dt: float,  # grid resolution along t-axis
        delta=0, # dividends
    ) -> None:
        self.option = option
        self.r = r
        self.sigma = sigma
        self.dx = dx
        self.dt = dt
        self.delta = delta
        self.lambd = self.dt/np.power(self.dx, 2)
        self.kappa = self.dt/self.dx
        self.alpha = 1 + self.lambd*np.power(self.sigma*self.x_axis, 2) + self.r*self.dt
        self.M = self.x_axis.size 
    def beta(self, S, S_bar):
        return -0.5*self.lambd*np.power(self.sigma, 2)*np.power(self.x_axis, 2) + 0.5*self.kappa*self.x_axis*((self.r-self.delta) - (S_bar - S)/(self.dt*S))
    def gamma(self, S, S_bar):
        return -0.5*self.lambd*np.power(self.sigma, 2)*np.power(self.x_axis, 2) - 0.5*self.kappa*self.x_axis*((self.r-self.delta) - (S_bar - S)/(self.dt*S))
    @abstractmethod
    def solve_non_linear_system(self, V, S_bar) -> float:
        pass
    def solve(self):
        S_bar = self.option.K
        V = np.zeros_like(self.x_axis)
        for _ in np.arange(0, self.option.T, self.dt):
            S_bar = self.solve_non_linear_system(V, S_bar)
        return S_bar*self.x_axis, V, S_bar
class CallOptionImplicitSolver(ImplicitSolver):
    def __init__(self, 
        option: Option,
        r: float,  # risk-free interest rate
        sigma: float,  # sigma price volatitliy
        dx: float,  # grid resolution along x-axis
        dt: float,  # grid resolution along t-axis
        delta=0, #dividends
        maxiter=1000,
        tolerance=1e-24,
        method='lm'
    ) -> None:
        self.x_axis = np.arange(0, 1+dx, dx)
        super().__init__(option, r, sigma, dx, dt, delta)
        self.maxiter=maxiter
        self.tolerance=tolerance
        self.method=method
    def jacobian(self, y, S_bar):
        p, s, = y[:-1], y[-1]
        beta = self.beta(s, S_bar)
        gamma = self.gamma(s, S_bar)
        dgamma_ds = - 0.5 * (1/self.dx) * self.x_axis * S_bar / s**2
        dbeta_ds = 0.5 * (1/self.dx) * self.x_axis * S_bar / s**2
        retVal = diags([self.alpha[2:], beta[1:-1], gamma[1:-1]],
                    [-1, 0, 1], shape=(self.M-2, self.M-2)).toarray()
        retVal[-1, :] = 0
        retVal[:, -1] = 0
        retVal[0, -1] = dbeta_ds[1]*p[0] + dgamma_ds[1]*p[1]
        retVal[1:-2, -1] = dbeta_ds[2:-3]*p[1:-1] + dgamma_ds[2:-3]*p[2:]
        retVal[-2, -1] = dbeta_ds[-3]*p[-2]
        retVal[-2, -1] -= dgamma_ds[-3]*self.option.payoff((1-self.dx)*s) + gamma[-3]*(1 - self.dx)
        retVal[-1, -2] = self.alpha[-2]
        retVal[-1, -1] -= dgamma_ds[-2]*self.option.payoff(s) + gamma[-2] + dbeta_ds[-2]*self.option.payoff((1-self.dx)*s) - beta[1]*(1-self.dx)
        return retVal
    def system(self, y, b, S_bar):
        *v, s = y
        beta = self.beta(s, S_bar)
        gamma = self.gamma(s, S_bar)
        A = diags([self.alpha[2:-1], beta[1:-2], gamma[1:-3]],
                [-1, 0, 1], shape=(self.M-2, self.M-3)).toarray()
        f = b[1:-1]
        f[-2] -= gamma[-3]*((1-self.dx)*s-self.option.K)
        f[-1] -= gamma[-2]*(s-self.option.K) + beta[-2]*((1-self.dx)*s - self.option.K)
        res = A@v - f
        return res
    def solve_non_linear_system(self, V: np.ndarray, S_bar: np.ndarray):
        *V[1:-2], S_bar = root(
            lambda y: self.system(y, np.copy(V[:]), S_bar),
            # jac=lambda y: self.jacobian(y, S_bar),
            x0=np.concatenate([V[1:-2], [S_bar]]),
            method=self.method,
            options=dict(xtol=self.tolerance, maxiter=self.maxiter)
        )['x']
        V[-2] = (1-self.dx)*S_bar-self.option.K
        V[-1] = S_bar-self.option.K
        return S_bar
class PutOptionImplicitSolver(ImplicitSolver):
    def __init__(self, 
        option: Option,
        r: float,  # risk-free interest rate
        sigma: float,  # sigma price volatitliy
        dx: float,  # grid resolution along x-axis
        dt: float,  # grid resolution along t-axis
        delta=0, # dividends
        x_max=2,
        maxiter=1000,
        tolerance=1e-24,
        method='lm'
    ) -> None:
        self.x_axis = np.arange(1, x_max+dx, dx)
        super().__init__(option, r, sigma, dx, dt, delta)
        self.maxiter=maxiter
        self.tolerance=tolerance
        self.method=method
    def jacobian(self, y, S_bar):
        p, s, = y[:-1], y[-1]
        beta = self.beta(s, S_bar)
        gamma = self.gamma(s, S_bar)

        dgamma_ds = - 0.5 * (1/self.dx) * self.x_axis * S_bar / s**2
        dbeta_ds = 0.5 * (1/self.dx) * self.x_axis * S_bar / s**2

        retVal = diags([beta[3:-1], self.alpha[2:-1], gamma[1:-1]],
                    [-2, -1, 0], shape=(self.M-2, self.M-2)).toarray()
        retVal[-1, :] = 0
        retVal[:, -1] = 0
        retVal[0, -1] = dgamma_ds[1]*p[0] + dbeta_ds[1] * \
            (self.option.K - S_bar) - beta[1] - self.alpha[1]*(1+self.dx)
        retVal[1, -1] = dgamma_ds[2]*p[1] + dbeta_ds[2] * \
            (self.option.K - (1+self.dx)*S_bar) - beta[2]*(1+self.dx)
        retVal[2:-1, -1] = dbeta_ds[3:-2]*p[:-2] + dgamma_ds[3:-2]*p[2:]
        retVal[-1, -3] = beta[-2]
        retVal[-1, -2] = self.alpha[-2]
        retVal[-1, -1] = dbeta_ds[-2]*p[-2]
        return retVal
    def system(self, y, b, S_bar):
        p, s, = y[:-1], y[-1]
        _beta = self.beta(s, S_bar)
        _gamma = self.gamma(s, S_bar)
        A = diags([_beta[3:-1], self.alpha[2:-1], _gamma[1:-2]],
                [-2, -1, 0], shape=(self.M-2, self.M-3)).toarray()
        f = b[1:-1]
        f[0] -= _beta[1]*(self.option.K-s) + self.alpha[1]*((self.option.K-(1+self.dx)*s))
        f[1] -= _beta[2]*(self.option.K-(1+self.dx)*s)
        res = A@p - f
        return res
    def solve_non_linear_system(self, V: np.ndarray, S_bar: np.ndarray):
        *V[2:-1], S_bar = root(
            lambda y: self.system(y, np.copy(V[:]), S_bar),
            # jac=lambda y: self.jacobian(y, S_bar),
            x0=np.concatenate([V[2:-1], [S_bar]]),
            method=self.method,
            options=dict(xtol=self.tolerance, maxiter=self.maxiter)
        )['x']
        V[0] = self.option.payoff(S_bar)
        V[1] = self.option.payoff((1+self.dx)*S_bar)    
        return S_bar
\end{lstlisting}
\begin{lstlisting}[language=Python, caption=Explicit solver for Company transformation.]
class ExplicitSolver(Solver):
    def __init__(
        self,
        r: float,  # risk-free interest rate
        sigma: float,  # sigma price volatitliy
        dx: float,  # grid resolution along x-axis
        dt: float,  # grid resolution along t-axis
        delta  # dividends
    ): 
        self.r = r
        self.sigma = sigma
        self.dx = dx
        self.dt = dt
        self.delta = delta    
        self.lambd = self.dt / np.power(self.dx,2)
        self.A = np.power(self.sigma, 2)
        self.A -= (self.r-self.delta-0.5*np.power(self.sigma,2))*self.dx
        self.A *= self.lambd / 2
        self.B = 1 - np.power(self.sigma, 2) * self.lambd - self.r*self.dt
        self.C = np.power(self.sigma, 2) 
        self.C += (self.r-self.delta-0.5*np.power(self.sigma,2))*self.dx
        self.C *= self.lambd / 2
    @abstractmethod
    def compute_time_iteration(self, p: np.ndarray, S_bar: float) -> float:
        pass
    def solve(self, option: Option):
            """Solve front-fixing method explicitly.

            Parameters:
                option (AmericanOption): the option to price.
                r (float): the stock price risk-free interest rate.
                sigma (float): the stock price volatility.
                x_max (float): large value used as infity in the spatial.
                dx (float): Grid resolution in the spatial direction.
                dt (float): Grid resolution in the time direction.
            """
            p = np.zeros_like(self.x_axis)
            S_bar = 1
            for _ in np.arange(0, option.T, self.dt):
                S_bar = self.compute_time_iteration(p, S_bar)
            S = S_bar * np.exp(self.x_axis)
            return option.K*S, option.K*p, S_bar
class CallExplicitSolver(ExplicitSolver):
    def __init__(
        self,
        r: float,  # risk-free interest rate
        sigma: float,  # sigma price volatitliy
        x_min: float,  # sufficiently large value for x
        dx: float,  # grid resolution along x-axis
        dt: float,  # grid resolution along t-axis
        delta=0  # dividends
    ):                 
        self.x_axis = np.arange(x_min, dx, dx)
        self.x_axis[-1] = 0
        super().__init__(r=r, sigma=sigma, dx=dx, dt=dt, delta=delta)
        self.alpha = -self.r*np.power(self.dx/self.sigma, 2) - 1
        self.beta = -1 + self.dx - 0.5*np.power(self.dx, 2) - np.power(self.dx/self.sigma, 2)*self.delta 
    def compute_time_iteration(self, p: np.ndarray, S_bar: float) -> float:
        d = self.alpha - (self.A*p[-3] + self.B*p[-2] + self.C*p[-1] - (p[-1]-p[-3])/(2*self.dx))
        d /= (p[-1]-p[-3])/(2*self.dx) + self.beta*S_bar
        S_bar_new = d*S_bar
        a = self.A - (S_bar_new - S_bar)/(2*self.dx*S_bar)
        c = self.C + (S_bar_new- S_bar)/(2*self.dx*S_bar)
        S_bar = S_bar_new
        p[1:-2] = a*p[:-3] + self.B*p[1:-2] + c*p[2:-1]
        p[-2] = self.alpha - self.beta*S_bar 
        p[-1] = S_bar - 1
        return S_bar_new
class PutExplicitSolver(ExplicitSolver):
    def __init__(
        self,
        r: float,  # risk-free interest rate
        sigma: float,  # sigma price volatitliy
        x_max: float,  # sufficiently large value for x
        dx: float,  # grid resolution along x-axis
        dt: float,  # grid resolution along t-axis
        delta=0  # dividends
    ):                 
        self.x_axis = np.arange(0, x_max+dx, dx)
        super().__init__(r=r, sigma=sigma, dx=dx, dt=dt, delta=delta)
        self.alpha = 1 + self.r*np.power(self.dx/self.sigma, 2)
        self.beta = 1 + self.dx + 0.5*np.power(self.dx, 2) + np.power(self.dx/self.sigma, 2)*self.delta 
    def compute_time_iteration(self, p: np.ndarray, S_bar: float) -> float:
        d = self.alpha - (self.A*p[0] + self.B*p[1] + self.C*p[2] - (p[2]-p[0])/(2*self.dx))
        d /= (p[2]-p[0])/(2*self.dx) + self.beta*S_bar
        S_bar_new = d*S_bar

        a = self.A - (S_bar_new - S_bar)/(2*self.dx*S_bar)
        c = self.C + (S_bar_new- S_bar)/(2*self.dx*S_bar)

        p[2:-1] = a*p[1:-2] + self.B*p[2:-1] + c*p[3:]
        p[1] = self.alpha - self.beta*S_bar_new
        p[0] = 1 - S_bar_new
        return S_bar_new    
\end{lstlisting}
\begin{lstlisting}[language=Python, caption=PSOR-LCP solver for Company transformation.]
class LCPSolver:
    def __init__(
        self,
        r: float, 
        sigma: float, 
        dx: float, 
        dt: float,
        x_min = -3.,
        x_max = 3.,
        theta = 0., 
        delta=0.
    ):
        self.r = r
        self.sigma = sigma
        self.dx = dx
        self.dt = dt
        self.x_min = x_min
        self.x_max = x_max
        self.theta = theta
        self.delta = delta
        self.x_axis = np.arange(self.x_min, self.x_max + self.dx, step=self.dx)
    @abstractmethod
    def get_g(self, x_axis, tao_axis, q_delta, q):
        pass
    def solve(
        self,
        option: Option,
        wR = 1,
        eps = 1e-24
    ):
        # Parameters artificail space
        q = (2*self.r) / np.power(self.sigma, 2)
        q_delta = 2*(self.r-self.delta) / np.power(self.sigma, 2)
        t_axis = np.arange(0, option.T+self.dt, step=self.dt)
        tao_axis = 0.5*np.power(self.sigma,2)*t_axis
        dtao = np.diff(tao_axis)[0]
        # calculate lambda and alpha
        lambd = dtao/np.power(self.dx, 2)
        alpha = lambd*self.theta
        g = self.get_g(self.x_axis, tao_axis, q_delta, q)
        # compute initial condition
        w = np.empty(self.x_axis.size)
        w = g[:, 0]
        b = np.empty_like(self.x_axis)
        v_new = np.empty_like(self.x_axis)
        for i in range(tao_axis.size - 1):
            b[2:-2] = w[2:-2] + lambd*(1-self.theta)*(w[3:-1] - 2*w[2:-2] + w[1:-3])
            b[1] = w[1] + lambd*(1-self.theta)*(w[2] -2*w[1] + g[0, i]) + alpha*g[0, i+1]
            b[-2] = w[-2] + lambd*(1-self.theta)*(g[-1, i]-2*w[-2] + w[-3]) + alpha*g[-1, i+1]
            v = np.maximum(w, g[:, i+1])
            while True:
                v_new[0] = v_new[-1] = 0
                for j in range(1, v.size - 1):
                    rho = b[j] + alpha*(v_new[j-1] + v[j+1])
                    rho /= 1+2*alpha
                    v_new[j] = np.maximum(g[j, i+1], v[j] + wR*(rho - v[j]))
                if np.linalg.norm(v_new - v, ord=2) <= eps:
                    break
                v = v_new.copy()
            w = v.copy()
        S_axis = option.K*np.exp(self.x_axis)
        S_axis[0] = 0
        V = self.get_surface(option, w, tao_axis, q, q_delta)
        return S_axis, V
class CallLCPSolver(LCPSolver):
    def __init__(self, r: float, sigma: float, dx: float, dt: float, x_min=-3, x_max=3, theta=0, delta=0):
        super().__init__(r, sigma, dx, dt, x_min, x_max, theta, delta)
    def get_g(self, x_axis, tao_axis, q_delta, q):
        return CallLCPSolver.GUtil(x_axis, tao_axis, q_delta, q)
    def get_early_exercise(self, option, S, V):
        eps = 1e-5
        S_bar = S[np.abs(option.K - S + V) <= eps][-1]
        return S_bar
    def get_surface(self, option, w, tao_axis, q, q_delta, ):
        V = option.K*w*np.exp(-(self.x_axis/2)*(q_delta - 1)) * np.exp(-tao_axis[-1]*((1/4)*np.power(q_delta - 1, 2) + q))
        V[0] = 0
        return V[:]      
    class GUtil: 
        def __init__(self, x, tao, q_delta, q):
            self.x = x
            self.tao = tao
            self.q_delta = q_delta
            self.q = q  
        def __getitem__(self, indx):
            xi, ti = indx
            return self.payoff(self.x[xi], self.tao[ti])
        def payoff(self, x, tao): return np.exp((tao/4)*(np.power(self.q_delta-1, 2) + 4*self.q)) * \
            np.maximum(np.exp((x/2)*(self.q_delta + 1)) - np.exp((x/2)*(self.q_delta - 1)), 0)
class PutLCPSolver(LCPSolver):
    def __init__(self, r: float, sigma: float, dx: float, dt: float, x_min=-3, x_max=3, theta=0, delta=0):
        super().__init__(r, sigma, dx, dt, x_min, x_max, theta, delta)
    def get_g(self, x_axis, tao_axis, q_delta, q):
        return PutLCPSolver.GUtil(x_axis, tao_axis, q_delta, q)
    def get_early_exercise(self, option, S, V):
        eps = 1e-5
        S_bar = S[np.abs(V + S - option.K) <= eps][-1]
        return S_bar
    def get_surface(self, option, w, tao_axis, q, q_delta):
        V = option.K*w*np.exp(-(self.x_axis/2)*(q_delta - 1)) * np.exp(-tao_axis[-1]*((1/4)*np.power(q_delta - 1, 2) + q))
        V[0] = option.K
        return V
    class GUtil: 
        def __init__(self, x, tao, q_delta, q):
            self.x = x
            self.tao = tao
            self.q_delta = q_delta
            self.q = q

        def __getitem__(self, indx):
            xi, ti = indx
            return self.payoff(self.x[xi], self.tao[ti])

        def payoff(self, x, tao): return np.exp((tao/4)*(np.power(self.q_delta-1, 2) + 4*self.q)) * \
            np.maximum(np.exp((x/2)*(self.q_delta - 1)) - np.exp((x/2)*(self.q_delta + 1)), 0)
\end{lstlisting}